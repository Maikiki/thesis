% !Mode:: "TeX:UTF-8"

\chapter{FST在自然语言处理上的应用}
有限状态传感器提供了一种方法,用于执行上下文相关的重写规则方面的数学运算,这种方法通常用于实现基础的自然语言处理任务。多种规则可能会组成一个单通的复杂规则,可以显着提高基于规则的系统的效率。
\section{有限状态机的快速入门}
一个有限状态机或者自动机是一种抽象的数学计算模型,可以存储当前的状态或者根据输入来改变这种状态。有限状态机由于本身有限的存储,虽然没有像图灵机之类的其他计算模型一样强大的能力,但也适用于许多电子建模、工程和语言处理方面的问题。一个有限状态机由一组各种各样表示系统状态的节点和这些节点之间代表从一种状态转移到另一种状态的带有标记的边组成,这些标记表示了这些状态转换的条件。一个输入流(一次输入包含一个串)会被有限状态机“处理”,造成一系列的状态转换。 下面的一个关于雪貂的精确计算模型就是一个简单的有限状态机实例: 

\pic[htbp]{雪貂FSM实例}{width=0.6\textwidth}{trans1}

我的雪貂“Pangur Bán”每天的行为,可以很好的被表示为这样一个串“累了,累了,累了,饿了,累了,累了,累了,无聊了,累了,累了,累了”。

\section{语言处理的有限状态机}
在语言处理中,一个有限状态机包含了个起始状态(节点)和一个最终状态可以用来生成或者识别一种由所有可能符号(条件标签)组合而成的语言,这些符号(条件标签)在每次从有限状态机初始状态到最终状态的过程中产生。有限自动机产生的语言类称为正则语言类。

\section{有限状态转移机}
有限状态转移机(FST)是一种特殊形式的有限状态机。特别的事,一个有限状态转移机有两种串,一个输入串和一个输出串。相比起对于某种输入串要么不改变,要么就接受条件做出转移的有限状态机来说,有限状态转移机将这种输入串翻译成了输出串。或者换种说法,它接受一个字符串的输入,然后输出另一个字符串。
\section{上下文相关规则}
FST在实现一些自然语言处理的任务中是十分有用的。上下文重写规则(比如ax$\rightarrow$bx)足够实现许多计算语言方面的任务,例如形态产生和词性标注。这些重写规则也等同于为FST提供了一种独特的路径用于优化基于规则的系统。

那么,举个例子,下面是几个有序的上下文规则:

1) 	如果紧跟在后面的是‘x’,把‘c’变成‘b’  cx$\rightarrow$bx

2) 	如果在前面的是‘rs’,把‘a’变成‘b’  	 rsa$\rightarrow$rsb

3) 	如果跟在前面的是‘rs’,跟在后面的是‘xy’,把‘b’变成‘a’ rsbxy$\rightarrow$rsaxy

所以当给定的输入字符串是rsaxyrscxy时,根据以上三个规则,我们就可以做出以下的变换:

1) rsaxyrscxy$\rightarrow$rsaxyrs  b  xy

2) rsaxyrsbxy$\rightarrow$rs  b  xyrsbxy

3) rsbxyrsbxy$\rightarrow$rs  a  xyrs  a  xy

提供一个上下文相关的规则所需要的时间取决于规则的数目,上下文容量的大小和输入字符串的字符数量。在这个转换的例子中效率明显非常低,通过多步变化需要将‘c’转变为‘b’然后将‘b’转变为‘a’通过规则1和3,通过规则2和3把‘a’转换成‘b’,又把‘b’转换成‘a’是多余的转换。有限状态转移机提供了一种消除这种低效工作的路径。不过首先我们需要将规则转变为状态机。
\section{将规则转换成有限状态转移机}
为了达到目的,我们首先讲过每个规则表示成一个有限状态转移机,每个状态之间的链接表示接受了输入字符,并输出相应的字符。这种输入/输出的组合是指在有限状态转移机标签杓袁的输入和输出特性之间用字符‘/’分隔。下面就是上述三种规则转变之后的有限状态转移机:

\pic[htbp]{FST}{width=0.8\textwidth}{trans2}

\section{扩展有限状态转移机}
当上面的有限状态转移机表示了我们上下文相关规则的集合时,他们将会很少用于匹配一个输入字符串,而是用于精确处理上下文窗口,描述他们相对应的规则。为了使每个有限状态转移机适用于任意长度的字符串并且在每次满足规则时都执行必要的转换,我们将需要扩展这些转移机。我们通过允许在每个状态接受任意可能的输入来达到这个目的。比如说,规则1必须能够处理字符串‘rsaxyrscxy’。当规则1匹配了字符串‘cx’而且输出了 字符串‘bx’之后,他必须在最后遇到‘c’和‘x’之前处理字符‘r’,‘s’,‘a’,‘x’,‘y’,‘r’和‘s’。然后在最后处理最后字符‘y’。每一个这些可能的字符(还有其他所有可能的字符)都应该明确的被列在它自己独立的边上,不过为了简化过程,我们可以建立一条独立的边,标注为‘?/?’,用于匹配和输出任何此状态下尚未被其他边表示的字符。有限状态转移机扩展规则中的尾部将需要使用只有输入没有输出的边(标注为‘ε’用于输出)和拥有多个输出的边。下面就是对上述有限状态转移机的扩展:

\pic[htbp]{扩展FST}{width=0.6\textwidth}{trans3}

\section{合并成单一的有限状态转移机}
合并运算是在两个有限状态转移机中可以执行的操作之一。合并操作需要使用两个确定的FST,A和B,然后将他们的节点和边联合成一个确定的有限状态转移机。有此产生的转移机能够接受一个任意长度的输入字符串然后输出一个字符串,这个字符串与先经过转移机A处理再经过转移机B处理得到的字符串一致。一个完整的有限状态转移机合并运算算法已经超出本文描写的范围了。下面是将FST1和FST2合并运算之后得到的FST再和FST3合并运算得到的FST:

\pic[htbp]{最终FST}{width=0.8\textwidth}{trans4}

注意,当应用最终有限状态转移机来处理输入字符串‘rsaxyrscxy’时,得到的输出结果已经完全等同于使用独立的上下文相关规则来处理同样字符串时得到的结果,转换只需要通过一次最终有限状态转移机,没有造成任何低效的转换。与耗时取决于规则的数目,上下文容量的大小和输入字符串字符数量的原始规则想比,最终有限状态转移机的耗时仅仅取决于输入字符串字符数量。
