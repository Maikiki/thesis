% !Mode:: "TeX:UTF-8"

\chapter{The Name of the Game}
\section{引言}
\subsection{引言}
Mybatis是一个支持自定义SQL查询、存储过程和高级映射的一流持久化框架。Mybatis消除了绝大部分JDBC代码和需要手工配置的参数及检索结果。Mybatis使用简单的XML文件或者JAVA注释来配置自身和原始映射,将接口和 Java 的 POJOs(Plan Old Java Objects,普通的 Java 对象)映射成数据库中的记录。
\section{入门}
每一个MyBatis应用围绕一个SqlSessionFactory实例。每个SqlSessionFactory实例可以通过SqlSessionFactoryBuilder获得。SqlSessionFactoryBuilder可以从XML配置文件或者一个自定义配置类的实例建立SqlSessionFactory实例。
\subsection{通过XML构建SqlSessionFactory}
通过一个XML文件来构建一个SqlSessionFactory实例很简单。建议你使用类路径下的资 源文件来配置,但是你可以使用任意InputStream实例来构建,包括由文字形式的文件路径 创建的实例或 URL 形式的文件路径 file://创建的实例。Mybatis包含一个名为Resources的工具类,它包含了一系列的方法,这些方法使得从类路径和其他位置加载资源文件更为简单。

\noindent
\ttfamily
\hlstd{\hllin{1\ }String\ resource\ }\hlopt{=\ }\hlstd{}\hlstr{"org/mybatis/example/mybatis{-}config.xml"}\hlstd{}\hlopt{;}\\
\hllin{2\ }\hlstd{InputStream\ inputStream\ }\hlopt{=\ }\hlstd{Resources}\hlopt{.}\hlstd{}\hlkwd{getResourceAsStream}\hlstd{}\hlopt{(}\Righttorque\\
\hllin{3\ }\hlstd{resource}\hlopt{);}\\
\hllin{4\ }\hlstd{sqlSessionFactory\ }\hlopt{=\ }\hlstd{}\hlkwa{new\ }\hlstd{}\hlkwd{SqlSessionFactoryBuilder}\hlstd{}\hlopt{().}\hlstd{}\hlkwd{build}\hlstd{}\hlopt{(}\Righttorque\\
\hllin{5\ }\hlstd{inputStream}\hlopt{);}\\
\hllin{6\ }\hlstd{} 
\mbox{}
\normalfont
\normalsize


XML配置文件包含了Mybatis系统的核心设置,其中包括用于获取数据库连接实例的数据源(DataSource),以及决定事务范围和控制的事务管理器(TransactionManager) 。这里展示一个简单的例子,文档后续内容将提供XML配置文件全部的详细说明。

\noindent
\ttfamily
\hlstd{}\hllin{1\ }\hlopt{\symbol{60}}\hlstd{?xml\ version}\hlopt{=}\hlstd{}\hlstr{"1.0"}\hlstd{\ encoding}\hlopt{=}\hlstd{}\hlstr{"UTF{-}8"}\hlstd{\ ?}\hlopt{\symbol{62}}\\
\hllin{2\ }\hlstd{}\hlopt{\symbol{60}!}\hlstd{DOCTYPE\ configuration\\
\hllin{3\ }}\hlstd{\ \ }\hlstd{PUBLIC\ }\hlstr{"{-}//mybatis.org//DTD\ Config\ 3.0//EN"}\hlstd{\\
\hllin{4\ }}\hlstd{\ \ }\hlstd{}\hlstr{"http://mybatis.org/dtd/mybatis{-}3{-}config.dtd"}\hlstd{}\hlopt{\symbol{62}}\\
\hllin{5\ }\hlstd{}\hlopt{\symbol{60}}\hlstd{configuration}\hlopt{\symbol{62}}\\
\hllin{6\ }\hlstd{}\hlstd{\ \ }\hlstd{}\hlopt{\symbol{60}}\hlstd{environments\ }\hlkwa{default}\hlstd{}\hlopt{=}\hlstd{}\hlstr{"development"}\hlstd{}\hlopt{\symbol{62}}\\
\hllin{7\ }\hlstd{}\hlstd{\ \ \ \ }\hlstd{}\hlopt{\symbol{60}}\hlstd{environment\ id}\hlopt{=}\hlstd{}\hlstr{"development"}\hlstd{}\hlopt{\symbol{62}}\\
\hllin{8\ }\hlstd{}\hlstd{\ \ \ \ \ \ }\hlstd{}\hlopt{\symbol{60}}\hlstd{transactionManager\ type}\hlopt{=}\hlstd{}\hlstr{"JDBC"}\hlstd{}\hlopt{/\symbol{62}}\\
\hllin{9\ }\hlstd{}\hlstd{\ \ \ \ \ \ }\hlstd{}\hlopt{\symbol{60}}\hlstd{dataSource\ type}\hlopt{=}\hlstd{}\hlstr{"POOLED"}\hlstd{}\hlopt{\symbol{62}}\\
\hllin{10\ }\hlstd{}\hlstd{\ \ \ \ \ \ \ \ }\hlstd{}\hlopt{\symbol{60}}\hlstd{}\hlkwa{property\ }\hlstd{name}\hlopt{=}\hlstd{}\hlstr{"driver"}\hlstd{\ value}\hlopt{=}\hlstd{}\hlstr{"\$\symbol{123}driver\symbol{125}"}\hlstd{}\hlopt{/\symbol{62}}\\
\hllin{11\ }\hlstd{}\hlstd{\ \ \ \ \ \ \ \ }\hlstd{}\hlopt{\symbol{60}}\hlstd{}\hlkwa{property\ }\hlstd{name}\hlopt{=}\hlstd{}\hlstr{"url"}\hlstd{\ value}\hlopt{=}\hlstd{}\hlstr{"\$\symbol{123}url\symbol{125}"}\hlstd{}\hlopt{/\symbol{62}}\\
\hllin{12\ }\hlstd{}\hlstd{\ \ \ \ \ \ \ \ }\hlstd{}\hlopt{\symbol{60}}\hlstd{}\hlkwa{property\ }\hlstd{name}\hlopt{=}\hlstd{}\hlstr{"username"}\hlstd{\ value}\hlopt{=}\hlstd{}\hlstr{"\$\symbol{123}username\symbol{125}"}\hlstd{}\hlopt{/\symbol{62}}\\
\hllin{13\ }\hlstd{}\hlstd{\ \ \ \ \ \ \ \ }\hlstd{}\hlopt{\symbol{60}}\hlstd{}\hlkwa{property\ }\hlstd{name}\hlopt{=}\hlstd{}\hlstr{"password"}\hlstd{\ value}\hlopt{=}\hlstd{}\hlstr{"\$\symbol{123}password\symbol{125}"}\hlstd{}\hlopt{/\symbol{62}}\\
\hllin{14\ }\hlstd{}\hlstd{\ \ \ \ \ \ }\hlstd{}\hlopt{\symbol{60}/}\hlstd{dataSource}\hlopt{\symbol{62}}\\
\hllin{15\ }\hlstd{}\hlstd{\ \ \ \ }\hlstd{}\hlopt{\symbol{60}/}\hlstd{environment}\hlopt{\symbol{62}}\\
\hllin{16\ }\hlstd{}\hlstd{\ \ }\hlstd{}\hlopt{\symbol{60}/}\hlstd{environments}\hlopt{\symbol{62}}\\
\hllin{17\ }\hlstd{}\hlstd{\ \ }\hlstd{}\hlopt{\symbol{60}}\hlstd{mappers}\hlopt{\symbol{62}}\\
\hllin{18\ }\hlstd{}\hlstd{\ \ \ \ }\hlstd{}\hlopt{\symbol{60}}\hlstd{mapper\ resource}\hlopt{=}\hlstd{}\hlstr{"org/mybatis/example/BlogMapper.xml"}\hlstd{}\hlopt{/\symbol{62}}\\
\hllin{19\ }\hlstd{}\hlstd{\ \ }\hlstd{}\hlopt{\symbol{60}/}\hlstd{mappers}\hlopt{\symbol{62}}\\
\hllin{20\ }\hlstd{}\hlopt{\symbol{60}/}\hlstd{configuration}\hlopt{\symbol{62}}\hlstd{} 
\mbox{}
\normalfont
\normalsize


当然,XML 配置文件中还有很多可以配置的,在 上面的示例指出的则是最关键的部分。要注意 XML 文档头部需要验证 XML 文档正确性。Environment元素包含这对事务管理和连接池的环境配置。Mappers元素包含着mapper列表——mapper是包含SQL代码及映射定义的XML文件。
\subsection{不使用XML来构建SqlSessionFactory}
如果你更愿意直接从JAVA程序中而不是XML构建配置实例,或者创建你自己的配置生成器,Mybatis提供了一个完整的配置类(Configuration),它提供所有与XML 文件中一样的配置选项。

\noindent
\ttfamily
\hlstd{\hllin{1\ }DataSource\ dataSource\ }\hlopt{=\ }\hlstd{BlogDataSourceFactory}\hlopt{.}\Righttorque\\
\hllin{2\ }\hlstd{}\hlkwd{getBlogDataSource}\hlstd{}\hlopt{();}\\
\hllin{3\ }\hlstd{TransactionFactory\ transactionFactory\ }\hlopt{=\ }\hlstd{}\hlkwa{new\ }\Righttorque\\
\hllin{4\ }\hlstd{}\hlkwd{JdbcTransactionFactory}\hlstd{}\hlopt{();}\\
\hllin{5\ }\hlstd{Environment\ environment\ }\hlopt{=\ }\hlstd{}\hlkwa{new\ }\hlstd{}\hlkwd{Environment}\hlstd{}\hlopt{(}\hlstd{}\hlstr{"development"}\hlstd{}\hlopt{,\ }\Righttorque\\
\hllin{6\ }\hlstd{transactionFactory}\hlopt{,\ }\hlstd{dataSource}\hlopt{);}\\
\hllin{7\ }\hlstd{Configuration\ configuration\ }\hlopt{=\ }\hlstd{}\hlkwa{new\ }\hlstd{}\hlkwd{Configuration}\hlstd{}\hlopt{(}\hlstd{environment}\hlopt{);}\\
\hllin{8\ }\hlstd{configuration}\hlopt{.}\hlstd{}\hlkwd{addMapper}\hlstd{}\hlopt{(}\hlstd{BlogMapper}\hlopt{.}\hlstd{}\hlkwa{class}\hlstd{}\hlopt{);}\\
\hllin{9\ }\hlstd{SqlSessionFactory\ sqlSessionFactory\ }\hlopt{=\ }\hlstd{}\hlkwa{new\ }\Righttorque\\
\hllin{10\ }\hlstd{}\hlkwd{SqlSessionFactoryBuilder}\hlstd{}\hlopt{().}\hlstd{}\hlkwd{build}\hlstd{}\hlopt{(}\hlstd{configuration}\hlopt{);}\\
\hllin{11\ }\hlstd{} 
\mbox{}
\normalfont
\normalsize


注意这种情况下配置是添加映射类。映射类是 Java 类,这些类包含 SQL 映射语句的注 解从而避免了 XML 文件的依赖。但是,由于JAVA注释的一些限制以及一些MyBatis映射的复杂性,XML映射仍然在大多数高级映射(比如:嵌套 Join 映射)时需要。由于以上原因,如果存在一个映射类对等的XML文件,Mybatis会自动的查找和加载它(在本例中,基于类路径和BlogMapper.class的类名,BlogMapper.xml将会被加载)。后面我们会了解更多。

\subsection{从SqlSessionFactory获得SqlSession}

现在你已经有了一个SqlSessionFactory,就如它名字所提示的,你可以从那获取一个SqlSession实例。SqlSession完全包含了各种在数据库中执行SQL命令的方法。你可以通过SqlSession实例直接执行已映射的SQL语句。例如:

\noindent
\ttfamily
\hlstd{\hllin{1\ }SqlSession\ session\ }\hlopt{=\ }\hlstd{sqlSessionFactory}\hlopt{.}\hlstd{}\hlkwd{openSession}\hlstd{}\hlopt{();}\\
\hllin{2\ }\hlstd{}\hlkwa{try\ }\hlstd{}\hlopt{\symbol{123}}\\
\hllin{3\ }\hlstd{}\hlstd{\ \ \ \ }\hlstd{Blog\ blog\ }\hlopt{=\ (}\hlstd{Blog}\hlopt{)\ }\hlstd{session}\hlopt{.}\hlstd{}\hlkwd{selectOne}\hlstd{}\hlopt{(}\hlstd{}\hlstr{"org.mybatis.}\Righttorque\\
\hllin{4\ }\hlstr{}\hlstd{\ \ \ \ }\hlstr{example.BlogMapper.selectBlog"}\hlstd{}\hlopt{,\ }\hlstd{}\hlnum{101}\hlstd{}\hlopt{);}\\
\hllin{5\ }\hlstd{}\hlopt{\symbol{125}}\\
\hllin{6\ }\hlstd{}\hlkwa{finally\ }\hlstd{}\hlopt{\symbol{123}}\\
\hllin{7\ }\hlstd{}\hlstd{\ \ \ \ }\hlstd{session}\hlopt{.}\hlstd{}\hlkwd{close}\hlstd{}\hlopt{();}\\
\hllin{8\ }\hlstd{}\hlopt{\symbol{125}}\\
\hllin{9\ }\hlstd{} 
\mbox{}
\normalfont
\normalsize


虽然这种方法可行,而且这种方法被Mybatis以前版本的用户所熟知,但现在有一种更为简洁的方法。使用合理描述参数和 SQL 语句返回值的接口(比如 BlogMapper.class) ,这样你现在就 可以执行更简洁,类型更安全的代码,没有容易发生的字符串文字和转换的错误。例如:

\noindent
\ttfamily
\hlstd{\hllin{1\ }SqlSession\ session\ }\hlopt{=\ }\hlstd{sqlSessionFactory}\hlopt{.}\hlstd{}\hlkwd{openSession}\hlstd{}\hlopt{();}\\
\hllin{2\ }\hlstd{}\hlkwa{try\ }\hlstd{}\hlopt{\symbol{123}}\\
\hllin{3\ }\hlstd{}\hlstd{\ \ \ \ }\hlstd{BlogMapper\ mapper\ }\hlopt{=\ }\hlstd{session}\hlopt{.}\hlstd{}\hlkwd{getMapper}\hlstd{}\hlopt{(}\hlstd{BlogMapper}\hlopt{.}\hlstd{}\hlkwa{class}\hlstd{}\hlopt{);}\\
\hllin{4\ }\hlstd{}\hlstd{\ \ \ \ }\hlstd{Blog\ blog\ }\hlopt{=\ }\hlstd{mapper}\hlopt{.}\hlstd{}\hlkwd{selectBlog}\hlstd{}\hlopt{(}\hlstd{}\hlnum{101}\hlstd{}\hlopt{);}\\
\hllin{5\ }\hlstd{}\hlopt{\symbol{125}}\\
\hllin{6\ }\hlstd{}\hlkwa{finally\ }\hlstd{}\hlopt{\symbol{123}}\\
\hllin{7\ }\hlstd{}\hlstd{\ \ \ \ }\hlstd{session}\hlopt{.}\hlstd{}\hlkwd{close}\hlstd{}\hlopt{();}\\
\hllin{8\ }\hlstd{}\hlopt{\symbol{125}}\\
\hllin{9\ }\hlstd{} 
\mbox{}
\normalfont
\normalsize


现在我们来探究一下这里到底执行了什么。
\subsection{探究已映射的SQL语句}
这里你也许想知道通过 SqlSession 和 Mapper 对象到底执行了什么操作。已映射的 SQL 语句是一个很大的主题, 而且这个主题会贯穿本文档的大部分内容。 为了给一个宏观的概 念,这里有一些示例。

上面提到的任何一个示例,语句是通过 XML 或注解定义的。我们先来看看 XML。使用基于 XML 的映射语言可以实现MyBatis所提供的完整特性。在过去的几年中,基于 XML 的映射语言使得MyBatis 非常流行。如果你之前使用过MyBatis,这个概念你应该很熟悉,但是 XML 映射文件也有 很多的改进,后面我们会详细来说。这里给出一个基于 XML 映射语句的示例,这些语句可以满足上述示例中 SqlSession 对象的调用。


\noindent
\ttfamily
\hlstd{}\hllin{1\ }\hlopt{\symbol{60}}\hlstd{?xml\ version}\hlopt{=}\hlstd{}\hlstr{"1.0"}\hlstd{\ encoding}\hlopt{=}\hlstd{}\hlstr{"UTF{-}8"}\hlstd{\ ?}\hlopt{\symbol{62}}\\
\hllin{2\ }\hlstd{}\hlopt{\symbol{60}!}\hlstd{DOCTYPE\ mapper\\
\hllin{3\ }}\hlstd{\ \ }\hlstd{PUBLIC\ }\hlstr{"{-}//mybatis.org//DTD\ Mapper\ 3.0//EN"}\hlstd{\\
\hllin{4\ }}\hlstd{\ \ }\hlstd{}\hlstr{"http://mybatis.org/dtd/mybatis{-}3{-}mapper.dtd"}\hlstd{}\hlopt{\symbol{62}}\\
\hllin{5\ }\hlstd{}\hlopt{\symbol{60}}\hlstd{mapper\ namespace}\hlopt{=}\hlstd{}\hlstr{"org.mybatis.example.BlogMapper"}\hlstd{}\hlopt{\symbol{62}}\\
\hllin{6\ }\hlstd{}\hlstd{\ \ }\hlstd{}\hlopt{\symbol{60}}\hlstd{select\ id}\hlopt{=}\hlstd{}\hlstr{"selectBlog"}\hlstd{\ parameterType}\hlopt{=}\hlstd{}\hlstr{"int"}\hlstd{\ \Righttorque\\
\hllin{7\ }}\hlstd{\ \ }\hlstd{resultType}\hlopt{=}\hlstd{}\hlstr{"Blog"}\hlstd{}\hlopt{\symbol{62}}\\
\hllin{8\ }\hlstd{}\hlstd{\ \ \ \ }\hlstd{select\ }\hlopt{{*}\ }\hlstd{from\ Blog\ where\ id\ }\hlopt{=\ }\hlstd{\#}\hlopt{\symbol{123}}\hlstd{id}\hlopt{\symbol{125}}\\
\hllin{9\ }\hlstd{}\hlstd{\ \ }\hlstd{}\hlopt{\symbol{60}/}\hlstd{select}\hlopt{\symbol{62}}\\
\hllin{10\ }\hlstd{}\hlopt{\symbol{60}/}\hlstd{mapper}\hlopt{\symbol{62}}\hlstd{} 
\mbox{}
\normalfont
\normalsize


这个简单的例子中看起来有很多额外的东西, 但是也相当简洁了。你可以在一个单独的XML映射文件中定义很多的映射语句,除XML头部和文档类型声明之外,你可以得到很多方便之处。在文件的剩余部分是很好的自我解释。在命名空间 “com.mybatis.example.BlogMapper”中,它定义了一个名为“selectBlog”的映射语句,这 样它允许你使用完全限定名 “org.mybatis.example.BlogMapper.selectBlog” 来调用映射语句, 我们下面示例中所有的写法也是这样的。

\noindent
\ttfamily
\hlstd{\hllin{1\ }Blog\ blog\ }\hlopt{=\ (}\hlstd{Blog}\hlopt{)\ }\hlstd{session}\hlopt{.}\hlstd{}\hlkwd{selectOne}\hlstd{}\hlopt{(}\hlstd{}\hlstr{"org.mybatis.example.}\Righttorque\\
\hllin{2\ }\hlstr{BlogMapper.selectBlog"}\hlstd{}\hlopt{,\ }\hlstd{}\hlnum{101}\hlstd{}\hlopt{);}\hlstd{} 
\mbox{}
\normalfont
\normalsize


要注意这个使用完全限定名调用Java对象的方法是相似的,这样做是有原因的。这个命名可以直接映射到类名与命名空间一致的映射类,映射类包含一个与已映射查询语句的名称、参数、返回值都一致的方法即可。这就允许你非常容易地调用映射器接口中的方法, 这和你前面看到 的是一样的,下面这个示例中它又出现了。

\noindent
\ttfamily
\hlstd{\hllin{1\ }BlogMapper\ mapper\ }\hlopt{=\ }\hlstd{session}\hlopt{.}\hlstd{}\hlkwd{getMapper}\hlstd{}\hlopt{(}\hlstd{BlogMapper}\hlopt{.}\hlstd{}\hlkwa{class}\hlstd{}\hlopt{);}\\
\hllin{2\ }\hlstd{Blog\ blog\ }\hlopt{=\ }\hlstd{mapper}\hlopt{.}\hlstd{}\hlkwd{selectBlog}\hlstd{}\hlopt{(}\hlstd{}\hlnum{101}\hlstd{}\hlopt{);}\\
\hllin{3\ }\hlstd{} 
\mbox{}
\normalfont
\normalsize


第二种方式有很多有点,首先它不是基于文字的,所以更安全。第二,如果你的 IDE 有代码补全功能,那么你可以利用它来操纵已映射的SQL语句。

命名空间的一些注释:

命名空间在之前版本的MyBatis中是可选项,非常混乱也没有帮助。现在,命名空间是必须的,而且有一个目的,不仅仅是简单地使用更长的完全限定名来隔离语句。

就像你看到的那样,命名空间使得接口绑定成为可能。而且就算你认为你现在不会使用他们,你应该按照下面给出示例的来练习,以免改变自己的想法。使用命名空间,并将它放在合适的Java包空间之下,将会使你的代码变得简洁,在很长的时间内提高MyBatis 的可用性。

命名解析: 为了减少输入量,MyBatis对所有的命名配置元素使用如下的命名解析规则, 配置元素包括语句,结果映射,缓存等。

•   直接查找完全限定名(比如“com.mypackage.MyMapper.selectAllThings” ),如果发现就使用。

•   短名称(比如“selectAllThings” )可以用来引用任意明确的条目。但是如果有两个 或两个以上的条目(比如“com.foo.selectAllThings ”和“com.bar.selectAllThings” ), 那么就会得到错误报告,说短名称是含糊的,因此就必须使用完全限定名。

像BlogMapper这样的映射器类来说,还有一个窍门。它们中间映射的语句可以不需要在XML中来写,而可以使用Java注解来替换。比如,上面的XML示例可以替换为:

\noindent
\ttfamily
\hlstd{}\hllin{1\ }\hlkwa{package\ }\hlstd{org}\hlopt{.}\hlstd{mybatis}\hlopt{.}\hlstd{example}\hlopt{;}\\
\hllin{2\ }\hlstd{}\hlkwa{public\ interface\ }\hlstd{BlogMapper\ }\hlopt{\symbol{123}}\\
\hllin{3\ }\hlstd{}\hlstd{\ \ \ \ }\hlstd{}\hlkwc{@Select}\hlstd{}\hlopt{(}\hlstd{}\hlstr{"SELECT\ {*}\ FROM\ blog\ WHERE\ id\ =\ \#\symbol{123}id\symbol{125}"}\hlstd{}\hlopt{)}\\
\hllin{4\ }\hlstd{}\hlstd{\ \ \ \ }\hlstd{Blog\ }\hlkwd{selectBlog}\hlstd{}\hlopt{(}\hlstd{}\hlkwb{int\ }\hlstd{id}\hlopt{);}\\
\hllin{5\ }\hlstd{}\hlopt{\symbol{125}}\\
\hllin{6\ }\hlstd{} 
\mbox{}
\normalfont
\normalsize


对于简单语句来说,使用注解会更加清晰。然而Java注解对于复杂语句来说有局限性,且更加混乱。因此,如果你不得不做复杂的事情, 那么最好使用XML来映射语句。

当然这也取决于你和你的项目团队的决定, 看哪种更适合你来使用, 用一致的方式去定义你的映射语句很重要。这表示,不要将自己局限在一种方式中。你可以轻松地将注解型映射语句换成XML映射语句,反之亦然。

\subsection{范围和生命周期}
理解我们目前已经讨论过的不同范围和生命周期类是很重要的。 不正确的使用它们会导 致严重的并发问题。
\subsubsection{SqlSessionFactoryBuilder}
这个类可以被实例化,使用和丢弃。一旦你创建了 SqlSessionFactory 后,这个类就不需 要存在了。因此 SqlSessionFactoryBuilder 实例的最佳范围是方法范围 (也就是本地方法变量)。你可以重用 SqlSessionFactoryBuilder 来创建多个 SqlSessionFactory 实例,但是最好不要保持它,以确保所有的XML解析资源能够释放去做更重要的事。
\subsubsection{SqlSessionFactory}
SqlSessionFactory一旦被创建,它应该持续在你的应用执行期间都存在。没有理由来处理或重新创建它。使用 SqlSessionFactory 的最佳实践是在应用运行期间不要重复创建多次。这样的 操作将被视为是很糟糕的。因此SqlSessionFactory的最佳范围是应用范围。有很多方法可 以做到,最简单的就是使用单例模式或者静态单例模式。
\subsubsection{SqlSession}
每个线程都需要有它自己的SqlSession实例。SqlSession实例不能被共享且不是线程安全的。因此SqlSession的最佳范围是请求范围或方法范围。绝对不能将 SqlSession 实例的引用放在一个 类的静态字段甚至是实例字段中。比如 Serlvet 架构中的 HttpSession。 如果你现在正用任意的 Web 框架, 要考虑 SqlSession 放在一个和 HTTP 请求对象相似的范围内。换句话说,基于收到的 HTTP 请求,你可以打开 了一个 SqlSession,然后返回响应,就可以关闭它了。关闭 Session 很重要,你应该确保使 用 finally 块来关闭它。下面的示例就是一个确保 SqlSession 关闭的基本模式:

\noindent
\ttfamily
\hlstd{\hllin{1\ }SqlSession\ session\ }\hlopt{=\ }\hlstd{sqlSessionFactory}\hlopt{.}\hlstd{}\hlkwd{openSession}\hlstd{}\hlopt{();}\\
\hllin{2\ }\hlstd{}\hlkwa{try\ }\hlstd{}\hlopt{\symbol{123}}\\
\hllin{3\ }\hlstd{}\hlstd{\ \ \ \ }\hlstd{}\hlslc{//\ do\ work}\\
\hllin{4\ }\hlstd{}\hlopt{\symbol{125}\ }\hlstd{}\hlkwa{finally\ }\hlstd{}\hlopt{\symbol{123}}\\
\hllin{5\ }\hlstd{}\hlstd{\ \ \ \ }\hlstd{session}\hlopt{.}\hlstd{}\hlkwd{close}\hlstd{}\hlopt{();}\\
\hllin{6\ }\hlstd{}\hlopt{\symbol{125}}\\
\hllin{7\ }\hlstd{} 
\mbox{}
\normalfont
\normalsize


在你的代码中一贯地使用这种模式,将会保证所有数据库资源都正确地关闭。
\subsubsection{Mapper实例}
映射器是你创建来绑定映射语句的接口。映射器接口的实例可以从 SqlSession 中获得。那 么从技术上来说,当被请求时,任意映射器实例的最宽范围和 SqlSession 是相同的。然而, 映射器实例的最佳范围是方法范围。也就是说,它们应该在使用它们的方法中被请求,然后 就抛弃掉。它们不需要明确地关闭,那么在请求对象中保留它们也就不是什么问题了,这和 SqlSession 相似。你也许会发现,在这个水平上管理太多的资源的话会让你忙不过来。保持简单,将 映射器放在方法范围内。下面的示例就展示了这个实践:

\noindent
\ttfamily
\hlstd{\hllin{1\ }SqlSession\ session\ }\hlopt{=\ }\hlstd{sqlSessionFactory}\hlopt{.}\hlstd{}\hlkwd{openSession}\hlstd{}\hlopt{();}\\
\hllin{2\ }\hlstd{}\hlkwa{try\ }\hlstd{}\hlopt{\symbol{123}}\\
\hllin{3\ }\hlstd{}\hlstd{\ \ \ \ }\hlstd{}\hlslc{//\ do\ work}\\
\hllin{4\ }\hlstd{}\hlopt{\symbol{125}\ }\hlstd{}\hlkwa{finally\ }\hlstd{}\hlopt{\symbol{123}}\\
\hllin{5\ }\hlstd{}\hlstd{\ \ \ \ }\hlstd{session}\hlopt{.}\hlstd{}\hlkwd{close}\hlstd{}\hlopt{();}\\
\hllin{6\ }\hlstd{}\hlopt{\symbol{125}}\\
\hllin{7\ }\hlstd{} 
\mbox{}
\normalfont
\normalsize


对象生命周期与依赖注入框架的注释:

依赖注入框架可以创建线程安全的事务型的SqlSessions实例和映射器(Mapper)实例,并将它们直接注入到你的java bean中,所以你可以不考虑它们的生命周期了。你可以看一下Mybatis-Spring或者Mybatis-Guice这两个子项目去了解更多关于在依赖注入框架下使用Mybatis的内容。

\section{Mapper XML文件}
MyBatis真正的强大之处是在映射语句中。这是奇迹发生的地方。相对于MyBatis的强大,SQL映射的XML文件是简单的。当然如果你将它们和对等功能的JDBC代码来比较,你会马上发现映射文件节省了大约 95\%的代码量。MyBatis就是聚焦于 SQL而构建的,并且竭尽全力帮你脱离JDBC编程。

SQL 映射文件只有很少的几个顶级元素(按照它们应该被定义的顺序):

•	cache – 指定命名空间的缓存配置。

•	cache-ref – 从其他命名空间引用的缓存配置。

•	resultMap – 最复杂,也是最强大的元素,用来描述如何从数据库结果集中来加 载你的对象。

•	parameterMap – 不赞成使用!这是久风格的参数映射。内联参数是首选,这个元 素在将来可能被移除。这里就不阐述了。

•	sql – 可以被其他语句引用的可重用的 SQL 块。

•	insert – 映射插入语句

•	update – 映射更新语句

•	delete – 映射删除语句

•	select – 映射查询语句

下一部分将从语句本身开始来描述每个元素的细节。
\subsection{Select}
查询语句是使用MyBatis时最常用的元素之一。存入数据库的数据只有在你从数据库取出来时才有价值,所以大多数应用程序的查询要比更改数据多的多。 对于每次插入、更新或删除,其中也可能会伴随很多的查询操作。这是MyBatis的一个基本原则,也是将重心和努力放到查询和结果映射的原因。对简单的查询情况,查询元素也是较为简单的。比如:

\noindent
\ttfamily
\hlstd{}\hllin{1\ }\hlopt{\symbol{60}}\hlstd{select\ id}\hlopt{=}\hlstd{}\hlstr{"selectPerson"}\hlstd{\ parameterType}\hlopt{=}\hlstd{}\hlstr{"int"}\hlstd{\ \Righttorque\\
\hllin{2\ }resultType}\hlopt{=}\hlstd{}\hlstr{"hashmap"}\hlstd{}\hlopt{\symbol{62}}\\
\hllin{3\ }\hlstd{}\hlstd{\ \ }\hlstd{SELECT\ }\hlopt{{*}\ }\hlstd{FROM\ PERSON\ WHERE\ ID\ }\hlopt{=\ }\hlstd{\#}\hlopt{\symbol{123}}\hlstd{id}\hlopt{\symbol{125}}\\
\hllin{4\ }\hlstd{}\hlopt{\symbol{60}/}\hlstd{select}\hlopt{\symbol{62}}\hlstd{} 
\mbox{}
\normalfont
\normalsize


这个查询语句称为selectPerson,使用一个Int(或Integer)类型的参数,并返回一个HashMap类型的对象,其中的键是列名,值是列对应的值。
注意参数注释:\#\{id\} 
它告诉MyBatis创建一个预处理语句参数。在JDBC编程中,这样的一个参数在SQL中会 由一个“?”来标识,并被传递到一个新的预处理语句中,就像这样:

\noindent
\ttfamily
\hlstd{}\hllin{1\ }\hlslc{//\ Similar\ JDBC\ code,\ NOT\ MyBatis…}\\
\hllin{2\ }\hlstd{String\ selectPerson\ }\hlopt{=\ }\hlstd{}\hlstr{"SELECT\ {*}\ FROM\ PERSON\ WHERE\ ID=?"}\hlstd{}\hlopt{;}\\
\hllin{3\ }\hlstd{PreparedStatement\ ps\ }\hlopt{=\ }\hlstd{conn}\hlopt{.}\hlstd{}\hlkwd{prepareStatement}\hlstd{}\hlopt{(}\hlstd{selectPerson}\hlopt{);}\\
\hllin{4\ }\hlstd{ps}\hlopt{.}\hlstd{}\hlkwd{setInt}\hlstd{}\hlopt{(}\hlstd{}\hlnum{1}\hlstd{}\hlopt{,}\hlstd{id}\hlopt{);}\\
\hllin{5\ }\hlstd{} 
\mbox{}
\normalfont
\normalsize


当然, 这需要很多单独的 JDBC 的代码来提取结果并将它们映射到对象实例中, 这就是 MyBatis为你减少工作量之处。我们需要深入了解参数和结果映射。那些细节部分我们下面来了解。

select 元素有很多属性,这些属性使你可以配置每条语句运作的细节。
\noindent
\ttfamily
\hlstd{}\hllin{1\ }\hlopt{\symbol{60}}\hlstd{select\\
\hllin{2\ }}\hlstd{\ \ }\hlstd{id}\hlopt{=}\hlstd{}\hlstr{"selectPerson"}\hlstd{\\
\hllin{3\ }}\hlstd{\ \ }\hlstd{parameterType}\hlopt{=}\hlstd{}\hlstr{"int"}\hlstd{\\
\hllin{4\ }}\hlstd{\ \ }\hlstd{parameterMap}\hlopt{=}\hlstd{}\hlstr{"deprecated"}\hlstd{\\
\hllin{5\ }}\hlstd{\ \ }\hlstd{resultType}\hlopt{=}\hlstd{}\hlstr{"hashmap"}\hlstd{\\
\hllin{6\ }}\hlstd{\ \ }\hlstd{resultMap}\hlopt{=}\hlstd{}\hlstr{"personResultMap"}\hlstd{\\
\hllin{7\ }}\hlstd{\ \ }\hlstd{flushCache}\hlopt{=}\hlstd{}\hlstr{"false"}\hlstd{\\
\hllin{8\ }}\hlstd{\ \ }\hlstd{useCache}\hlopt{=}\hlstd{}\hlstr{"true"}\hlstd{\\
\hllin{9\ }}\hlstd{\ \ }\hlstd{timeout}\hlopt{=}\hlstd{}\hlstr{"10000"}\hlstd{\\
\hllin{10\ }}\hlstd{\ \ }\hlstd{fetchSize}\hlopt{=}\hlstd{}\hlstr{"256"}\hlstd{\\
\hllin{11\ }}\hlstd{\ \ }\hlstd{statementType}\hlopt{=}\hlstd{}\hlstr{"PREPARED"}\hlstd{\\
\hllin{12\ }}\hlstd{\ \ }\hlstd{resultSetType}\hlopt{=}\hlstd{}\hlstr{"FORWARD\symbol{95}ONLY"}\hlstd{}\hlopt{\symbol{62}}\hlstd{} 
\mbox{}
\normalfont
\normalsize


\subsection{Insert, Update and Delete}
数据变更语句insert、update和delete在它们的实现中非常相似:

\noindent
\ttfamily
\hlstd{}\hllin{1\ }\hlopt{\symbol{60}}\hlstd{insert\\
\hllin{2\ }}\hlstd{\ \ }\hlstd{id}\hlopt{=}\hlstd{}\hlstr{"insertAuthor"}\hlstd{\\
\hllin{3\ }}\hlstd{\ \ }\hlstd{parameterType}\hlopt{=}\hlstd{}\hlstr{"domain.blog.Author"}\hlstd{\\
\hllin{4\ }}\hlstd{\ \ }\hlstd{flushCache}\hlopt{=}\hlstd{}\hlstr{"true"}\hlstd{\\
\hllin{5\ }}\hlstd{\ \ }\hlstd{statementType}\hlopt{=}\hlstd{}\hlstr{"PREPARED"}\hlstd{\\
\hllin{6\ }}\hlstd{\ \ }\hlstd{keyProperty}\hlopt{=}\hlstd{}\hlstr{""}\hlstd{\\
\hllin{7\ }}\hlstd{\ \ }\hlstd{keyColumn}\hlopt{=}\hlstd{}\hlstr{""}\hlstd{\\
\hllin{8\ }}\hlstd{\ \ }\hlstd{useGeneratedKeys}\hlopt{=}\hlstd{}\hlstr{""}\hlstd{\\
\hllin{9\ }}\hlstd{\ \ }\hlstd{timeout}\hlopt{=}\hlstd{}\hlstr{"20000"}\hlstd{}\hlopt{\symbol{62}}\\
\hllin{10\ }\hlstd{}\\
\hllin{11\ }\hlopt{\symbol{60}}\hlstd{update\\
\hllin{12\ }}\hlstd{\ \ }\hlstd{id}\hlopt{=}\hlstd{}\hlstr{"insertAuthor"}\hlstd{\\
\hllin{13\ }}\hlstd{\ \ }\hlstd{parameterType}\hlopt{=}\hlstd{}\hlstr{"domain.blog.Author"}\hlstd{\\
\hllin{14\ }}\hlstd{\ \ }\hlstd{flushCache}\hlopt{=}\hlstd{}\hlstr{"true"}\hlstd{\\
\hllin{15\ }}\hlstd{\ \ }\hlstd{statementType}\hlopt{=}\hlstd{}\hlstr{"PREPARED"}\hlstd{\\
\hllin{16\ }}\hlstd{\ \ }\hlstd{timeout}\hlopt{=}\hlstd{}\hlstr{"20000"}\hlstd{}\hlopt{\symbol{62}}\\
\hllin{17\ }\hlstd{}\\
\hllin{18\ }\hlopt{\symbol{60}}\hlstd{delete\\
\hllin{19\ }}\hlstd{\ \ }\hlstd{id}\hlopt{=}\hlstd{}\hlstr{"insertAuthor"}\hlstd{\\
\hllin{20\ }}\hlstd{\ \ }\hlstd{parameterType}\hlopt{=}\hlstd{}\hlstr{"domain.blog.Author"}\hlstd{\\
\hllin{21\ }}\hlstd{\ \ }\hlstd{flushCache}\hlopt{=}\hlstd{}\hlstr{"true"}\hlstd{\\
\hllin{22\ }}\hlstd{\ \ }\hlstd{statementType}\hlopt{=}\hlstd{}\hlstr{"PREPARED"}\hlstd{\\
\hllin{23\ }}\hlstd{\ \ }\hlstd{timeout}\hlopt{=}\hlstd{}\hlstr{"20000"}\hlstd{}\hlopt{\symbol{62}}\hlstd{} 
\mbox{}
\normalfont
\normalsize


下面是insert、update和delete语句的示例:

\noindent
\ttfamily
\hlstd{}\hllin{1\ }\hlopt{\symbol{60}}\hlstd{insert\ id}\hlopt{=}\hlstd{}\hlstr{"insertAuthor"}\hlstd{\ parameterType}\hlopt{=}\hlstd{}\hlstr{"domain.blog.Author"}\hlstd{}\hlopt{\symbol{62}}\\
\hllin{2\ }\hlstd{}\hlstd{\ \ }\hlstd{insert\ into\ }\hlkwd{Author\ }\hlstd{}\hlopt{(}\hlstd{id}\hlopt{,}\hlstd{username}\hlopt{,}\hlstd{password}\hlopt{,}\hlstd{email}\hlopt{,}\hlstd{bio}\hlopt{)}\\
\hllin{3\ }\hlstd{}\hlstd{\ \ }\hlstd{}\hlkwd{values\ }\hlstd{}\hlopt{(}\hlstd{\#}\hlopt{\symbol{123}}\hlstd{id}\hlopt{\symbol{125},}\hlstd{\#}\hlopt{\symbol{123}}\hlstd{username}\hlopt{\symbol{125},}\hlstd{\#}\hlopt{\symbol{123}}\hlstd{password}\hlopt{\symbol{125},}\hlstd{\#}\hlopt{\symbol{123}}\hlstd{email}\hlopt{\symbol{125},}\hlstd{\#}\hlopt{\symbol{123}}\hlstd{bio}\hlopt{\symbol{125})}\\
\hllin{4\ }\hlstd{}\hlopt{\symbol{60}/}\hlstd{insert}\hlopt{\symbol{62}}\\
\hllin{5\ }\hlstd{}\hlopt{\symbol{60}}\hlstd{update\ id}\hlopt{=}\hlstd{}\hlstr{"updateAuthor"}\hlstd{\ parameterType}\hlopt{=}\hlstd{}\hlstr{"domain.blog.Author"}\hlstd{}\hlopt{\symbol{62}}\\
\hllin{6\ }\hlstd{}\hlstd{\ \ }\hlstd{update\ Author\ set\\
\hllin{7\ }}\hlstd{\ \ \ \ }\hlstd{username\ }\hlopt{=\ }\hlstd{\#}\hlopt{\symbol{123}}\hlstd{username}\hlopt{\symbol{125},}\\
\hllin{8\ }\hlstd{}\hlstd{\ \ \ \ }\hlstd{password\ }\hlopt{=\ }\hlstd{\#}\hlopt{\symbol{123}}\hlstd{password}\hlopt{\symbol{125},}\\
\hllin{9\ }\hlstd{}\hlstd{\ \ \ \ }\hlstd{email\ }\hlopt{=\ }\hlstd{\#}\hlopt{\symbol{123}}\hlstd{email}\hlopt{\symbol{125},}\\
\hllin{10\ }\hlstd{}\hlstd{\ \ \ \ }\hlstd{bio\ }\hlopt{=\ }\hlstd{\#}\hlopt{\symbol{123}}\hlstd{bio}\hlopt{\symbol{125}}\\
\hllin{11\ }\hlstd{}\hlstd{\ \ }\hlstd{where\ id\ }\hlopt{=\ }\hlstd{\#}\hlopt{\symbol{123}}\hlstd{id}\hlopt{\symbol{125}}\\
\hllin{12\ }\hlstd{}\hlopt{\symbol{60}/}\hlstd{update}\hlopt{\symbol{62}}\\
\hllin{13\ }\hlstd{}\hlopt{\symbol{60}}\hlstd{delete\ id}\hlopt{=}\hlstd{}\hlstr{"deleteAuthor"}\hlstd{\ parameterType}\hlopt{=}\hlstd{}\hlstr{"int"}\hlstd{}\hlopt{\symbol{62}}\\
\hllin{14\ }\hlstd{}\hlstd{\ \ }\hlstd{delete\ from\ Author\ where\ id\ }\hlopt{=\ }\hlstd{\#}\hlopt{\symbol{123}}\hlstd{id}\hlopt{\symbol{125}}\\
\hllin{15\ }\hlstd{}\hlopt{\symbol{60}/}\hlstd{delete}\hlopt{\symbol{62}}\hlstd{} 
\mbox{}
\normalfont
\normalsize


如前所述,插入语句有一点多,它有一些属性和子元素用来处理主键的生成。
首先,如果你的数据库支持自动生成主键的字段(比如MySQL和SQL Server),那么你可以设置 useGeneratedKeys=”true”,并且设置keyProperty到你已经做好的目标属性上就行了。例如,如果上面的Author表已经对id使用了自动生成的列类型,那么语句可以修改为:

\noindent
\ttfamily
\hlstd{}\hllin{1\ }\hlopt{\symbol{60}}\hlstd{insert\ id}\hlopt{=}\hlstd{}\hlstr{"insertAuthor"}\hlstd{\ parameterType}\hlopt{=}\hlstd{}\hlstr{"domain.blog.Author"}\hlstd{\ \Righttorque\\
\hllin{2\ }useGeneratedKeys}\hlopt{=}\hlstd{}\hlstr{"true"}\hlstd{\\
\hllin{3\ }}\hlstd{\ \ \ \ }\hlstd{keyProperty}\hlopt{=}\hlstd{}\hlstr{"id"}\hlstd{}\hlopt{\symbol{62}}\\
\hllin{4\ }\hlstd{}\hlstd{\ \ }\hlstd{insert\ into\ }\hlkwd{Author\ }\hlstd{}\hlopt{(}\hlstd{username}\hlopt{,}\hlstd{password}\hlopt{,}\hlstd{email}\hlopt{,}\hlstd{bio}\hlopt{)}\\
\hllin{5\ }\hlstd{}\hlstd{\ \ }\hlstd{}\hlkwd{values\ }\hlstd{}\hlopt{(}\hlstd{\#}\hlopt{\symbol{123}}\hlstd{username}\hlopt{\symbol{125},}\hlstd{\#}\hlopt{\symbol{123}}\hlstd{password}\hlopt{\symbol{125},}\hlstd{\#}\hlopt{\symbol{123}}\hlstd{email}\hlopt{\symbol{125},}\hlstd{\#}\hlopt{\symbol{123}}\hlstd{bio}\hlopt{\symbol{125})}\\
\hllin{6\ }\hlstd{}\hlopt{\symbol{60}/}\hlstd{insert}\hlopt{\symbol{62}}\hlstd{} 
\mbox{}
\normalfont
\normalsize


MyBatis 有另外一种方法来处理数据库不支持自动生成类型,或者可能JDBC驱动不支持自动生成主键时的主键生成问题。

这里有一个简单(有点傻)的示例,它可以生成一个随机ID(可能你不会这么做,但是这展示了MyBatis处理问题的灵活性,因为它并不关心ID的生成):


\noindent
\ttfamily
\hlstd{}\hllin{1\ }\hlopt{\symbol{60}}\hlstd{insert\ id}\hlopt{=}\hlstd{}\hlstr{"insertAuthor"}\hlstd{\ parameterType}\hlopt{=}\hlstd{}\hlstr{"domain.blog.Author"}\hlstd{}\hlopt{\symbol{62}}\\
\hllin{2\ }\hlstd{}\hlstd{\ \ }\hlstd{}\hlopt{\symbol{60}}\hlstd{selectKey\ keyProperty}\hlopt{=}\hlstd{}\hlstr{"id"}\hlstd{\ resultType}\hlopt{=}\hlstd{}\hlstr{"int"}\hlstd{\ \Righttorque\\
\hllin{3\ }}\hlstd{\ \ }\hlstd{order}\hlopt{=}\hlstd{}\hlstr{"BEFORE"}\hlstd{}\hlopt{\symbol{62}}\\
\hllin{4\ }\hlstd{}\hlstd{\ \ \ \ }\hlstd{select\ }\hlkwd{CAST}\hlstd{}\hlopt{(}\hlstd{}\hlkwd{RANDOM}\hlstd{}\hlopt{(){*}}\hlstd{}\hlnum{1000000\ }\hlstd{}\hlkwa{as\ }\hlstd{INTEGER}\hlopt{)\ }\hlstd{a\ from\ SYSIBM}\hlopt{.}\Righttorque\\
\hllin{5\ }\hlstd{}\hlstd{\ \ \ \ }\hlstd{SYSDUMMY1\\
\hllin{6\ }}\hlstd{\ \ }\hlstd{}\hlopt{\symbol{60}/}\hlstd{selectKey}\hlopt{\symbol{62}}\\
\hllin{7\ }\hlstd{}\hlstd{\ \ }\hlstd{insert\ into\ Author\\
\hllin{8\ }}\hlstd{\ \ \ \ }\hlstd{}\hlopt{(}\hlstd{id}\hlopt{,\ }\hlstd{username}\hlopt{,\ }\hlstd{password}\hlopt{,\ }\hlstd{email}\hlopt{,}\hlstd{bio}\hlopt{,\ }\hlstd{favourite\symbol{95}section}\hlopt{)}\\
\hllin{9\ }\hlstd{}\hlstd{\ \ }\hlstd{values\\
\hllin{10\ }}\hlstd{\ \ \ \ }\hlstd{}\hlopt{(}\hlstd{\#}\hlopt{\symbol{123}}\hlstd{id}\hlopt{\symbol{125},\ }\hlstd{\#}\hlopt{\symbol{123}}\hlstd{username}\hlopt{\symbol{125},\ }\hlstd{\#}\hlopt{\symbol{123}}\hlstd{password}\hlopt{\symbol{125},\ }\hlstd{\#}\hlopt{\symbol{123}}\hlstd{email}\hlopt{\symbol{125},\ }\hlstd{\#}\hlopt{\symbol{123}}\hlstd{bio}\hlopt{\symbol{125},\ }\hlstd{\#}\hlopt{\symbol{123}}\Righttorque\\
\hllin{11\ }\hlstd{}\hlstd{\ \ \ \ }\hlstd{favouriteSection}\hlopt{,}\hlstd{jdbcType}\hlopt{=}\hlstd{VARCHAR}\hlopt{\symbol{125})}\\
\hllin{12\ }\hlstd{}\hlopt{\symbol{60}/}\hlstd{insert}\hlopt{\symbol{62}}\hlstd{} 
\mbox{}
\normalfont
\normalsize


在上面的示例中,selectKey 元素将会首先运行,Author的id会被设置,然后插入语句 会被调用。这给你了一个类似的行为在你的数据库中来处理自动生成的主键, 而不需要使你的 Java 代码变得复杂。

selectKey 元素描述如下:

\noindent
\ttfamily
\hlstd{}\hllin{1\ }\hlopt{\symbol{60}}\hlstd{selectKey\\
\hllin{2\ }}\hlstd{\ \ }\hlstd{keyProperty}\hlopt{=}\hlstd{}\hlstr{"id"}\hlstd{\\
\hllin{3\ }}\hlstd{\ \ }\hlstd{resultType}\hlopt{=}\hlstd{}\hlstr{"int"}\hlstd{\\
\hllin{4\ }}\hlstd{\ \ }\hlstd{order}\hlopt{=}\hlstd{}\hlstr{"BEFORE"}\hlstd{\\
\hllin{5\ }}\hlstd{\ \ }\hlstd{statementType}\hlopt{=}\hlstd{}\hlstr{"PREPARED"}\hlstd{}\hlopt{\symbol{62}}\hlstd{} 
\mbox{}
\normalfont
\normalsize


\subsection{Sql}
这个元素可以被用来定义可重用的SQL代码段,可以包含在其他语句中。比如:

\noindent
\ttfamily
\hlstd{}\hllin{1\ }\hlopt{\symbol{60}}\hlstd{sql\ id}\hlopt{=}\hlstd{}\hlstr{"userColumns"}\hlstd{}\hlopt{\symbol{62}\ }\hlstd{id}\hlopt{,}\hlstd{username}\hlopt{,}\hlstd{password\ }\hlopt{\symbol{60}/}\hlstd{sql}\hlopt{\symbol{62}}\hlstd{} 
\mbox{}
\normalfont
\normalsize


这个 SQL 片段可以被包含在其他语句中,例如:

\noindent
\ttfamily
\hlstd{}\hllin{1\ }\hlopt{\symbol{60}}\hlstd{select\ id}\hlopt{=}\hlstd{}\hlstr{"selectUsers"}\hlstd{\ parameterType}\hlopt{=}\hlstd{}\hlstr{"int"}\hlstd{\ \Righttorque\\
\hllin{2\ }resultType}\hlopt{=}\hlstd{}\hlstr{"hashmap"}\hlstd{}\hlopt{\symbol{62}}\\
\hllin{3\ }\hlstd{}\hlstd{\ \ }\hlstd{select\ }\hlopt{\symbol{60}}\hlstd{include\ refid}\hlopt{=}\hlstd{}\hlstr{"userColumns"}\hlstd{}\hlopt{/\symbol{62}}\\
\hllin{4\ }\hlstd{}\hlstd{\ \ }\hlstd{from\ some\symbol{95}table\\
\hllin{5\ }}\hlstd{\ \ }\hlstd{where\ id\ }\hlopt{=\ }\hlstd{\#}\hlopt{\symbol{123}}\hlstd{id}\hlopt{\symbol{125}}\\
\hllin{6\ }\hlstd{}\hlopt{\symbol{60}/}\hlstd{select}\hlopt{\symbol{62}}\hlstd{} 
\mbox{}
\normalfont
\normalsize

\subsection{Parameters}
在之前的语句中, 你已经看到了一些简单参数的示例。 MyBatis 中,参数是非常强大的元素。对于大概占了90\%全部情况的简单情况,参数并不会太复杂,比如:

\noindent
\ttfamily
\hlstd{}\hllin{1\ }\hlopt{\symbol{60}}\hlstd{select\ id}\hlopt{=}\hlstd{}\hlstr{"selectUsers"}\hlstd{\ parameterType}\hlopt{=}\hlstd{}\hlstr{"int"}\hlstd{\ \Righttorque\\
\hllin{2\ }resultType}\hlopt{=}\hlstd{}\hlstr{"User"}\hlstd{}\hlopt{\symbol{62}}\\
\hllin{3\ }\hlstd{}\hlstd{\ \ }\hlstd{select\ id}\hlopt{,\ }\hlstd{username}\hlopt{,\ }\hlstd{password\\
\hllin{4\ }}\hlstd{\ \ }\hlstd{from\ users\\
\hllin{5\ }}\hlstd{\ \ }\hlstd{where\ id\ }\hlopt{=\ }\hlstd{\#}\hlopt{\symbol{123}}\hlstd{id}\hlopt{\symbol{125}}\\
\hllin{6\ }\hlstd{}\hlopt{\symbol{60}/}\hlstd{select}\hlopt{\symbol{62}}\hlstd{} 
\mbox{}
\normalfont
\normalsize


上面的这个示例展示了一个非常简单的命名参数映射。参数类型被设置为“int” ,因此这个参数可以被设置成任何内容。 原生的类型或简单数据类型,比如整型和没有相关属性的字符串,因此它会完全用参数来替代。然而,如果你传递了一个复杂的对象,那么 MyBatis 的处理方式就会有一点不同。比如:

\noindent
\ttfamily
\hlstd{}\hllin{1\ }\hlopt{\symbol{60}}\hlstd{select\ id}\hlopt{=}\hlstd{}\hlstr{"selectUsers"}\hlstd{\ parameterType}\hlopt{=}\hlstd{}\hlstr{"int"}\hlstd{\ \Righttorque\\
\hllin{2\ }resultType}\hlopt{=}\hlstd{}\hlstr{"User"}\hlstd{}\hlopt{\symbol{62}}\\
\hllin{3\ }\hlstd{}\hlstd{\ \ }\hlstd{select\ id}\hlopt{,\ }\hlstd{username}\hlopt{,\ }\hlstd{password\\
\hllin{4\ }}\hlstd{\ \ }\hlstd{from\ users\\
\hllin{5\ }}\hlstd{\ \ }\hlstd{where\ id\ }\hlopt{=\ }\hlstd{\#}\hlopt{\symbol{123}}\hlstd{id}\hlopt{\symbol{125}}\\
\hllin{6\ }\hlstd{}\hlopt{\symbol{60}/}\hlstd{select}\hlopt{\symbol{62}}\hlstd{} 
\mbox{}
\normalfont
\normalsize


如果 User 类型的参数对象传递到了语句中,id、username 和 password 属性将会被找出来,然后它们的值就被传递到预处理语句的参数中。