% !Mode:: "TeX:UTF-8"

\chapter{The Name of the Game}
\section{Introduction}
\subsection{MyBatis是什么?}
MyBatis is a first class persistence framework with support for custom SQL, stored procedures and advanced mappings. MyBatis eliminates almost all of the JDBC code and manual setting of parameters and retrieval of results. MyBatis can use simple XML or Annotations for configuration and map primitives, Map interfaces and Java POJOs (Plain Old Java Objects) to database records.
\section{Getting Started}
Every MyBatis application centers around an instance of SqlSessionFactory. A SqlSessionFactory instance can be acquired by using the SqlSessionFactoryBuilder. SqlSessionFactoryBuilder can build a SqlSessionFactory instance from an XML configuration file, or from a custom prepared instance of the Configuration class.
\subsection{Building SqlSessionFactory from XML}
Building a SqlSessionFactory instance from an XML file is very simple. It is recommended that you use a classpath resource for this configuration, but you could use any InputStream instance, including one created from a literal file path or a file://URL. MyBatis includes a utility class, called Resources, that contains a number of methods that make it simpler to load resources from the classpath and other locations.

\noindent
\ttfamily
\hlstd{\hllin{1\ }String\ resource\ }\hlopt{=\ }\hlstd{}\hlstr{"org/mybatis/example/mybatis{-}config.xml"}\hlstd{}\hlopt{;}\\
\hllin{2\ }\hlstd{InputStream\ inputStream\ }\hlopt{=\ }\hlstd{Resources}\hlopt{.}\hlstd{}\hlkwd{getResourceAsStream}\hlstd{}\hlopt{(}\Righttorque\\
\hllin{3\ }\hlstd{resource}\hlopt{);}\\
\hllin{4\ }\hlstd{sqlSessionFactory\ }\hlopt{=\ }\hlstd{}\hlkwa{new\ }\hlstd{}\hlkwd{SqlSessionFactoryBuilder}\hlstd{}\hlopt{().}\hlstd{}\hlkwd{build}\hlstd{}\hlopt{(}\Righttorque\\
\hllin{5\ }\hlstd{inputStream}\hlopt{);}\\
\hllin{6\ }\hlstd{} 
\mbox{}
\normalfont
\normalsize


The configuration XML file contains settings for the core of the MyBatis system, including a DataSource for acquiring database Connection instances, as well as a TransactionManager for determining how transactions should be scoped and controlled. The full details of the XML configuration file can be found later in this document, but here is a simple example:

\noindent
\ttfamily
\hlstd{}\hllin{1\ }\hlopt{\symbol{60}}\hlstd{?xml\ version}\hlopt{=}\hlstd{}\hlstr{"1.0"}\hlstd{\ encoding}\hlopt{=}\hlstd{}\hlstr{"UTF{-}8"}\hlstd{\ ?}\hlopt{\symbol{62}}\\
\hllin{2\ }\hlstd{}\hlopt{\symbol{60}!}\hlstd{DOCTYPE\ configuration\\
\hllin{3\ }}\hlstd{\ \ }\hlstd{PUBLIC\ }\hlstr{"{-}//mybatis.org//DTD\ Config\ 3.0//EN"}\hlstd{\\
\hllin{4\ }}\hlstd{\ \ }\hlstd{}\hlstr{"http://mybatis.org/dtd/mybatis{-}3{-}config.dtd"}\hlstd{}\hlopt{\symbol{62}}\\
\hllin{5\ }\hlstd{}\hlopt{\symbol{60}}\hlstd{configuration}\hlopt{\symbol{62}}\\
\hllin{6\ }\hlstd{}\hlstd{\ \ }\hlstd{}\hlopt{\symbol{60}}\hlstd{environments\ }\hlkwa{default}\hlstd{}\hlopt{=}\hlstd{}\hlstr{"development"}\hlstd{}\hlopt{\symbol{62}}\\
\hllin{7\ }\hlstd{}\hlstd{\ \ \ \ }\hlstd{}\hlopt{\symbol{60}}\hlstd{environment\ id}\hlopt{=}\hlstd{}\hlstr{"development"}\hlstd{}\hlopt{\symbol{62}}\\
\hllin{8\ }\hlstd{}\hlstd{\ \ \ \ \ \ }\hlstd{}\hlopt{\symbol{60}}\hlstd{transactionManager\ type}\hlopt{=}\hlstd{}\hlstr{"JDBC"}\hlstd{}\hlopt{/\symbol{62}}\\
\hllin{9\ }\hlstd{}\hlstd{\ \ \ \ \ \ }\hlstd{}\hlopt{\symbol{60}}\hlstd{dataSource\ type}\hlopt{=}\hlstd{}\hlstr{"POOLED"}\hlstd{}\hlopt{\symbol{62}}\\
\hllin{10\ }\hlstd{}\hlstd{\ \ \ \ \ \ \ \ }\hlstd{}\hlopt{\symbol{60}}\hlstd{}\hlkwa{property\ }\hlstd{name}\hlopt{=}\hlstd{}\hlstr{"driver"}\hlstd{\ value}\hlopt{=}\hlstd{}\hlstr{"\$\symbol{123}driver\symbol{125}"}\hlstd{}\hlopt{/\symbol{62}}\\
\hllin{11\ }\hlstd{}\hlstd{\ \ \ \ \ \ \ \ }\hlstd{}\hlopt{\symbol{60}}\hlstd{}\hlkwa{property\ }\hlstd{name}\hlopt{=}\hlstd{}\hlstr{"url"}\hlstd{\ value}\hlopt{=}\hlstd{}\hlstr{"\$\symbol{123}url\symbol{125}"}\hlstd{}\hlopt{/\symbol{62}}\\
\hllin{12\ }\hlstd{}\hlstd{\ \ \ \ \ \ \ \ }\hlstd{}\hlopt{\symbol{60}}\hlstd{}\hlkwa{property\ }\hlstd{name}\hlopt{=}\hlstd{}\hlstr{"username"}\hlstd{\ value}\hlopt{=}\hlstd{}\hlstr{"\$\symbol{123}username\symbol{125}"}\hlstd{}\hlopt{/\symbol{62}}\\
\hllin{13\ }\hlstd{}\hlstd{\ \ \ \ \ \ \ \ }\hlstd{}\hlopt{\symbol{60}}\hlstd{}\hlkwa{property\ }\hlstd{name}\hlopt{=}\hlstd{}\hlstr{"password"}\hlstd{\ value}\hlopt{=}\hlstd{}\hlstr{"\$\symbol{123}password\symbol{125}"}\hlstd{}\hlopt{/\symbol{62}}\\
\hllin{14\ }\hlstd{}\hlstd{\ \ \ \ \ \ }\hlstd{}\hlopt{\symbol{60}/}\hlstd{dataSource}\hlopt{\symbol{62}}\\
\hllin{15\ }\hlstd{}\hlstd{\ \ \ \ }\hlstd{}\hlopt{\symbol{60}/}\hlstd{environment}\hlopt{\symbol{62}}\\
\hllin{16\ }\hlstd{}\hlstd{\ \ }\hlstd{}\hlopt{\symbol{60}/}\hlstd{environments}\hlopt{\symbol{62}}\\
\hllin{17\ }\hlstd{}\hlstd{\ \ }\hlstd{}\hlopt{\symbol{60}}\hlstd{mappers}\hlopt{\symbol{62}}\\
\hllin{18\ }\hlstd{}\hlstd{\ \ \ \ }\hlstd{}\hlopt{\symbol{60}}\hlstd{mapper\ resource}\hlopt{=}\hlstd{}\hlstr{"org/mybatis/example/BlogMapper.xml"}\hlstd{}\hlopt{/\symbol{62}}\\
\hllin{19\ }\hlstd{}\hlstd{\ \ }\hlstd{}\hlopt{\symbol{60}/}\hlstd{mappers}\hlopt{\symbol{62}}\\
\hllin{20\ }\hlstd{}\hlopt{\symbol{60}/}\hlstd{configuration}\hlopt{\symbol{62}}\hlstd{} 
\mbox{}
\normalfont
\normalsize


While there is a lot more to the XML configuration file, the above example points out the most critical parts. Notice the XML header, required to validate the XML document. The body of the environment element contains the environment configuration for transaction management and connection pooling. The mappers element contains a list of mappers – the XML files that contain the SQL code and mapping definitions.
\subsection{Building SqlSessionFactory without XML}
If you prefer to directly build the configuration from Java, rather than XML, or create your own configuration builder, MyBatis provides a complete Configuration class that provides all of the same configuration options as the XML file.

\noindent
\ttfamily
\hlstd{\hllin{1\ }DataSource\ dataSource\ }\hlopt{=\ }\hlstd{BlogDataSourceFactory}\hlopt{.}\Righttorque\\
\hllin{2\ }\hlstd{}\hlkwd{getBlogDataSource}\hlstd{}\hlopt{();}\\
\hllin{3\ }\hlstd{TransactionFactory\ transactionFactory\ }\hlopt{=\ }\hlstd{}\hlkwa{new\ }\Righttorque\\
\hllin{4\ }\hlstd{}\hlkwd{JdbcTransactionFactory}\hlstd{}\hlopt{();}\\
\hllin{5\ }\hlstd{Environment\ environment\ }\hlopt{=\ }\hlstd{}\hlkwa{new\ }\hlstd{}\hlkwd{Environment}\hlstd{}\hlopt{(}\hlstd{}\hlstr{"development"}\hlstd{}\hlopt{,\ }\Righttorque\\
\hllin{6\ }\hlstd{transactionFactory}\hlopt{,\ }\hlstd{dataSource}\hlopt{);}\\
\hllin{7\ }\hlstd{Configuration\ configuration\ }\hlopt{=\ }\hlstd{}\hlkwa{new\ }\hlstd{}\hlkwd{Configuration}\hlstd{}\hlopt{(}\hlstd{environment}\hlopt{);}\\
\hllin{8\ }\hlstd{configuration}\hlopt{.}\hlstd{}\hlkwd{addMapper}\hlstd{}\hlopt{(}\hlstd{BlogMapper}\hlopt{.}\hlstd{}\hlkwa{class}\hlstd{}\hlopt{);}\\
\hllin{9\ }\hlstd{SqlSessionFactory\ sqlSessionFactory\ }\hlopt{=\ }\hlstd{}\hlkwa{new\ }\Righttorque\\
\hllin{10\ }\hlstd{}\hlkwd{SqlSessionFactoryBuilder}\hlstd{}\hlopt{().}\hlstd{}\hlkwd{build}\hlstd{}\hlopt{(}\hlstd{configuration}\hlopt{);}\\
\hllin{11\ }\hlstd{} 
\mbox{}
\normalfont
\normalsize


Notice in this case the configuration is adding a mapper class. Mapper classes are Java classes that contain SQL Mapping Annotations that avoid the need for XML. However, due to some limitations of Java Annotations and the complexity of some MyBatis mappings, XML mapping is still required for the most advanced mappings (e.g. Nested Join Mapping). For this reason, MyBatis will automatically look for and load a peer XML file if it exists (in this case, BlogMapper.xml would be loaded based on the classpath and name of BlogMapper.class). More on this later.

\subsection{Acquiring a SqlSession from SqlSessionFactory}

Now that you have a SqlSessionFactory, as the name suggests, you can acquire an instance of SqlSession. The SqlSession contains absolutely every method needed to execute SQL commands against the database. You can execute mapped SQL statements directly against the SqlSession instance. For exmaple:

\input{Coding/Code4.java.tex}

While this approach works, and is familiar to users of previous versions of MyBatis, there is now a cleaner approach. Using an interface (e.g. BlogMapper.class) that properly describes the parameter and return value for a given statement, you can now execute cleaner and more type safe code, without error prone string literals and casting.
For example:

\noindent
\ttfamily
\hlstd{\hllin{1\ }SqlSession\ session\ }\hlopt{=\ }\hlstd{sqlSessionFactory}\hlopt{.}\hlstd{}\hlkwd{openSession}\hlstd{}\hlopt{();}\\
\hllin{2\ }\hlstd{}\hlkwa{try\ }\hlstd{}\hlopt{\symbol{123}}\\
\hllin{3\ }\hlstd{}\hlstd{\ \ \ \ }\hlstd{BlogMapper\ mapper\ }\hlopt{=\ }\hlstd{session}\hlopt{.}\hlstd{}\hlkwd{getMapper}\hlstd{}\hlopt{(}\hlstd{BlogMapper}\hlopt{.}\hlstd{}\hlkwa{class}\hlstd{}\hlopt{);}\\
\hllin{4\ }\hlstd{}\hlstd{\ \ \ \ }\hlstd{Blog\ blog\ }\hlopt{=\ }\hlstd{mapper}\hlopt{.}\hlstd{}\hlkwd{selectBlog}\hlstd{}\hlopt{(}\hlstd{}\hlnum{101}\hlstd{}\hlopt{);}\\
\hllin{5\ }\hlstd{}\hlopt{\symbol{125}}\\
\hllin{6\ }\hlstd{}\hlkwa{finally\ }\hlstd{}\hlopt{\symbol{123}}\\
\hllin{7\ }\hlstd{}\hlstd{\ \ \ \ }\hlstd{session}\hlopt{.}\hlstd{}\hlkwd{close}\hlstd{}\hlopt{();}\\
\hllin{8\ }\hlstd{}\hlopt{\symbol{125}}\\
\hllin{9\ }\hlstd{} 
\mbox{}
\normalfont
\normalsize


Now let's explore what exactly is being executed here.
\subsection{Exploring Mapped SQL Statements}
At this point you may be wondering what exactly is being executed by the SqlSession or Mapper class. The topic of Mapped SQL Statements is a big one, and that topic will likely dominate the majority of this documentation. But to give you an idea of what exactly is being run, here are a couple of examples.

In either of the examples above, the statements could have been defined by either XML or Annotations. Let's take a look at XML first. The full set of features provided by MyBatis can be realized by using the XML based mapping language that has made MyBatis popular over the years.If you've used MyBatis before, the concept will be familiar to you, but there have been numerous improvements to the XML mapping documents that will become clear later. Here is an example of an XML based mapped statement that would satisfy the above SqlSession calls.

\noindent
\ttfamily
\hlstd{}\hllin{1\ }\hlopt{\symbol{60}}\hlstd{?xml\ version}\hlopt{=}\hlstd{}\hlstr{"1.0"}\hlstd{\ encoding}\hlopt{=}\hlstd{}\hlstr{"UTF{-}8"}\hlstd{\ ?}\hlopt{\symbol{62}}\\
\hllin{2\ }\hlstd{}\hlopt{\symbol{60}!}\hlstd{DOCTYPE\ mapper\\
\hllin{3\ }}\hlstd{\ \ }\hlstd{PUBLIC\ }\hlstr{"{-}//mybatis.org//DTD\ Mapper\ 3.0//EN"}\hlstd{\\
\hllin{4\ }}\hlstd{\ \ }\hlstd{}\hlstr{"http://mybatis.org/dtd/mybatis{-}3{-}mapper.dtd"}\hlstd{}\hlopt{\symbol{62}}\\
\hllin{5\ }\hlstd{}\hlopt{\symbol{60}}\hlstd{mapper\ namespace}\hlopt{=}\hlstd{}\hlstr{"org.mybatis.example.BlogMapper"}\hlstd{}\hlopt{\symbol{62}}\\
\hllin{6\ }\hlstd{}\hlstd{\ \ }\hlstd{}\hlopt{\symbol{60}}\hlstd{select\ id}\hlopt{=}\hlstd{}\hlstr{"selectBlog"}\hlstd{\ parameterType}\hlopt{=}\hlstd{}\hlstr{"int"}\hlstd{\ \Righttorque\\
\hllin{7\ }}\hlstd{\ \ }\hlstd{resultType}\hlopt{=}\hlstd{}\hlstr{"Blog"}\hlstd{}\hlopt{\symbol{62}}\\
\hllin{8\ }\hlstd{}\hlstd{\ \ \ \ }\hlstd{select\ }\hlopt{{*}\ }\hlstd{from\ Blog\ where\ id\ }\hlopt{=\ }\hlstd{\#}\hlopt{\symbol{123}}\hlstd{id}\hlopt{\symbol{125}}\\
\hllin{9\ }\hlstd{}\hlstd{\ \ }\hlstd{}\hlopt{\symbol{60}/}\hlstd{select}\hlopt{\symbol{62}}\\
\hllin{10\ }\hlstd{}\hlopt{\symbol{60}/}\hlstd{mapper}\hlopt{\symbol{62}}\hlstd{} 
\mbox{}
\normalfont
\normalsize


While this looks like a lot of overhead for this simple example, it is actually very light. You can define as many mapped statements in a single mapper XML file as you like, so you get a lot of mileage out of the XML header and doctype declaration. The rest of the file is pretty self explanatory. It defines a name for the mapped statement “selectBlog”, in the namespace “org.mybatis.example.BlogMapper”, which would allow you to call it by specifying the fully qualified name of “org.mybatis.example.BlogMapper.selectBlog”, as we did above in the following example:

\noindent
\ttfamily
\hlstd{\hllin{1\ }Blog\ blog\ }\hlopt{=\ (}\hlstd{Blog}\hlopt{)\ }\hlstd{session}\hlopt{.}\hlstd{}\hlkwd{selectOne}\hlstd{}\hlopt{(}\hlstd{}\hlstr{"org.mybatis.example.}\Righttorque\\
\hllin{2\ }\hlstr{BlogMapper.selectBlog"}\hlstd{}\hlopt{,\ }\hlstd{}\hlnum{101}\hlstd{}\hlopt{);}\hlstd{} 
\mbox{}
\normalfont
\normalsize


Notice how similar this is to calling a method on a fully qualified Java class, and there's a reason for that. This name can be directly mapped to a Mapper class of the same name as the namespace, with a method that matches the name, parameter, and return type as the mapped select statement. This allows you to very simply call the method against the Mapper interface as you sawabove, but here it is again in the following example:

\noindent
\ttfamily
\hlstd{\hllin{1\ }BlogMapper\ mapper\ }\hlopt{=\ }\hlstd{session}\hlopt{.}\hlstd{}\hlkwd{getMapper}\hlstd{}\hlopt{(}\hlstd{BlogMapper}\hlopt{.}\hlstd{}\hlkwa{class}\hlstd{}\hlopt{);}\\
\hllin{2\ }\hlstd{Blog\ blog\ }\hlopt{=\ }\hlstd{mapper}\hlopt{.}\hlstd{}\hlkwd{selectBlog}\hlstd{}\hlopt{(}\hlstd{}\hlnum{101}\hlstd{}\hlopt{);}\\
\hllin{3\ }\hlstd{} 
\mbox{}
\normalfont
\normalsize


The second approach has a lot of advantages. First, it doesn't depend on a string literal, so it's much safer. Second, if your IDE has code completion, you can leverage that when navigating your mapped SQL statements.

NOTE A note about namespaces.

Namespaces were optional in previous versions of MyBatis, which was confusing and unhelpful. Namespaces are now required and have a purpose beyond simply isolating statements with longer, fully-qualified names.

Namespaces enable the interface bindings as you see here, and even if you don’t think you’ll use them today, you should follow these practices laid out here in case you change your mind. Using the namespace once, and putting it in a proper Java package namespace will clean up your code and improve the usability of MyBatis in the long term.

Name Resolution: To reduce the amount of typing, MyBatis uses the following name resolution rules for all named configuration elements, including statements, result maps, caches, etc.

•	Fully qualified names (e.g. “com.mypackage.MyMapper.selectAllThings”) are looked up directly and used if found.

•	Short names (e.g. “selectAllThings”) can be used to reference any unambiguous entry. However if there are two or more (e.g. “com.foo.selectAllThings and com.bar.selectAllThings”), then you will receive an error reporting that the short name is ambiguous and therefore must be fully qualified.


There's one more trick to Mapper classes like BlogMapper. Their mapped statements don't need to be mapped with XML at all. Instead they can use Java Annotations. For example, the XML above could be eliminated and replaced with:

\noindent
\ttfamily
\hlstd{}\hllin{1\ }\hlkwa{package\ }\hlstd{org}\hlopt{.}\hlstd{mybatis}\hlopt{.}\hlstd{example}\hlopt{;}\\
\hllin{2\ }\hlstd{}\hlkwa{public\ interface\ }\hlstd{BlogMapper\ }\hlopt{\symbol{123}}\\
\hllin{3\ }\hlstd{}\hlstd{\ \ \ \ }\hlstd{}\hlkwc{@Select}\hlstd{}\hlopt{(}\hlstd{}\hlstr{"SELECT\ {*}\ FROM\ blog\ WHERE\ id\ =\ \#\symbol{123}id\symbol{125}"}\hlstd{}\hlopt{)}\\
\hllin{4\ }\hlstd{}\hlstd{\ \ \ \ }\hlstd{Blog\ }\hlkwd{selectBlog}\hlstd{}\hlopt{(}\hlstd{}\hlkwb{int\ }\hlstd{id}\hlopt{);}\\
\hllin{5\ }\hlstd{}\hlopt{\symbol{125}}\\
\hllin{6\ }\hlstd{} 
\mbox{}
\normalfont
\normalsize


The annotations are a lot cleaner for simple statements, however, Java Annotations are both limited and messier for more complicated statements. Therefore, if you have to do anything complicated, you're better off with XML mapped statements.

It will be up to you and your project team to determine which is right for you, and how important it is to you that your mapped statements be defined in a consistent way. That said, you're never locked into a single approach. You can very easily migrate Annotation based Mapped Statements to XML and vice versa.
\subsection{Scope and Lifecycle}
It's very important to understand the various scopes and lifecycles classes we've discussed so far. Using them incorrectly can cause severe concurrency problems.
\subsubsection{SqlSessionFactoryBuilder}
This class can be instantiated, used and thrown away. There is no need to keep it around once you've created your SqlSessionFactory. Therefore the best scope for instances of SqlSessionFactoryBuilder is method scope (i.e. a local method variable). You can reuse the SqlSessionFactoryBuilder to build multiple SqlSessionFactory instances, but it's still best not to keep it around to ensure that all of the XML parsing resources are freed up for more important things.
\subsubsection{SqlSessionFactory}
Once created, the SqlSessionFactory should exist for the duration of your application execution. There should be little or no reason to ever dispose of it or recreate it. It's a best practice to not rebuild the SqlSessionFactory multiple times in an application run. Doing so should be considered a “bad smell”. Therefore the best scope of SqlSessionFactory is application scope. This can be achieved a number of ways. The simplest is to use a Singleton pattern or Static Singleton pattern.
\subsubsection{SqlSession}
Each thread should have its own instance of SqlSession. Instances of SqlSession are not to be shared and are not thread safe. Therefore the best scope is request or method scope. Never keep references to a SqlSession instance in a static field or even an instance field of a class. Never keep references to a SqlSession in any sort of managed scope, such as HttpSession of of the Servlet framework. If you're using a web framework of any sort, consider the SqlSession to follow a similar scope to that of an HTTP request. In other words, upon receiving an HTTP request, you can open a SqlSession, then upon returning the response, you can close it. Closing the session is very important. You should always ensure that it's closed within a finally block. The following is the standard pattern for ensuring that SqlSessions are closed:

\noindent
\ttfamily
\hlstd{\hllin{1\ }SqlSession\ session\ }\hlopt{=\ }\hlstd{sqlSessionFactory}\hlopt{.}\hlstd{}\hlkwd{openSession}\hlstd{}\hlopt{();}\\
\hllin{2\ }\hlstd{}\hlkwa{try\ }\hlstd{}\hlopt{\symbol{123}}\\
\hllin{3\ }\hlstd{}\hlstd{\ \ \ \ }\hlstd{}\hlslc{//\ do\ work}\\
\hllin{4\ }\hlstd{}\hlopt{\symbol{125}\ }\hlstd{}\hlkwa{finally\ }\hlstd{}\hlopt{\symbol{123}}\\
\hllin{5\ }\hlstd{}\hlstd{\ \ \ \ }\hlstd{session}\hlopt{.}\hlstd{}\hlkwd{close}\hlstd{}\hlopt{();}\\
\hllin{6\ }\hlstd{}\hlopt{\symbol{125}}\\
\hllin{7\ }\hlstd{} 
\mbox{}
\normalfont
\normalsize


Using this pattern consistently throughout your code will ensure that all database resources are properly closed.
\subsubsection{Mapper Instances}
Mappers are interfaces that you create to bind to your mapped statements. Instances of the mapper interfaces are acquired from the SqlSession. As such, technically the broadest scope of any mapper instance is the same as the SqlSession from which they were requested. However, the best scope for mapper instances is method scope. That is, they should be requested within the method that they are used, and then be discarded. They do not need to be closed explicitly. While it's not a problem to keep them around throughout a request, similar to the SqlSession, you might find that managing too many resources at this level will quickly get out of hand. Keep it simple, keep Mappers in the method scope. The following example demonstrates this practice.

\noindent
\ttfamily
\hlstd{\hllin{1\ }SqlSession\ session\ }\hlopt{=\ }\hlstd{sqlSessionFactory}\hlopt{.}\hlstd{}\hlkwd{openSession}\hlstd{}\hlopt{();}\\
\hllin{2\ }\hlstd{}\hlkwa{try\ }\hlstd{}\hlopt{\symbol{123}}\\
\hllin{3\ }\hlstd{}\hlstd{\ \ \ \ }\hlstd{}\hlslc{//\ do\ work}\\
\hllin{4\ }\hlstd{}\hlopt{\symbol{125}\ }\hlstd{}\hlkwa{finally\ }\hlstd{}\hlopt{\symbol{123}}\\
\hllin{5\ }\hlstd{}\hlstd{\ \ \ \ }\hlstd{session}\hlopt{.}\hlstd{}\hlkwd{close}\hlstd{}\hlopt{();}\\
\hllin{6\ }\hlstd{}\hlopt{\symbol{125}}\\
\hllin{7\ }\hlstd{} 
\mbox{}
\normalfont
\normalsize


NOTE Object lifecycle and Dependency Injection Frameworks

Dependency Injection frameworks can create thread safe, transactional SqlSessions and mappers and inject them directly into your beans so you can just forget about their lifecycle. You may want to have a look at MyBatis-Spring or MyBatis-Guice sub-projects to know more about using MyBatis with DI frameworks.

\section{Mapper XML Files}
The true power of MyBatis is in the Mapped Statements. This is where the magic happens. For all of their power, the Mapper XML files are relatively simple. Certainly if you were to compare them to the equivalent JDBC code, you would immediately see a savings of 95\% of the code. MyBatis was built to focus on the SQL, and does its best to stay out of your way.

The Mapper XML files have only a few first class elements (in the order that they should be defined):

•	cache – Configuration of the cache for a given namespace.

•	cache-ref – Reference to a cache configuration from another namespace.

•	resultMap – The most complicated and powerful element that describes how to load your objects from the database result sets.

•	parameterMap – Deprecated! Old-school way to map parameters. Inline parameters are preferred and this element may be removed in the future. Not documented here.

•	sql – A reusable chunk of SQL that can be referenced by other statements.

•	insert – A mapped INSERT statement.

•	update – A mapped UPDATE statement.

•	delete – A mapped DELETE statement.

•	select – A mapped SELECT statement.

The next sections will describe each of these elements in detail, starting with the statements themselves.
\subsection{Select}
The select statement is one of the most popular elements that you'll use in MyBatis. Putting data in a database isn't terribly valuable until you get it back out, so most applications query far more than they modify the data. For every insert, update or delete, there is probably many selects. This is one of the founding principles of MyBatis, and is the reason so much focus and effort was placed on querying and result mapping. The select element is quite simple for simple cases. For example:

\noindent
\ttfamily
\hlstd{}\hllin{1\ }\hlopt{\symbol{60}}\hlstd{select\ id}\hlopt{=}\hlstd{}\hlstr{"selectPerson"}\hlstd{\ parameterType}\hlopt{=}\hlstd{}\hlstr{"int"}\hlstd{\ \Righttorque\\
\hllin{2\ }resultType}\hlopt{=}\hlstd{}\hlstr{"hashmap"}\hlstd{}\hlopt{\symbol{62}}\\
\hllin{3\ }\hlstd{}\hlstd{\ \ }\hlstd{SELECT\ }\hlopt{{*}\ }\hlstd{FROM\ PERSON\ WHERE\ ID\ }\hlopt{=\ }\hlstd{\#}\hlopt{\symbol{123}}\hlstd{id}\hlopt{\symbol{125}}\\
\hllin{4\ }\hlstd{}\hlopt{\symbol{60}/}\hlstd{select}\hlopt{\symbol{62}}\hlstd{} 
\mbox{}
\normalfont
\normalsize


This statement is called selectPerson, takes a parameter of type int (or Integer), and returns a HashMap keyed by column names mapped to row values.
Notice the parameter notation: \#\{id\} 
This tells MyBatis to create a PreparedStatement parameter. With JDBC, such a parameter would be identified by a "?" in SQL passed to a new PreparedStatement, something like this:

\noindent
\ttfamily
\hlstd{}\hllin{1\ }\hlslc{//\ Similar\ JDBC\ code,\ NOT\ MyBatis…}\\
\hllin{2\ }\hlstd{String\ selectPerson\ }\hlopt{=\ }\hlstd{}\hlstr{"SELECT\ {*}\ FROM\ PERSON\ WHERE\ ID=?"}\hlstd{}\hlopt{;}\\
\hllin{3\ }\hlstd{PreparedStatement\ ps\ }\hlopt{=\ }\hlstd{conn}\hlopt{.}\hlstd{}\hlkwd{prepareStatement}\hlstd{}\hlopt{(}\hlstd{selectPerson}\hlopt{);}\\
\hllin{4\ }\hlstd{ps}\hlopt{.}\hlstd{}\hlkwd{setInt}\hlstd{}\hlopt{(}\hlstd{}\hlnum{1}\hlstd{}\hlopt{,}\hlstd{id}\hlopt{);}\\
\hllin{5\ }\hlstd{} 
\mbox{}
\normalfont
\normalsize


Of course, there's a lot more code required by JDBC alone to extract the results and map them to an instance of an object, which is what MyBatis saves you from having to do. There's a lot more to know about parameter and result mapping. Those details warrant their own section, which follows later in this section.

The select element has more attributes that allow you to configure the details of how each statement should behave.

\noindent
\ttfamily
\hlstd{}\hllin{1\ }\hlopt{\symbol{60}}\hlstd{select\\
\hllin{2\ }}\hlstd{\ \ }\hlstd{id}\hlopt{=}\hlstd{}\hlstr{"selectPerson"}\hlstd{\\
\hllin{3\ }}\hlstd{\ \ }\hlstd{parameterType}\hlopt{=}\hlstd{}\hlstr{"int"}\hlstd{\\
\hllin{4\ }}\hlstd{\ \ }\hlstd{parameterMap}\hlopt{=}\hlstd{}\hlstr{"deprecated"}\hlstd{\\
\hllin{5\ }}\hlstd{\ \ }\hlstd{resultType}\hlopt{=}\hlstd{}\hlstr{"hashmap"}\hlstd{\\
\hllin{6\ }}\hlstd{\ \ }\hlstd{resultMap}\hlopt{=}\hlstd{}\hlstr{"personResultMap"}\hlstd{\\
\hllin{7\ }}\hlstd{\ \ }\hlstd{flushCache}\hlopt{=}\hlstd{}\hlstr{"false"}\hlstd{\\
\hllin{8\ }}\hlstd{\ \ }\hlstd{useCache}\hlopt{=}\hlstd{}\hlstr{"true"}\hlstd{\\
\hllin{9\ }}\hlstd{\ \ }\hlstd{timeout}\hlopt{=}\hlstd{}\hlstr{"10000"}\hlstd{\\
\hllin{10\ }}\hlstd{\ \ }\hlstd{fetchSize}\hlopt{=}\hlstd{}\hlstr{"256"}\hlstd{\\
\hllin{11\ }}\hlstd{\ \ }\hlstd{statementType}\hlopt{=}\hlstd{}\hlstr{"PREPARED"}\hlstd{\\
\hllin{12\ }}\hlstd{\ \ }\hlstd{resultSetType}\hlopt{=}\hlstd{}\hlstr{"FORWARD\symbol{95}ONLY"}\hlstd{}\hlopt{\symbol{62}}\hlstd{} 
\mbox{}
\normalfont
\normalsize


\subsection{Insert, Update and Delete}
The data modification statements insert, update and delete are very similar in their implementation:

\noindent
\ttfamily
\hlstd{}\hllin{1\ }\hlopt{\symbol{60}}\hlstd{insert\\
\hllin{2\ }}\hlstd{\ \ }\hlstd{id}\hlopt{=}\hlstd{}\hlstr{"insertAuthor"}\hlstd{\\
\hllin{3\ }}\hlstd{\ \ }\hlstd{parameterType}\hlopt{=}\hlstd{}\hlstr{"domain.blog.Author"}\hlstd{\\
\hllin{4\ }}\hlstd{\ \ }\hlstd{flushCache}\hlopt{=}\hlstd{}\hlstr{"true"}\hlstd{\\
\hllin{5\ }}\hlstd{\ \ }\hlstd{statementType}\hlopt{=}\hlstd{}\hlstr{"PREPARED"}\hlstd{\\
\hllin{6\ }}\hlstd{\ \ }\hlstd{keyProperty}\hlopt{=}\hlstd{}\hlstr{""}\hlstd{\\
\hllin{7\ }}\hlstd{\ \ }\hlstd{keyColumn}\hlopt{=}\hlstd{}\hlstr{""}\hlstd{\\
\hllin{8\ }}\hlstd{\ \ }\hlstd{useGeneratedKeys}\hlopt{=}\hlstd{}\hlstr{""}\hlstd{\\
\hllin{9\ }}\hlstd{\ \ }\hlstd{timeout}\hlopt{=}\hlstd{}\hlstr{"20000"}\hlstd{}\hlopt{\symbol{62}}\\
\hllin{10\ }\hlstd{}\\
\hllin{11\ }\hlopt{\symbol{60}}\hlstd{update\\
\hllin{12\ }}\hlstd{\ \ }\hlstd{id}\hlopt{=}\hlstd{}\hlstr{"insertAuthor"}\hlstd{\\
\hllin{13\ }}\hlstd{\ \ }\hlstd{parameterType}\hlopt{=}\hlstd{}\hlstr{"domain.blog.Author"}\hlstd{\\
\hllin{14\ }}\hlstd{\ \ }\hlstd{flushCache}\hlopt{=}\hlstd{}\hlstr{"true"}\hlstd{\\
\hllin{15\ }}\hlstd{\ \ }\hlstd{statementType}\hlopt{=}\hlstd{}\hlstr{"PREPARED"}\hlstd{\\
\hllin{16\ }}\hlstd{\ \ }\hlstd{timeout}\hlopt{=}\hlstd{}\hlstr{"20000"}\hlstd{}\hlopt{\symbol{62}}\\
\hllin{17\ }\hlstd{}\\
\hllin{18\ }\hlopt{\symbol{60}}\hlstd{delete\\
\hllin{19\ }}\hlstd{\ \ }\hlstd{id}\hlopt{=}\hlstd{}\hlstr{"insertAuthor"}\hlstd{\\
\hllin{20\ }}\hlstd{\ \ }\hlstd{parameterType}\hlopt{=}\hlstd{}\hlstr{"domain.blog.Author"}\hlstd{\\
\hllin{21\ }}\hlstd{\ \ }\hlstd{flushCache}\hlopt{=}\hlstd{}\hlstr{"true"}\hlstd{\\
\hllin{22\ }}\hlstd{\ \ }\hlstd{statementType}\hlopt{=}\hlstd{}\hlstr{"PREPARED"}\hlstd{\\
\hllin{23\ }}\hlstd{\ \ }\hlstd{timeout}\hlopt{=}\hlstd{}\hlstr{"20000"}\hlstd{}\hlopt{\symbol{62}}\hlstd{} 
\mbox{}
\normalfont
\normalsize


The following are some examples of insert, update and delete statements.

\noindent
\ttfamily
\hlstd{}\hllin{1\ }\hlopt{\symbol{60}}\hlstd{insert\ id}\hlopt{=}\hlstd{}\hlstr{"insertAuthor"}\hlstd{\ parameterType}\hlopt{=}\hlstd{}\hlstr{"domain.blog.Author"}\hlstd{}\hlopt{\symbol{62}}\\
\hllin{2\ }\hlstd{}\hlstd{\ \ }\hlstd{insert\ into\ }\hlkwd{Author\ }\hlstd{}\hlopt{(}\hlstd{id}\hlopt{,}\hlstd{username}\hlopt{,}\hlstd{password}\hlopt{,}\hlstd{email}\hlopt{,}\hlstd{bio}\hlopt{)}\\
\hllin{3\ }\hlstd{}\hlstd{\ \ }\hlstd{}\hlkwd{values\ }\hlstd{}\hlopt{(}\hlstd{\#}\hlopt{\symbol{123}}\hlstd{id}\hlopt{\symbol{125},}\hlstd{\#}\hlopt{\symbol{123}}\hlstd{username}\hlopt{\symbol{125},}\hlstd{\#}\hlopt{\symbol{123}}\hlstd{password}\hlopt{\symbol{125},}\hlstd{\#}\hlopt{\symbol{123}}\hlstd{email}\hlopt{\symbol{125},}\hlstd{\#}\hlopt{\symbol{123}}\hlstd{bio}\hlopt{\symbol{125})}\\
\hllin{4\ }\hlstd{}\hlopt{\symbol{60}/}\hlstd{insert}\hlopt{\symbol{62}}\\
\hllin{5\ }\hlstd{}\hlopt{\symbol{60}}\hlstd{update\ id}\hlopt{=}\hlstd{}\hlstr{"updateAuthor"}\hlstd{\ parameterType}\hlopt{=}\hlstd{}\hlstr{"domain.blog.Author"}\hlstd{}\hlopt{\symbol{62}}\\
\hllin{6\ }\hlstd{}\hlstd{\ \ }\hlstd{update\ Author\ set\\
\hllin{7\ }}\hlstd{\ \ \ \ }\hlstd{username\ }\hlopt{=\ }\hlstd{\#}\hlopt{\symbol{123}}\hlstd{username}\hlopt{\symbol{125},}\\
\hllin{8\ }\hlstd{}\hlstd{\ \ \ \ }\hlstd{password\ }\hlopt{=\ }\hlstd{\#}\hlopt{\symbol{123}}\hlstd{password}\hlopt{\symbol{125},}\\
\hllin{9\ }\hlstd{}\hlstd{\ \ \ \ }\hlstd{email\ }\hlopt{=\ }\hlstd{\#}\hlopt{\symbol{123}}\hlstd{email}\hlopt{\symbol{125},}\\
\hllin{10\ }\hlstd{}\hlstd{\ \ \ \ }\hlstd{bio\ }\hlopt{=\ }\hlstd{\#}\hlopt{\symbol{123}}\hlstd{bio}\hlopt{\symbol{125}}\\
\hllin{11\ }\hlstd{}\hlstd{\ \ }\hlstd{where\ id\ }\hlopt{=\ }\hlstd{\#}\hlopt{\symbol{123}}\hlstd{id}\hlopt{\symbol{125}}\\
\hllin{12\ }\hlstd{}\hlopt{\symbol{60}/}\hlstd{update}\hlopt{\symbol{62}}\\
\hllin{13\ }\hlstd{}\hlopt{\symbol{60}}\hlstd{delete\ id}\hlopt{=}\hlstd{}\hlstr{"deleteAuthor"}\hlstd{\ parameterType}\hlopt{=}\hlstd{}\hlstr{"int"}\hlstd{}\hlopt{\symbol{62}}\\
\hllin{14\ }\hlstd{}\hlstd{\ \ }\hlstd{delete\ from\ Author\ where\ id\ }\hlopt{=\ }\hlstd{\#}\hlopt{\symbol{123}}\hlstd{id}\hlopt{\symbol{125}}\\
\hllin{15\ }\hlstd{}\hlopt{\symbol{60}/}\hlstd{delete}\hlopt{\symbol{62}}\hlstd{} 
\mbox{}
\normalfont
\normalsize


As mentioned, insert is a little bit more rich in that it has a few extra attributes and sub-elements that allow it to deal with key generation in a number of ways.

First, if your database supports auto-generated key fields (e.g. MySQL and SQL Server), then you can simply set useGeneratedKeys="true" and set the keyProperty to the target property and you're done. For example, if the Author table above had used an auto-generated column type for the id, the statement would be modified as follows:

\noindent
\ttfamily
\hlstd{}\hllin{1\ }\hlopt{\symbol{60}}\hlstd{insert\ id}\hlopt{=}\hlstd{}\hlstr{"insertAuthor"}\hlstd{\ parameterType}\hlopt{=}\hlstd{}\hlstr{"domain.blog.Author"}\hlstd{\ \Righttorque\\
\hllin{2\ }useGeneratedKeys}\hlopt{=}\hlstd{}\hlstr{"true"}\hlstd{\\
\hllin{3\ }}\hlstd{\ \ \ \ }\hlstd{keyProperty}\hlopt{=}\hlstd{}\hlstr{"id"}\hlstd{}\hlopt{\symbol{62}}\\
\hllin{4\ }\hlstd{}\hlstd{\ \ }\hlstd{insert\ into\ }\hlkwd{Author\ }\hlstd{}\hlopt{(}\hlstd{username}\hlopt{,}\hlstd{password}\hlopt{,}\hlstd{email}\hlopt{,}\hlstd{bio}\hlopt{)}\\
\hllin{5\ }\hlstd{}\hlstd{\ \ }\hlstd{}\hlkwd{values\ }\hlstd{}\hlopt{(}\hlstd{\#}\hlopt{\symbol{123}}\hlstd{username}\hlopt{\symbol{125},}\hlstd{\#}\hlopt{\symbol{123}}\hlstd{password}\hlopt{\symbol{125},}\hlstd{\#}\hlopt{\symbol{123}}\hlstd{email}\hlopt{\symbol{125},}\hlstd{\#}\hlopt{\symbol{123}}\hlstd{bio}\hlopt{\symbol{125})}\\
\hllin{6\ }\hlstd{}\hlopt{\symbol{60}/}\hlstd{insert}\hlopt{\symbol{62}}\hlstd{} 
\mbox{}
\normalfont
\normalsize


MyBatis has another way to deal with key generation for databases that don't support auto-generated column types, or perhaps don't yet support the JDBC driver support for auto-generated keys.

Here's a simple (silly) example that would generate a random ID (something you'd likely never do, but this demonstrates the flexibility and how MyBatis really doesn't mind):

\noindent
\ttfamily
\hlstd{}\hllin{1\ }\hlopt{\symbol{60}}\hlstd{insert\ id}\hlopt{=}\hlstd{}\hlstr{"insertAuthor"}\hlstd{\ parameterType}\hlopt{=}\hlstd{}\hlstr{"domain.blog.Author"}\hlstd{}\hlopt{\symbol{62}}\\
\hllin{2\ }\hlstd{}\hlstd{\ \ }\hlstd{}\hlopt{\symbol{60}}\hlstd{selectKey\ keyProperty}\hlopt{=}\hlstd{}\hlstr{"id"}\hlstd{\ resultType}\hlopt{=}\hlstd{}\hlstr{"int"}\hlstd{\ \Righttorque\\
\hllin{3\ }}\hlstd{\ \ }\hlstd{order}\hlopt{=}\hlstd{}\hlstr{"BEFORE"}\hlstd{}\hlopt{\symbol{62}}\\
\hllin{4\ }\hlstd{}\hlstd{\ \ \ \ }\hlstd{select\ }\hlkwd{CAST}\hlstd{}\hlopt{(}\hlstd{}\hlkwd{RANDOM}\hlstd{}\hlopt{(){*}}\hlstd{}\hlnum{1000000\ }\hlstd{}\hlkwa{as\ }\hlstd{INTEGER}\hlopt{)\ }\hlstd{a\ from\ SYSIBM}\hlopt{.}\Righttorque\\
\hllin{5\ }\hlstd{}\hlstd{\ \ \ \ }\hlstd{SYSDUMMY1\\
\hllin{6\ }}\hlstd{\ \ }\hlstd{}\hlopt{\symbol{60}/}\hlstd{selectKey}\hlopt{\symbol{62}}\\
\hllin{7\ }\hlstd{}\hlstd{\ \ }\hlstd{insert\ into\ Author\\
\hllin{8\ }}\hlstd{\ \ \ \ }\hlstd{}\hlopt{(}\hlstd{id}\hlopt{,\ }\hlstd{username}\hlopt{,\ }\hlstd{password}\hlopt{,\ }\hlstd{email}\hlopt{,}\hlstd{bio}\hlopt{,\ }\hlstd{favourite\symbol{95}section}\hlopt{)}\\
\hllin{9\ }\hlstd{}\hlstd{\ \ }\hlstd{values\\
\hllin{10\ }}\hlstd{\ \ \ \ }\hlstd{}\hlopt{(}\hlstd{\#}\hlopt{\symbol{123}}\hlstd{id}\hlopt{\symbol{125},\ }\hlstd{\#}\hlopt{\symbol{123}}\hlstd{username}\hlopt{\symbol{125},\ }\hlstd{\#}\hlopt{\symbol{123}}\hlstd{password}\hlopt{\symbol{125},\ }\hlstd{\#}\hlopt{\symbol{123}}\hlstd{email}\hlopt{\symbol{125},\ }\hlstd{\#}\hlopt{\symbol{123}}\hlstd{bio}\hlopt{\symbol{125},\ }\hlstd{\#}\hlopt{\symbol{123}}\Righttorque\\
\hllin{11\ }\hlstd{}\hlstd{\ \ \ \ }\hlstd{favouriteSection}\hlopt{,}\hlstd{jdbcType}\hlopt{=}\hlstd{VARCHAR}\hlopt{\symbol{125})}\\
\hllin{12\ }\hlstd{}\hlopt{\symbol{60}/}\hlstd{insert}\hlopt{\symbol{62}}\hlstd{} 
\mbox{}
\normalfont
\normalsize


In the example above, the selectKey statement would be run first, the Author id property would be set, and then the insert statement would be called. This gives you a similar behavior to an auto-generated key in your database without complicating your Java code.

The selectKey element is described as follows:

\noindent
\ttfamily
\hlstd{}\hllin{1\ }\hlopt{\symbol{60}}\hlstd{selectKey\\
\hllin{2\ }}\hlstd{\ \ }\hlstd{keyProperty}\hlopt{=}\hlstd{}\hlstr{"id"}\hlstd{\\
\hllin{3\ }}\hlstd{\ \ }\hlstd{resultType}\hlopt{=}\hlstd{}\hlstr{"int"}\hlstd{\\
\hllin{4\ }}\hlstd{\ \ }\hlstd{order}\hlopt{=}\hlstd{}\hlstr{"BEFORE"}\hlstd{\\
\hllin{5\ }}\hlstd{\ \ }\hlstd{statementType}\hlopt{=}\hlstd{}\hlstr{"PREPARED"}\hlstd{}\hlopt{\symbol{62}}\hlstd{} 
\mbox{}
\normalfont
\normalsize


\subsection{Sql}
This element can be used to define a reusable fragment of SQL code that can be included in other statements. For example:

\noindent
\ttfamily
\hlstd{}\hllin{1\ }\hlopt{\symbol{60}}\hlstd{sql\ id}\hlopt{=}\hlstd{}\hlstr{"userColumns"}\hlstd{}\hlopt{\symbol{62}\ }\hlstd{id}\hlopt{,}\hlstd{username}\hlopt{,}\hlstd{password\ }\hlopt{\symbol{60}/}\hlstd{sql}\hlopt{\symbol{62}}\hlstd{} 
\mbox{}
\normalfont
\normalsize


The SQL fragment can then be included in another statement, for example:

\noindent
\ttfamily
\hlstd{}\hllin{1\ }\hlopt{\symbol{60}}\hlstd{select\ id}\hlopt{=}\hlstd{}\hlstr{"selectUsers"}\hlstd{\ parameterType}\hlopt{=}\hlstd{}\hlstr{"int"}\hlstd{\ \Righttorque\\
\hllin{2\ }resultType}\hlopt{=}\hlstd{}\hlstr{"hashmap"}\hlstd{}\hlopt{\symbol{62}}\\
\hllin{3\ }\hlstd{}\hlstd{\ \ }\hlstd{select\ }\hlopt{\symbol{60}}\hlstd{include\ refid}\hlopt{=}\hlstd{}\hlstr{"userColumns"}\hlstd{}\hlopt{/\symbol{62}}\\
\hllin{4\ }\hlstd{}\hlstd{\ \ }\hlstd{from\ some\symbol{95}table\\
\hllin{5\ }}\hlstd{\ \ }\hlstd{where\ id\ }\hlopt{=\ }\hlstd{\#}\hlopt{\symbol{123}}\hlstd{id}\hlopt{\symbol{125}}\\
\hllin{6\ }\hlstd{}\hlopt{\symbol{60}/}\hlstd{select}\hlopt{\symbol{62}}\hlstd{} 
\mbox{}
\normalfont
\normalsize

\subsection{Parameters}
In all of the past statements, you've seen examples of simple parameters. Parameters are very powerful elements in MyBatis. For simple situations, probably 90\% of the cases, there's not much too them, for example:

\input{Coding/Code21.java.tex}

The example above demonstrates a very simple named parameter mapping. The parameterType is set to int, so therefore the parameter could be named anything. Primitive or simply data types such as Integer and String have no relevant properties, and thus will replace the full value of the parameter entirely. However, if you pass in a complex object, then the behavior is a little different.For example:

\noindent
\ttfamily
\hlstd{}\hllin{1\ }\hlopt{\symbol{60}}\hlstd{select\ id}\hlopt{=}\hlstd{}\hlstr{"selectUsers"}\hlstd{\ parameterType}\hlopt{=}\hlstd{}\hlstr{"int"}\hlstd{\ \Righttorque\\
\hllin{2\ }resultType}\hlopt{=}\hlstd{}\hlstr{"User"}\hlstd{}\hlopt{\symbol{62}}\\
\hllin{3\ }\hlstd{}\hlstd{\ \ }\hlstd{select\ id}\hlopt{,\ }\hlstd{username}\hlopt{,\ }\hlstd{password\\
\hllin{4\ }}\hlstd{\ \ }\hlstd{from\ users\\
\hllin{5\ }}\hlstd{\ \ }\hlstd{where\ id\ }\hlopt{=\ }\hlstd{\#}\hlopt{\symbol{123}}\hlstd{id}\hlopt{\symbol{125}}\\
\hllin{6\ }\hlstd{}\hlopt{\symbol{60}/}\hlstd{select}\hlopt{\symbol{62}}\hlstd{} 
\mbox{}
\normalfont
\normalsize


If a parameter object of type User was passed into that statement, the id, username and password property would be looked up and their values passed to a PreparedStatement parameter.