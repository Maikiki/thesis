% !Mode:: "TeX:UTF-8"

\chapter{系统详细设计与实现}

\section{数据搜集模块}
数据搜集模块用于访问FTP服务器并获取FTP服务器上的所有文件信息。并将文件信息以文本的形式存储在本地的磁盘上,以供其他需要文件信息文本的模块使用。整个模块被封装在FTPSpider类中。FTPSpider类的结构如下:

\noindent
\ttfamily
\hlstd{}\hllin{1\ }\hlkwa{public\ class\ }\hlstd{FTPSpider\ }\hlopt{\symbol{123}}\\
\hllin{2\ }\hlstd{}\hlstd{\ \ \ \ }\hlstd{}\hlkwa{private\ }\hlstd{FTPClient}\hlstd{\ \ \ \ \ }\hlstd{ftp}\hlopt{;}\hlstd{\ \ \ \ \ \ \ \ \ \ \ \ }\hlopt{}\hlstd{}\hlslc{//\ FTP客户端类}\\
\hllin{3\ }\hlstd{}\hlstd{\ \ \ \ }\hlstd{}\hlkwa{private\ }\hlstd{String}\hlstd{\ \ \ \ \ \ \ \ }\hlstd{server}\hlopt{;}\hlstd{\ \ \ \ \ \ \ \ \ }\hlopt{}\hlstd{}\hlslc{//\ }\Righttorque\\
\hllin{4\ }\hlslc{}\hlstd{\ \ \ \ }\hlslc{连接的FTP服务器}\\
\hllin{5\ }\hlstd{}\hlstd{\ \ \ \ }\hlstd{}\hlkwa{private\ }\hlstd{}\hlkwb{int}\hlstd{\ \ \ \ \ \ \ \ \ \ \ }\hlkwb{}\hlstd{port}\hlopt{;}\hlstd{\ \ \ \ \ \ \ \ \ \ \ }\hlopt{}\hlstd{}\hlslc{//\ }\Righttorque\\
\hllin{6\ }\hlslc{}\hlstd{\ \ \ \ }\hlslc{FTP服务器开放的端口}\\
\hllin{7\ }\hlstd{}\hlstd{\ \ \ \ }\hlstd{}\hlkwa{private\ }\hlstd{String}\hlstd{\ \ \ \ \ \ \ \ }\hlstd{controlEncoding}\hlopt{;\ }\hlstd{}\hlslc{//\ 控制编码}\\
\hllin{8\ }\hlstd{}\hlstd{\ \ \ \ }\hlstd{}\hlkwa{private\ }\hlstd{String}\hlstd{\ \ \ \ \ \ \ \ }\hlstd{charEncoding}\hlopt{;}\hlstd{\ \ \ }\hlopt{}\hlstd{}\hlslc{//\ 字符集编码}\\
\hllin{9\ }\hlstd{\\
\hllin{10\ }}\hlstd{\ \ \ \ }\hlstd{}\hlkwa{private\ }\hlstd{String}\hlstd{\ \ \ \ \ \ \ \ }\hlstd{outputFile}\hlopt{;}\hlstd{\ \ \ \ \ }\hlopt{}\hlstd{}\hlslc{//\ }\Righttorque\\
\hllin{11\ }\hlslc{}\hlstd{\ \ \ \ }\hlslc{输出文件的位置}\\
\hllin{12\ }\hlstd{}\hlstd{\ \ \ \ }\hlstd{}\hlkwa{private\ }\hlstd{Queue}\hlopt{\symbol{60}}\hlstd{String}\hlopt{\symbol{62}\ }\hlstd{pathQueue}\hlopt{;}\hlstd{\ \ \ \ \ \ }\hlopt{}\hlstd{}\hlslc{//\ }\Righttorque\\
\hllin{13\ }\hlslc{}\hlstd{\ \ \ \ }\hlslc{路径队列,遍历FTP服务器时使用}\\
\hllin{14\ }\hlstd{}\hlstd{\ \ \ \ }\hlstd{}\hlkwa{private\ }\hlstd{}\hlkwb{int}\hlstd{\ \ \ \ \ \ \ \ \ \ \ }\hlkwb{}\hlstd{count\ }\hlopt{=\ }\hlstd{}\hlnum{0}\hlstd{}\hlopt{;}\hlstd{\ \ \ \ \ \ }\hlopt{}\hlstd{}\hlslc{//\ }\Righttorque\\
\hllin{15\ }\hlslc{}\hlstd{\ \ \ \ }\hlslc{记录遍历的文件数量}\\
\hllin{16\ }\hlstd{\\
\hllin{17\ }}\hlstd{\ \ \ \ }\hlstd{}\hlkwa{public\ }\hlstd{}\hlkwb{int\ }\hlstd{}\hlkwd{getCount}\hlstd{}\hlopt{()}\hlstd{\ \ }\hlopt{}\hlstd{}\hlslc{//遍历文件数量}\\
\hllin{18\ }\hlstd{}\hlstd{\ \ \ \ }\hlstd{}\hlkwa{public\ }\hlstd{}\hlkwd{FTPSpider}\hlstd{}\hlopt{(}\hlstd{String\ server}\hlopt{,\ }\hlstd{}\hlkwb{int\ }\hlstd{port}\hlopt{,\ }\hlstd{String\ \Righttorque\\
\hllin{19\ }}\hlstd{\ \ \ \ }\hlstd{charEncoding}\hlopt{,}\hlstd{String\ outputFile}\hlopt{)}\hlstd{\ \ }\hlopt{}\hlstd{}\hlslc{//构造方法}\\
\hllin{20\ }\hlstd{}\hlstd{\ \ \ \ }\hlstd{}\hlkwa{public\ }\hlstd{}\hlkwd{FTPSpider}\hlstd{}\hlopt{(}\hlstd{String\ server}\hlopt{,\ }\hlstd{String\ outputFile}\hlopt{)}\hlstd{\ \ }\hlopt{}\Righttorque\\
\hllin{21\ }\hlstd{}\hlstd{\ \ \ \ }\hlstd{}\hlslc{//构造方法}\\
\hllin{22\ }\hlstd{}\hlstd{\ \ \ \ }\hlstd{}\hlkwa{public\ }\hlstd{}\hlkwb{void\ }\hlstd{}\hlkwd{connectFTP}\hlstd{}\hlopt{()}\hlstd{\ \ }\hlopt{}\hlstd{}\hlslc{//连接FTP服务器}\\
\hllin{23\ }\hlstd{}\hlstd{\ \ \ \ }\hlstd{}\hlkwa{public\ }\hlstd{}\hlkwb{void\ }\hlstd{}\hlkwd{getFTPInfo}\hlstd{}\hlopt{(}\hlstd{String\ remotePath}\hlopt{)}\hlstd{\ \ }\hlopt{}\Righttorque\\
\hllin{24\ }\hlstd{}\hlstd{\ \ \ \ }\hlstd{}\hlslc{//扫描指定目录}\\
\hllin{25\ }\hlstd{}\hlstd{\ \ \ \ }\hlstd{}\hlkwa{private\ }\hlstd{}\hlkwb{void\ }\hlstd{}\hlkwd{closeTheConnect}\hlstd{}\hlopt{()}\hlstd{\ \ }\hlopt{}\hlstd{}\hlslc{//退出FTP服务器}\\
\hllin{26\ }\hlstd{}\hlopt{\symbol{125}}\\
\hllin{27\ }\hlstd{} 
\mbox{}
\normalfont
\normalsize


\subsection{数据存储格式}
对于从FTP搜集下来的目录信息及文件信息,采用如表\ref{table1}所示规则存储:

\pictable[htbp]{数据存储格式表}{width=0.8\textwidth}{table1}

目录行:以“/”开头,第一个“/”代表FTP服务器根目录。
文件信息行:以数字开始,格式为“文件修改时间/文件类型/文件大小/文件名”。文件修改日期是以文件修改时间减去1970年1月1日00:00的时差,精确到秒。对于一般的文件来说,其类型为其文件名的后缀名,即最后一个“.”后的字符串。例如文件名“新建文本文档.txt”,它的类型为txt。但是最后一个“.”的位置不能是文件名的第一个字符,如“.classpath”这个文件名,它的文件类型并不是classpath。文件大小的单位为Byte。在本系统中,文件夹也被看作一种文件,它是类型为folder,大小为0。
图\ref{image021}给出了一个FTP信息文本的实例的一部分,它包括了上面所说的全部定义:
\newpage
\pic[ht]{数据存储实例}{width=0.6\textwidth}{image021}

\subsection{相关类及方法}
在本系统中,使用FTPClient类的实例ftp来完成所有与FTP服务器的交互工作,这些工作包括:连接并登陆匿名的FTP服务器、改变当前FTP会话的工作目录、获取工作目录下的文件信息。
这些工作的主要实现方法如下,他们都由FTPClient类来提供:

1. 连接匿名的FTP服务器:connect(server, port),该方法需要提供服务器的域名或者IP作为server参数,提供所开放的FTP端口号作为port参数。

2. 登陆匿名的FTP服务器:login(String username, String password),需提供用户名与密码来完成FTP服务器的登陆工作,由于本系统只对匿名的FTP服务器检索,所以在这里,username为“anonymous”,password为笔者的邮箱“269504518\\@qq.com”。

3. 改变当前FTP会话的工作目录:changeWorkingDirectory(String pathname),每当获取并完成对一个工作目录下的文件信息的存储后,需要使用该方法来改变当前工作目录,从而能够获取其他工作目录的信息,以完成对FTP指定目录遍历的目的。

4. 退出FTP服务器:disconnect(),当搜集模块完成了遍历任务后,用它来释放FTP连接。
\subsection{处理流程及实现}
整个搜集的过程如图\ref{image022}所示。pathQueue是一个String队列,里面存储着提供给changeWorkingDirectory(String pathname)使用的pathname,对于一个文件夹类型的文件,它自身所在的目录加上它的文件名就构成了新的pathname。

\pic[!htbp]{FTP数据搜集流程图}{width=0.6\textwidth}{image022}

\section{文件信息对比模块}
对于一个在FTP服务器的文件,文件的文件大小或修改日期发生变化时,称该文件被更改了。FTP服务器随时都有可能更改、删除某个文件,或者添加新的文件。文件信息对比模块将新搜集的文件信息文本与当前被Lucene索引的文件信息文本进行对比,找出FTP服务器文件信息的变化之处,为索引模块更新索引提供依据。文件信息对比模块所有的操作被封装在FileComparer类中,它与其他外部类的关系如图\ref{image023}所示。

\pic[htbp]{FileComparer类图}{width=0.8\textwidth}{image023}

\subsection{数据结构设计}
FileInfo类用于4.1.1数据存储格式中文件信息的实例化,每一条文件信息构建一个FileInfo实例。
Files类将工作目录和目录下的文件信息结合在一起,他包含path,以及在这个工作目录下的所有文件信息。这样就很好的与4.1.1数据存储格式对应了起来。
由于ArrayList在遍历时,其中的元素不能删除,故用Delete\-Index记录每个工作路径下需要被删除的文件信息。最后统一删除。
\subsection{处理流程及实现}
对比流程如图\ref{image026}所示。新搜集的FTP文件信息文本被读取并构建为Files实例,存入newFiles数组列表;之前被Lucene 索引的FTP文件信息文本被读取并构建为\\Files实例,存入oldFile数组列表,整个处理流程通过对比找出并删除两个数组中相同的FileInfo实例,即删除FTP中文件信息没有变化的文件。在完成整个处理流程后,newFiles数组列表中剩下的元素就是FTP服务器上新添加的文件。oldFiles中剩下的元素就是FTP服务器上被删除的文件。
\newpage
\pic[htbp]{文件信息对比模块流程图}{width=0.6\textwidth}{image026}

关键代码如下:

\noindent
\ttfamily
\hlstd{}\hllin{1\ }\hlkwa{for\ }\hlstd{}\hlopt{(}\hlstd{Files\ newFile\ }\hlopt{:\ }\hlstd{newFiles}\hlopt{)\ \symbol{123}}\\
\hllin{2\ }\hlstd{}\hlstd{\ \ \ \ }\hlstd{}\hlkwa{for\ }\hlstd{}\hlopt{(}\hlstd{Files\ oldFile\ }\hlopt{:\ }\hlstd{oldFiles}\hlopt{)\ \symbol{123}}\\
\hllin{3\ }\hlstd{}\hlstd{\ \ \ \ \ \ \ \ }\hlstd{}\hlkwa{if\ }\hlstd{}\hlopt{(}\hlstd{newFile}\hlopt{.}\hlstd{}\hlkwd{getPath}\hlstd{}\hlopt{().}\hlstd{}\hlkwd{equals}\hlstd{}\hlopt{(}\hlstd{oldFile}\hlopt{.}\hlstd{}\hlkwd{getPath}\hlstd{}\hlopt{()))\ \symbol{123}}\\
\hllin{4\ }\hlstd{}\hlstd{\ \ \ \ \ \ \ \ \ \ \ \ }\hlstd{modified\ }\hlopt{=\ }\hlstd{}\hlkwa{new\ }\hlstd{}\hlkwd{Files}\hlstd{}\hlopt{(}\hlstd{newFile}\hlopt{.}\hlstd{}\hlkwd{getPath}\hlstd{}\hlopt{());}\\
\hllin{5\ }\hlstd{}\hlstd{\ \ \ \ \ \ \ \ \ \ \ \ }\hlstd{indexOfNew\ }\hlopt{=\ }\hlstd{}\hlkwa{new\ }\hlstd{}\hlkwd{DeleteIndex}\hlstd{}\hlopt{();}\\
\hllin{6\ }\hlstd{}\hlstd{\ \ \ \ \ \ \ \ \ \ \ \ }\hlstd{indexOfOld\ }\hlopt{=\ }\hlstd{}\hlkwa{new\ }\hlstd{}\hlkwd{DeleteIndex}\hlstd{}\hlopt{();}\\
\hllin{7\ }\hlstd{}\hlstd{\ \ \ \ \ \ \ \ \ \ \ \ }\hlstd{indexOfNew}\hlopt{.}\hlstd{}\hlkwd{setFilesIndex}\hlstd{}\hlopt{(}\hlstd{newFiles}\hlopt{.}\hlstd{}\hlkwd{indexOf}\hlstd{}\hlopt{(}\Righttorque\\
\hllin{8\ }\hlstd{}\hlstd{\ \ \ \ \ \ \ \ \ \ \ \ }\hlstd{newFile}\hlopt{));}\\
\hllin{9\ }\hlstd{}\hlstd{\ \ \ \ \ \ \ \ \ \ \ \ }\hlstd{indexOfOld}\hlopt{.}\hlstd{}\hlkwd{setFilesIndex}\hlstd{}\hlopt{(}\hlstd{oldFiles}\hlopt{.}\hlstd{}\hlkwd{indexOf}\hlstd{}\hlopt{(}\Righttorque\\
\hllin{10\ }\hlstd{}\hlstd{\ \ \ \ \ \ \ \ \ \ \ \ }\hlstd{oldFile}\hlopt{));}\\
\hllin{11\ }\hlstd{\\
\hllin{12\ }}\hlstd{\ \ \ \ \ \ \ \ \ \ \ \ }\hlstd{newFile}\hlopt{.}\hlstd{}\hlkwd{compare}\hlstd{}\hlopt{(}\hlstd{oldFile}\hlopt{.}\hlstd{}\hlkwd{getFileInfos}\hlstd{}\hlopt{(),}\\
\hllin{13\ }\hlstd{}\hlstd{\ \ \ \ \ \ \ \ \ \ \ \ \ \ \ \ \ \ \ \ \ \ \ \ \ \ \ \ }\hlstd{indexOfNew}\hlopt{.}\hlstd{}\hlkwd{getFileInfoIndex}\hlstd{}\hlopt{(),}\\
\hllin{14\ }\hlstd{}\hlstd{\ \ \ \ \ \ \ \ \ \ \ \ \ \ \ \ \ \ \ \ \ \ \ \ \ \ \ \ }\hlstd{indexOfOld}\hlopt{.}\hlstd{}\hlkwd{getFileInfoIndex}\hlstd{}\hlopt{(),}\\
\hllin{15\ }\hlstd{}\hlstd{\ \ \ \ \ \ \ \ \ \ \ \ \ \ \ \ \ \ \ \ \ \ \ \ \ \ \ \ }\hlstd{modified}\hlopt{.}\hlstd{}\hlkwd{getFileInfos}\hlstd{}\hlopt{());}\\
\hllin{16\ }\hlstd{\\
\hllin{17\ }}\hlstd{\ \ \ \ \ \ \ \ \ \ \ \ }\hlstd{modifiedFiles}\hlopt{.}\hlstd{}\hlkwd{add}\hlstd{}\hlopt{(}\hlstd{modified}\hlopt{);}\\
\hllin{18\ }\hlstd{}\hlstd{\ \ \ \ \ \ \ \ \ \ \ \ }\hlstd{deleteInNew}\hlopt{.}\hlstd{}\hlkwd{add}\hlstd{}\hlopt{(}\hlstd{indexOfNew}\hlopt{);}\\
\hllin{19\ }\hlstd{}\hlstd{\ \ \ \ \ \ \ \ \ \ \ \ }\hlstd{deleteInOld}\hlopt{.}\hlstd{}\hlkwd{add}\hlstd{}\hlopt{(}\hlstd{indexOfOld}\hlopt{);}\\
\hllin{20\ }\hlstd{\\
\hllin{21\ }}\hlstd{\ \ \ \ \ \ \ \ \ \ \ \ }\hlstd{}\hlkwa{break}\hlstd{}\hlopt{;}\\
\hllin{22\ }\hlstd{}\hlstd{\ \ \ \ \ \ \ \ }\hlstd{}\hlopt{\symbol{125}}\\
\hllin{23\ }\hlstd{}\hlstd{\ \ \ \ }\hlstd{}\hlopt{\symbol{125}}\\
\hllin{24\ }\hlstd{}\hlopt{\symbol{125}}\\
\hllin{25\ }\hlstd{} 
\mbox{}
\normalfont
\normalsize


图\ref{image028}为图\ref{image026}中“对比目录下的文件信息”过程的细化。这个过程被封装在Files类的compare()方法中。

\pic[htbp]{文件信息对比流程图}{width=0.8\textwidth}{image028}

关键代码如下:

\noindent
\ttfamily
\hlstd{}\hllin{1\ }\hlkwa{public\ }\hlstd{}\hlkwb{void\ }\hlstd{}\hlkwd{compare}\hlstd{}\hlopt{(}\hlstd{ArrayList}\hlopt{\symbol{60}}\hlstd{FileInfo}\hlopt{\symbol{62}\ }\hlstd{oldFileInfos}\hlopt{,}\\
\hllin{2\ }\hlstd{}\hlstd{\ \ \ \ \ \ \ \ \ \ \ \ \ \ \ \ \ \ \ \ }\hlstd{ArrayList}\hlopt{\symbol{60}}\hlstd{FileInfo}\hlopt{\symbol{62}\ }\hlstd{indexOfNew}\hlopt{,\ }\Righttorque\\
\hllin{3\ }\hlstd{}\hlstd{\ \ \ \ \ \ \ \ \ \ \ \ \ \ \ \ \ \ \ \ }\hlstd{ArrayList}\hlopt{\symbol{60}}\hlstd{FileInfo}\hlopt{\symbol{62}\ }\hlstd{indexOfOld}\hlopt{,}\\
\hllin{4\ }\hlstd{}\hlstd{\ \ \ \ \ \ \ \ \ \ \ \ \ \ \ \ \ \ \ \ }\hlstd{ArrayList}\hlopt{\symbol{60}}\hlstd{FileInfo}\hlopt{\symbol{62}\ }\hlstd{modified}\hlopt{)\ \symbol{123}}\\
\hllin{5\ }\hlstd{}\hlstd{\ \ \ \ }\hlstd{}\hlkwa{for\ }\hlstd{}\hlopt{(}\hlstd{FileInfo\ newOne\ }\hlopt{:\ }\hlstd{}\hlkwa{this}\hlstd{}\hlopt{.}\hlstd{fileInfos}\hlopt{)\ \symbol{123}}\\
\hllin{6\ }\hlstd{}\hlstd{\ \ \ \ \ \ \ \ }\hlstd{}\hlkwa{for\ }\hlstd{}\hlopt{(}\hlstd{FileInfo\ oldOne\ }\hlopt{:\ }\hlstd{oldFileInfos}\hlopt{)\ \symbol{123}}\\
\hllin{7\ }\hlstd{}\hlstd{\ \ \ \ \ \ \ \ \ \ \ \ }\hlstd{}\hlkwa{if\ }\hlstd{}\hlopt{(}\hlstd{newOne}\hlopt{.}\hlstd{}\hlkwd{getName}\hlstd{}\hlopt{().}\hlstd{}\hlkwd{equals}\hlstd{}\hlopt{(}\hlstd{oldOne}\hlopt{.}\hlstd{}\hlkwd{getName}\hlstd{}\hlopt{())}\\
\hllin{8\ }\hlstd{}\hlstd{\ \ \ \ \ \ \ \ \ \ \ \ \ \ \ \ \ \ \ \ }\hlstd{}\hlopt{\&\&\ }\hlstd{newOne}\hlopt{.}\hlstd{}\hlkwd{getType}\hlstd{}\hlopt{().}\hlstd{}\hlkwd{equals}\hlstd{}\hlopt{(}\hlstd{oldOne}\hlopt{.}\Righttorque\\
\hllin{9\ }\hlstd{}\hlstd{\ \ \ \ \ \ \ \ \ \ \ \ \ \ \ \ \ \ \ \ }\hlstd{}\hlkwd{getType}\hlstd{}\hlopt{()))\ \symbol{123}}\\
\hllin{10\ }\hlstd{}\hlstd{\ \ \ \ \ \ \ \ \ \ \ \ \ \ \ \ }\hlstd{}\hlkwa{if\ }\hlstd{}\hlopt{(}\hlstd{newOne}\hlopt{.}\hlstd{}\hlkwd{getSize}\hlstd{}\hlopt{()\ ==\ }\hlstd{oldOne}\hlopt{.}\hlstd{}\hlkwd{getSize}\hlstd{}\hlopt{()}\\
\hllin{11\ }\hlstd{}\hlstd{\ \ \ \ \ \ \ \ \ \ \ \ \ \ \ \ \ \ \ \ \ \ \ \ }\hlstd{}\hlopt{\&\&\ }\hlstd{newOne}\hlopt{.}\hlstd{}\hlkwd{getDate}\hlstd{}\hlopt{()\ ==\ }\hlstd{oldOne}\hlopt{.}\Righttorque\\
\hllin{12\ }\hlstd{}\hlstd{\ \ \ \ \ \ \ \ \ \ \ \ \ \ \ \ \ \ \ \ \ \ \ \ }\hlstd{}\hlkwd{getDate}\hlstd{}\hlopt{())\ \symbol{123}}\\
\hllin{13\ }\hlstd{}\hlstd{\ \ \ \ \ \ \ \ \ \ \ \ \ \ \ \ \ \ \ \ }\hlstd{indexOfNew}\hlopt{.}\hlstd{}\hlkwd{add}\hlstd{}\hlopt{(}\hlstd{newOne}\hlopt{);}\\
\hllin{14\ }\hlstd{}\hlstd{\ \ \ \ \ \ \ \ \ \ \ \ \ \ \ \ \ \ \ \ }\hlstd{indexOfOld}\hlopt{.}\hlstd{}\hlkwd{add}\hlstd{}\hlopt{(}\hlstd{oldOne}\hlopt{);}\\
\hllin{15\ }\hlstd{}\hlstd{\ \ \ \ \ \ \ \ \ \ \ \ \ \ \ \ \ \ \ \ }\hlstd{}\hlkwa{break}\hlstd{}\hlopt{;}\\
\hllin{16\ }\hlstd{}\hlstd{\ \ \ \ \ \ \ \ \ \ \ \ \ \ \ \ }\hlstd{}\hlopt{\symbol{125}\ }\hlstd{}\hlkwa{else\ }\hlstd{}\hlopt{\symbol{123}}\\
\hllin{17\ }\hlstd{}\hlstd{\ \ \ \ \ \ \ \ \ \ \ \ \ \ \ \ \ \ \ \ }\hlstd{modified}\hlopt{.}\hlstd{}\hlkwd{add}\hlstd{}\hlopt{(}\hlstd{newOne}\hlopt{);}\\
\hllin{18\ }\hlstd{}\hlstd{\ \ \ \ \ \ \ \ \ \ \ \ \ \ \ \ \ \ \ \ }\hlstd{}\hlkwa{break}\hlstd{}\hlopt{;}\\
\hllin{19\ }\hlstd{}\hlstd{\ \ \ \ \ \ \ \ \ \ \ \ \ \ \ \ }\hlstd{}\hlopt{\symbol{125}}\\
\hllin{20\ }\hlstd{}\hlstd{\ \ \ \ \ \ \ \ \ \ \ \ }\hlstd{}\hlopt{\symbol{125}}\\
\hllin{21\ }\hlstd{}\hlstd{\ \ \ \ \ \ \ \ }\hlstd{}\hlopt{\symbol{125}}\\
\hllin{22\ }\hlstd{}\hlstd{\ \ \ \ }\hlstd{}\hlopt{\symbol{125}}\\
\hllin{23\ }\hlstd{}\hlopt{\symbol{125}}\\
\hllin{24\ }\hlstd{} 
\mbox{}
\normalfont
\normalsize


\subsection{结果存储格式}
对比结果文本分为4部分,如表\ref{table2}所示,目录行及文件信息行的定义与4.1.1中一致。
文件头部:指明了文件信息所属的服务器及该服务器所开放的端口。
被删除的条目:FTP服务器上被删除的文件,被索引模块用于删除Lucene索引中过时的条目。
新添加的条目:FTP服务器上新添加的文件,被索引模块用于向lucene索引添加新的条目。
发生变化的条目:可用于统计FTP每个目录发生变化的次数。
\newpage
\pictable[htbp]{文件信息对比结果存储格式表}{width=0.8\textwidth}{table2}

图\ref{image029}给出了一个FTP信息文本的实例的一部分,它包括了上面所说的全部定义。

\pic[!htbp]{文件信息对比结果实例}{width=0.4\textwidth}{image029}

\section{数据索引模块}
数据索引模块将执行Lucene的索引过程,对搜集到的FTP文件信息文本进行索引。数据只有经过索引之后才能够被用于搜索。
\subsection{相关类及方法}
IndexWriter类:IndexWriter类是org.apache.lucene.index包中的类,他主要负责创建并维护一个lucene索引。在数据索引模块中,使用indexWriter实例的\\add\-Docu\-ment\-(Document doc)方法来将相关数据构建成的Document实例添加至Lucene索引中。
\subsection{构建IndexWriter}
构建indexWriter实例需要经过以下的过程:

1.指定lucene索引的存放位置。

需要实例化一个Directory对象来完成这一工作:Directory dir = FSDirectory.\\open(new File(configer.getIndexDir()))其中,路径由从configer实例的getIndexDir()方法给出,在4.7节将给出configer的具体设计。

2. 构建IndexWriterConfig实例。

IndexWriterConfig的实例用于修改IndexWriter的一切配置,以使得新建的Index\-Writer能够符合需求。
例如:IndexWriterConfig iwconf = new IndexWriter\-Config\-(Version.\-LUCENE\_35, new Smart\-Chinese\-Analyzer(Version.LUCENE\_35));在构建Index\-Writer\-Config实例,需要确认使用的Lucene的版本号,以及Index\-Writer默认的分析器是什么。在例子中,使用的Lucene版本为3.5,使用的是SmartChineseAnalyzer分析器。

3. 构建IndexWriter实例。

在完成1、2两个步操作后,利用1、2所构建的对象实例来构建出indexWriter实例,例如:indexWriter = new IndexWriter(dir, iwconf)。
\subsection{构建Document}
在FTP搜索引擎中,Lucene的Document实例用于存储文件信息,一个document实例对应一个文件的文件信息。结构如表\ref{table3}所示。
\newpage
\pictable[ht]{FTP搜索引擎Lucene Document结构}{width=0.8\textwidth}{table3}

Document实例构建过程如下:

1. 使用构造方法,构建一个Document实例,例如:Document doc = new Document();

2. 使用Document实例的add(Fieldable field)方法向Document实例中添加实现了Fieldable接口的实例。
\subsection{Fieldable接口}
对于字符串类型的索引,使用Field(String name, String value, Field.Store store, Field.Index index)来构建。各参数的含义见表\ref{table4}。

\pictable[ht]{Field实例构造方法个参数意义}{width=0.8\textwidth}{table4}
\newpage
对于数字型的索引,使用NumericField(String name, Field.Store store, boolean index)来构建,各个参数的含义见表\ref{table5}。

\pictable[ht]{NumericField实例构造方法各参数意义}{width=0.8\textwidth}{table5}

当完成了NumericField实例的构造后,还需要调用其实例的一些方法,初始化实例。使用如表\ref{table6}所示的方法即可。

\pictable[htbp]{NumericField实例初始化数值的相关方法}{width=0.8\textwidth}{table6}

\subsection{数据索引流程}
整个数据索引流程如图\ref{image030}所示。索引模块先读取FTP文件信息文本中的FTP服务器信息及文件信息,利用文件信息建立Lucene文档并将其添加到Lucene索引。
\newpage
\pic[ht]{数据索引流程图}{width=0.6\textwidth}{image030}

将文件信息写入索引的关键代码如下:

\noindent
\ttfamily
\hlstd{}\hllin{1\ }\hlkwa{protected\ }\hlstd{}\hlkwb{void\ }\hlstd{}\hlkwd{addIntoIndex}\hlstd{}\hlopt{(}\hlstd{String\ name}\hlopt{,\ }\hlstd{String\ type}\hlopt{,\ }\hlstd{}\hlkwb{long\ }\Righttorque\\
\hllin{2\ }\hlstd{size}\hlopt{,\ }\hlstd{}\hlkwb{int\ }\hlstd{date}\hlopt{,}\\
\hllin{3\ }\hlstd{}\hlstd{\ \ \ \ \ \ \ \ \ \ \ \ \ \ \ \ \ \ \ \ \ \ \ \ \ \ \ \ }\hlstd{String\ path}\hlopt{)\ }\hlstd{}\hlkwa{throws\ }\hlstd{Exception\ }\hlopt{\symbol{123}}\\
\hllin{4\ }\hlstd{}\hlstd{\ \ \ \ }\hlstd{Document\ doc\ }\hlopt{=\ }\hlstd{}\hlkwa{new\ }\hlstd{}\hlkwd{Document}\hlstd{}\hlopt{();}\\
\hllin{5\ }\hlstd{}\hlstd{\ \ \ \ }\hlstd{}\hlslc{//初始化Lucene文档}\\
\hllin{6\ }\hlstd{}\hlstd{\ \ \ \ }\hlstd{doc}\hlopt{.}\hlstd{}\hlkwd{add}\hlstd{}\hlopt{(}\hlstd{}\hlkwa{new\ }\hlstd{}\hlkwd{Field}\hlstd{}\hlopt{(}\hlstd{}\hlstr{"name"}\hlstd{}\hlopt{,\ }\hlstd{name}\hlopt{,\ }\hlstd{Field}\hlopt{.}\hlstd{Store}\hlopt{.}\hlstd{YES}\hlopt{,\ }\hlstd{Field}\hlopt{.}\Righttorque\\
\hllin{7\ }\hlstd{}\hlstd{\ \ \ \ }\hlstd{Index}\hlopt{.}\hlstd{ANALYZED}\hlopt{));}\\
\hllin{8\ }\hlstd{}\hlstd{\ \ \ \ }\hlstd{doc}\hlopt{.}\hlstd{}\hlkwd{add}\hlstd{}\hlopt{(}\hlstd{}\hlkwa{new\ }\hlstd{}\hlkwd{Field}\hlstd{}\hlopt{(}\hlstd{}\hlstr{"type"}\hlstd{}\hlopt{,\ }\hlstd{type}\hlopt{,\ }\hlstd{Field}\hlopt{.}\hlstd{Store}\hlopt{.}\hlstd{YES}\hlopt{,\ }\hlstd{Field}\hlopt{.}\Righttorque\\
\hllin{9\ }\hlstd{}\hlstd{\ \ \ \ }\hlstd{Index}\hlopt{.}\hlstd{NOT\symbol{95}ANALYZED}\hlopt{));}\\
\hllin{10\ }\hlstd{}\hlstd{\ \ \ \ }\hlstd{doc}\hlopt{.}\hlstd{}\hlkwd{add}\hlstd{}\hlopt{(}\hlstd{}\hlkwa{new\ }\hlstd{}\hlkwd{NumericField}\hlstd{}\hlopt{(}\hlstd{}\hlstr{"size"}\hlstd{}\hlopt{,\ }\hlstd{Field}\hlopt{.}\hlstd{Store}\hlopt{.}\hlstd{YES}\hlopt{,\ }\hlstd{true}\hlopt{).}\Righttorque\\
\hllin{11\ }\hlstd{}\hlstd{\ \ \ \ }\hlstd{}\hlkwd{setLongValue}\hlstd{}\hlopt{(}\hlstd{size}\hlopt{));}\\
\hllin{12\ }\hlstd{}\hlstd{\ \ \ \ }\hlstd{doc}\hlopt{.}\hlstd{}\hlkwd{add}\hlstd{}\hlopt{(}\hlstd{}\hlkwa{new\ }\hlstd{}\hlkwd{NumericField}\hlstd{}\hlopt{(}\hlstd{}\hlstr{"date"}\hlstd{}\hlopt{,\ }\hlstd{Field}\hlopt{.}\hlstd{Store}\hlopt{.}\hlstd{YES}\hlopt{,\ }\hlstd{true}\hlopt{).}\Righttorque\\
\hllin{13\ }\hlstd{}\hlstd{\ \ \ \ }\hlstd{}\hlkwd{setIntValue}\hlstd{}\hlopt{(}\hlstd{date}\hlopt{));}\\
\hllin{14\ }\hlstd{}\hlstd{\ \ \ \ }\hlstd{doc}\hlopt{.}\hlstd{}\hlkwd{add}\hlstd{}\hlopt{(}\hlstd{}\hlkwa{new\ }\hlstd{}\hlkwd{Field}\hlstd{}\hlopt{(}\hlstd{}\hlstr{"path"}\hlstd{}\hlopt{,\ }\hlstd{path}\hlopt{,\ }\hlstd{Field}\hlopt{.}\hlstd{Store}\hlopt{.}\hlstd{YES}\hlopt{,\ }\hlstd{Field}\hlopt{.}\Righttorque\\
\hllin{15\ }\hlstd{}\hlstd{\ \ \ \ }\hlstd{Index}\hlopt{.}\hlstd{NOT\symbol{95}ANALYZED}\hlopt{));}\\
\hllin{16\ }\hlstd{}\hlstd{\ \ \ \ }\hlstd{}\hlslc{//将文档添加到Lucene索引}\\
\hllin{17\ }\hlstd{}\hlstd{\ \ \ \ }\hlstd{indexWriter}\hlopt{.}\hlstd{}\hlkwd{addDocument}\hlstd{}\hlopt{(}\hlstd{doc}\hlopt{);}\\
\hllin{18\ }\hlstd{}\hlopt{\symbol{125}}\\
\hllin{19\ }\hlstd{} 
\mbox{}
\normalfont
\normalsize


\subsection{索引中文档的更新}
对于Lucene来说,它只能够删除整个旧的文档,然后向索引添加新文档。Index\-Writer中的update\-Document(Term,Document)方法是通过调用deleteDocuments(Term)和\\add\-Document(Document)方法来实现的。所以在本系统中,我们采用如下方法更新索引中的文档:

1.使用IndexWriter的deleteDocuments(Query query);删除指定FTP服务器的所有在索引中的文件信息。

2.调用4.3.5的索引过程,将新的FTP服务器文件信息添加到索引中。
\section{搜索模块}
给用户提供搜索功能。所有的搜索操作的被封装在FTPSearcher类中。
\subsection{Levenshtein Distance算法设计}
算法设计如下:

Step 1

Set n to be the length of s.Set m to be the length of t.

If n = 0, return m and exit.If m = 0, return n and exit.

Construct a matrix containing 0..m rows and 0..n columns.

Step 2

Initialize the first row to 0..n.

Initialize the first column to 0..m.

Step 3 

Examine each character of s (i from 1 to n).

Step 4 

Examine each character of t (j from 1 to m).

Step 5 

If s[i] equals t[j], the cost is 0.

If s[i] doesn’t equal t[j], the cost is 1.

Step 6

Set cell d[I,j] of the matrix equal to the minimum of:

a. The cell immediately above plus 1: d[i-1,j] + 1.

b. The cell immediately to the left plus 1: d[i,j-1] + 1.
S
c. The cell diagonally above and to the left plus the cost: d[i-1,j-1] + cost.

Step 7

After the iteration steps (3, 4, 5, 6) are complete, the distance is found in cell d[n,m]. 

1、  得到源串s长度n与目标串t的长度m,如果一方为的长度0,则返回另一方的长度。

2、  初始化(n+1)*(m+1)的矩阵d,第一行第一列的值为0增至对应的长度。

3、  遍历数组中的每一个字符(i,j从1开始)。如果s[i]与t[j]的值相等,cost值为0,否则为1。D[i][j]的值为d[i-1,j] + 1(左边的值加1)、d[i,j-1] + 1.(上边的值加1)、d[i-1,j-1] + cost (斜上角的值加cost) 中的最小者。

4、  等第三步遍历完后,右下角d[n,m]的值就为两个字符串的LD距离,如图\ref{LD1}所示。
\newpage
\pic[htbp]{LD算法示意图}{width=0.6\textwidth}{LD1}

如果单纯靠编辑距离来匹配的话,搜索结果常常比较生硬,所以进行如下处理,得到distance之后,计算出匹配得分score,公式如下:

\[score=1-\frac{distance}{min(m,n)}\] 

将score作为一个变参数,在使用函数的时候传入,可比较灵活的定义匹配的模糊程度,简单地说score越大这两个项之间的相关性越大,score越小,相关性越小。

\subsection{相关类及方法}
IndexReader类:这是一种抽象类,为访问Lucene索引提供了一个接口,任何对索引的搜索操作都需要通过IndexReader类的这个接口。它为IndexSearcher的搜索提供基础。使用该类的静态方法IndexReader.open(Directory dir)来获取IndexReader\\的实例,和4.3.2中构建IndexWriter的对象实例一样,它同样需要指定lucene索引的存放位置。

IndexSearcher类:该类是搜索索引的门户,所有的搜索都通过IndexSearcher的实例来完成。当构建好了IndexReader对象实例后,使用IndexSearcher带有参数的构造方法IndexSearcher(reader)来构建IndexSearcher对象实例,这个参数是Index\-Reader的对象实例。

QueryParser类:将用户输入的查询语句解析为Query对象,使得IndexSearcher的实例能够通过Query对象完成查询任务。通过QueryParser的parse(String query)方法来解析一个查询字符串并返回一个Query对象。对于构建QueryParser类的对象实例,使用其带参数的构造方法:QueryParser(Version matchVersion, String fieldname, Analyzer defaultAnalyzer)。参数含义见表\ref{table7}。

\pictable[htbp]{QueryParser构造方法参数含义}{width=0.8\textwidth}{table7}

TopDocs类:存储IndexSearcher实例返回的结果。
ScoreDoc类:提供对TopDocs中每条搜索结果的访问接口。
在构建了IndexSearcher的对象实例后,通过调用它的search\\(Query query, int n)方法来进行搜索了,其中,第一个参数是Query的对象实例,第二个参数限制返回结果的数量,n大于0,也就是说search方法返回的结果数会小于等于n,这n个结果是与查询条件相关性最大的前n个。执行结果会返回一个TopDocs对象。在TopDocs.totalHits中存放着实际符合要求的文档的数量。当希望能够让Index\-Searcher实例返回所有符合条件的文档时,可以运用如下编程方式来实现:

\noindent
\ttfamily
\hlstd{\hllin{1\ }TopDocs\ hits\ }\hlopt{=\ }\hlstd{searcher}\hlopt{.}\hlstd{}\hlkwd{search}\hlstd{}\hlopt{(}\hlstd{query}\hlopt{,}\hlstd{}\hlnum{1}\hlstd{}\hlopt{);}\\
\hllin{2\ }\hlstd{}\hlkwa{if\ }\hlstd{}\hlopt{(}\hlstd{hits}\hlopt{.}\hlstd{totalHits\ }\hlopt{\symbol{62}\ }\hlstd{}\hlnum{0}\hlstd{}\hlopt{)}\\
\hllin{3\ }\hlstd{}\hlstd{\ \ \ \ }\hlstd{hits\ }\hlopt{=\ }\hlstd{searcher}\hlopt{.}\hlstd{}\hlkwd{search}\hlstd{}\hlopt{(}\hlstd{query}\hlopt{,}\hlstd{hits}\hlopt{.}\hlstd{totalHits}\hlopt{);}\\
\hllin{4\ }\hlstd{}\hlkwa{else}\\
\hllin{5\ }\hlstd{}\hlstd{\ \ \ \ }\hlstd{hits\ }\hlopt{=\ }\hlstd{searcher}\hlopt{.}\hlstd{}\hlkwd{search}\hlstd{}\hlopt{(}\hlstd{query}\hlopt{,}\hlstd{}\hlnum{1}\hlstd{}\hlopt{);}\\
\hllin{6\ }\hlstd{} 
\mbox{}
\normalfont
\normalsize 


第一次调用search()方法用于获知共有多少个文档符合查询语句的要求,由于search(Query query, int n)中的参数n需要大于0,故需要进行判断hits.totalHits是否大于0,当hits.totalHits 等于0时,search方法会抛出错误。

返回的TopDocs对象包含了一个ScoreDoc数组,ScoreDoc数组中是搜索结果对应文档的文档ID。使用IndexSearcher的doc(int docID)方法,就能把结果文档从Luc\-ene索引目录中取出。

\subsection{处理流程及实现}
\newpage
在本系统中,整个搜索过程如图\ref{image032}所示,

\pic[htpb]{搜索流程图}{width=0.5\textwidth}{image032}

在流程的最后会生成个FTPResults类的数组,该数组用于存储搜索结果,用于向Web前端输出。其结构如下:

\noindent
\ttfamily
\hlstd{}\hllin{1\ }\hlkwa{public\ class\ }\hlstd{FTPResults\ }\hlopt{\symbol{123}}\\
\hllin{2\ }\hlstd{}\hlstd{\ \ \ \ }\hlstd{}\hlkwa{private\ }\hlstd{String\ name}\hlopt{,}\hlstd{type}\hlopt{,}\hlstd{size}\hlopt{,}\hlstd{date}\hlopt{,}\hlstd{path}\hlopt{,}\hlstd{url}\hlopt{;}\\
\hllin{3\ }\hlstd{}\hlstd{\ \ \ \ }\hlstd{……\\
\hllin{4\ }}\hlstd{\ \ \ \ }\hlstd{省略了部分getter和setter\\
\hllin{5\ }}\hlstd{\ \ \ \ }\hlstd{……\\
\hllin{6\ }}\hlstd{\ \ \ \ }\hlstd{}\hlslc{//将Byte转换到其他单位}\\
\hllin{7\ }\hlstd{}\hlstd{\ \ \ \ }\hlstd{}\hlkwa{public\ }\hlstd{}\hlkwb{void\ }\hlstd{}\hlkwd{setSize}\hlstd{}\hlopt{(}\hlstd{String\ size}\hlopt{)\ \symbol{123}}\\
\hllin{8\ }\hlstd{}\hlstd{\ \ \ \ \ \ \ \ }\hlstd{}\hlkwb{long\ }\hlstd{numofsize\ }\hlopt{=\ }\hlstd{Long}\hlopt{.}\hlstd{}\hlkwd{valueOf}\hlstd{}\hlopt{(}\hlstd{size}\hlopt{);}\\
\hllin{9\ }\hlstd{}\hlstd{\ \ \ \ \ \ \ \ }\hlstd{}\hlkwb{int\ }\hlstd{level\ }\hlopt{=\ }\hlstd{}\hlnum{0}\hlstd{}\hlopt{;}\\
\hllin{10\ }\hlstd{}\hlstd{\ \ \ \ \ \ \ \ }\hlstd{}\hlkwa{while\ }\hlstd{}\hlopt{(}\hlstd{numofsize\ }\hlopt{\symbol{62}\ }\hlstd{}\hlnum{1023}\hlstd{}\hlopt{)\ \symbol{123}}\\
\hllin{11\ }\hlstd{}\hlstd{\ \ \ \ \ \ \ \ \ \ \ \ }\hlstd{numofsize\ }\hlopt{=\ }\hlstd{numofsize\ }\hlopt{/\ }\hlstd{}\hlnum{1024}\hlstd{}\hlopt{;}\\
\hllin{12\ }\hlstd{}\hlstd{\ \ \ \ \ \ \ \ \ \ \ \ }\hlstd{level}\hlopt{++;}\\
\hllin{13\ }\hlstd{}\hlstd{\ \ \ \ \ \ \ \ }\hlstd{}\hlopt{\symbol{125}}\\
\hllin{14\ }\hlstd{}\hlstd{\ \ \ \ \ \ \ \ }\hlstd{}\hlkwa{if\ }\hlstd{}\hlopt{(}\hlstd{level\ }\hlopt{==\ }\hlstd{}\hlnum{0}\hlstd{}\hlopt{)\ \symbol{123}}\\
\hllin{15\ }\hlstd{}\hlstd{\ \ \ \ \ \ \ \ \ \ \ \ }\hlstd{}\hlkwa{this}\hlstd{}\hlopt{.}\hlstd{size\ }\hlopt{=\ }\hlstd{}\hlstr{"大小:"}\hlstd{\ }\hlopt{+\ }\hlstd{numofsize\ }\hlopt{+\ }\hlstd{}\hlstr{"B"}\hlstd{}\hlopt{;}\\
\hllin{16\ }\hlstd{}\hlstd{\ \ \ \ \ \ \ \ }\hlstd{}\hlopt{\symbol{125}}\\
\hllin{17\ }\hlstd{}\hlstd{\ \ \ \ \ \ \ \ }\hlstd{}\hlkwa{if\ }\hlstd{}\hlopt{(}\hlstd{level\ }\hlopt{==\ }\hlstd{}\hlnum{1}\hlstd{}\hlopt{)\ \symbol{123}}\\
\hllin{18\ }\hlstd{}\hlstd{\ \ \ \ \ \ \ \ \ \ \ \ }\hlstd{}\hlkwa{this}\hlstd{}\hlopt{.}\hlstd{size\ }\hlopt{=\ }\hlstd{}\hlstr{"大小:"}\hlstd{\ }\hlopt{+\ }\hlstd{numofsize\ }\hlopt{+\ }\hlstd{}\hlstr{"KB"}\hlstd{}\hlopt{;}\\
\hllin{19\ }\hlstd{}\hlstd{\ \ \ \ \ \ \ \ }\hlstd{}\hlopt{\symbol{125}}\\
\hllin{20\ }\hlstd{}\hlstd{\ \ \ \ \ \ \ \ }\hlstd{}\hlkwa{if\ }\hlstd{}\hlopt{(}\hlstd{level\ }\hlopt{==\ }\hlstd{}\hlnum{2}\hlstd{}\hlopt{)\ \symbol{123}}\\
\hllin{21\ }\hlstd{}\hlstd{\ \ \ \ \ \ \ \ \ \ \ \ }\hlstd{}\hlkwa{this}\hlstd{}\hlopt{.}\hlstd{size\ }\hlopt{=\ }\hlstd{}\hlstr{"大小:"}\hlstd{\ }\hlopt{+\ }\hlstd{numofsize\ }\hlopt{+\ }\hlstd{}\hlstr{"MB"}\hlstd{}\hlopt{;}\\
\hllin{22\ }\hlstd{}\hlstd{\ \ \ \ \ \ \ \ }\hlstd{}\hlopt{\symbol{125}}\\
\hllin{23\ }\hlstd{}\hlstd{\ \ \ \ \ \ \ \ }\hlstd{}\hlkwa{if\ }\hlstd{}\hlopt{(}\hlstd{level\ }\hlopt{==\ }\hlstd{}\hlnum{3}\hlstd{}\hlopt{)\ \symbol{123}}\\
\hllin{24\ }\hlstd{}\hlstd{\ \ \ \ \ \ \ \ \ \ \ \ }\hlstd{}\hlkwa{this}\hlstd{}\hlopt{.}\hlstd{size\ }\hlopt{=\ }\hlstd{}\hlstr{"大小:"}\hlstd{\ }\hlopt{+\ }\hlstd{numofsize\ }\hlopt{+\ }\hlstd{}\hlstr{"GB"}\hlstd{}\hlopt{;}\\
\hllin{25\ }\hlstd{}\hlstd{\ \ \ \ \ \ \ \ }\hlstd{}\hlopt{\symbol{125}}\\
\hllin{26\ }\hlstd{}\hlstd{\ \ \ \ }\hlstd{}\hlopt{\symbol{125}}\\
\hllin{27\ }\hlstd{}\hlslc{//构造文件URL地址}\\
\hllin{28\ }\hlstd{}\hlstd{\ \ \ \ }\hlstd{}\hlkwa{public\ }\hlstd{}\hlkwb{void\ }\hlstd{}\hlkwd{setUrl}\hlstd{}\hlopt{(}\hlstd{String\ path}\hlopt{,\ }\hlstd{String\ name}\hlopt{,\ }\hlstd{String\ type}\hlopt{)}\Righttorque\\
\hllin{29\ }\hlstd{}\hlstd{\ \ \ \ }\hlstd{}\hlopt{\symbol{123}}\\
\hllin{30\ }\hlstd{}\hlstd{\ \ \ \ \ \ \ \ }\hlstd{}\hlkwa{if\ }\hlstd{}\hlopt{(}\hlstd{type}\hlopt{.}\hlstd{}\hlkwd{equals}\hlstd{}\hlopt{(}\hlstd{}\hlstr{"folder"}\hlstd{}\hlopt{))\ \symbol{123}}\\
\hllin{31\ }\hlstd{}\hlstd{\ \ \ \ \ \ \ \ \ \ \ \ }\hlstd{}\hlkwa{this}\hlstd{}\hlopt{.}\hlstd{url\ }\hlopt{=\ }\hlstd{path}\hlopt{+}\hlstd{name}\hlopt{+}\hlstd{}\hlstr{"/"}\hlstd{}\hlopt{;}\\
\hllin{32\ }\hlstd{}\hlstd{\ \ \ \ \ \ \ \ }\hlstd{}\hlopt{\symbol{125}\ }\hlstd{}\hlkwa{else\ }\hlstd{}\hlopt{\symbol{123}}\\
\hllin{33\ }\hlstd{}\hlstd{\ \ \ \ \ \ \ \ \ \ \ \ }\hlstd{}\hlkwa{this}\hlstd{}\hlopt{.}\hlstd{url\ }\hlopt{=\ }\hlstd{path}\hlopt{+}\hlstd{name}\hlopt{+}\hlstd{}\hlstr{"."}\hlstd{}\hlopt{+}\hlstd{type}\hlopt{;}\\
\hllin{34\ }\hlstd{}\hlstd{\ \ \ \ \ \ \ \ }\hlstd{}\hlopt{\symbol{125}}\\
\hllin{35\ }\hlstd{}\hlstd{\ \ \ \ }\hlstd{}\hlopt{\symbol{125}}\\
\hllin{36\ }\hlstd{}\hlopt{\symbol{125}}\\
\hllin{37\ }\hlstd{} 
\mbox{}
\normalfont
\normalsize


\section{基于MyBatis的数据库操作}
\subsection{MyBatis程序结构}
首先,Mybatis需要有一个全局配置文件。在全局配置文件中需要配置的信息主要包括如下几个方面\citeup{10}:

Properties:用于提供一系列的键值对组成的属性信息,该属性信息可以用于整个配置文件中。

Settings:用于设置 MyBatis 的运行时方式,比如是否启用延迟加载等。

typeAliases :为 Java 类型指定别名,可以在 XML 文件中用别名取代 Java 类的全限定名。

typeHandlers:在 MyBatis 通过 PreparedStatement 为占位符设置值,或者从 ResultSet 取出值时,特定类型的类型处理器会被执行。

objectFactory:MyBatis 通过 ObjectFactory 来创建结果对象。可以通过继承 DefaultObjectFactory 来实现自己的 ObjectFactory 类。

Plugins:用于配置一系列拦截器,用于拦截映射 SQL 语句的执行。可以通过实现 Interceptor 接口来实现自己的拦截器。

Environments:用于配置数据源信息,包括连接池、事务属性等。

Mappers:程序中所有用到的 SQL 映射文件都在这里列出,这些映射 SQL 都被 MyBatis 管理。

上面提及的大多数元素都不是必需的,通常 MyBatis 会为没有显式设置的元素提供缺省值。下面给出一个简单的全局配置代码:

\noindent
\ttfamily
\hlstd{}\hllin{1\ }\hlkwa{\symbol{60}?xml\ }\hlstd{}\hlkwb{version}\hlstd{=}\hlstr{"1.0"}\hlstd{\ }\hlkwb{encoding}\hlstd{=}\hlstr{"UTF{-}8"}\hlstd{\ }\hlkwa{?\symbol{62}}\\
\hllin{2\ }\hlstd{}\hlkwa{\symbol{60}!DOCTYPE\ }\hlstd{configuration\\
\hllin{3\ }}\hlstd{\ \ }\hlstd{PUBLIC\ }\hlstr{"{-}//mybatis.org//DTD\ Config\ 3.0//EN"}\hlstd{\\
\hllin{4\ }}\hlstd{\ \ }\hlstd{}\hlstr{"http://mybatis.org/dtd/mybatis{-}3{-}config.dtd"}\hlstd{}\hlkwa{\symbol{62}}\\
\hllin{5\ }\hlstd{}\hlkwa{\symbol{60}configuration\symbol{62}}\\
\hllin{6\ }\hlstd{}\hlstd{\ \ }\hlstd{}\hlkwa{\symbol{60}environments\ }\hlstd{}\hlkwb{default}\hlstd{=}\hlstr{"development"}\hlstd{}\hlkwa{\symbol{62}}\\
\hllin{7\ }\hlstd{}\hlstd{\ \ \ \ }\hlstd{}\hlkwa{\symbol{60}environment\ }\hlstd{}\hlkwb{id}\hlstd{=}\hlstr{"development"}\hlstd{}\hlkwa{\symbol{62}}\\
\hllin{8\ }\hlstd{}\hlstd{\ \ \ \ \ \ }\hlstd{}\hlkwa{\symbol{60}transactionManager\ }\hlstd{}\hlkwb{type}\hlstd{=}\hlstr{"JDBC"}\hlstd{}\hlkwa{/\symbol{62}}\\
\hllin{9\ }\hlstd{}\hlstd{\ \ \ \ \ \ }\hlstd{}\hlkwa{\symbol{60}dataSource\ }\hlstd{}\hlkwb{type}\hlstd{=}\hlstr{"POOLED"}\hlstd{}\hlkwa{\symbol{62}}\\
\hllin{10\ }\hlstd{}\hlstd{\ \ \ \ \ \ \ \ \ \ }\hlstd{}\hlkwa{\symbol{60}property\ }\hlstd{}\hlkwb{name}\hlstd{=}\hlstr{"driver"}\hlstd{\ }\hlkwb{value}\hlstd{=}\hlstr{"com.mysql.jdbc.}\Righttorque\\
\hllin{11\ }\hlstr{}\hlstd{\ \ \ \ \ \ \ \ \ \ }\hlstr{Driver"}\hlstd{}\hlkwa{/\symbol{62}}\\
\hllin{12\ }\hlstd{}\hlstd{\ \ \ \ \ \ \ \ \ \ }\hlstd{}\hlkwa{\symbol{60}property\ }\hlstd{}\hlkwb{name}\hlstd{=}\hlstr{"url"}\hlstd{\ }\hlkwb{value}\hlstd{=}\hlstr{"jdbc:mysql://localhost:}\Righttorque\\
\hllin{13\ }\hlstr{}\hlstd{\ \ \ \ \ \ \ \ \ \ }\hlstr{3306/test?useUnicode=true\&amp;}\Righttorque\\
\hllin{14\ }\hlstr{}\hlstd{\ \ \ \ \ \ \ \ \ \ }\hlstr{characterEncoding=UTF8"}\hlstd{}\hlkwa{/\symbol{62}}\\
\hllin{15\ }\hlstd{}\hlstd{\ \ \ \ \ \ \ \ \ \ }\hlstd{}\hlkwa{\symbol{60}property\ }\hlstd{}\hlkwb{name}\hlstd{=}\hlstr{"username"}\hlstd{\ }\hlkwb{value}\hlstd{=}\hlstr{"root"}\hlstd{}\hlkwa{/\symbol{62}}\\
\hllin{16\ }\hlstd{}\hlstd{\ \ \ \ \ \ \ \ \ \ }\hlstd{}\hlkwa{\symbol{60}property\ }\hlstd{}\hlkwb{name}\hlstd{=}\hlstr{"password"}\hlstd{\ }\hlkwb{value}\hlstd{=}\hlstr{"940920"}\hlstd{}\hlkwa{/\symbol{62}}\\
\hllin{17\ }\hlstd{}\hlstd{\ \ \ \ \ \ }\hlstd{}\hlkwa{\symbol{60}/dataSource\symbol{62}}\\
\hllin{18\ }\hlstd{}\hlstd{\ \ \ \ }\hlstd{}\hlkwa{\symbol{60}/environment\symbol{62}}\\
\hllin{19\ }\hlstd{}\hlstd{\ \ }\hlstd{}\hlkwa{\symbol{60}/environments\symbol{62}}\\
\hllin{20\ }\hlstd{}\hlstd{\ \ }\hlstd{}\hlkwa{\symbol{60}mappers\symbol{62}}\\
\hllin{21\ }\hlstd{}\hlstd{\ \ \ \ }\hlstd{}\hlkwa{\symbol{60}mapper\ }\Righttorque\\
\hllin{22\ }\hlstd{}\hlstd{\ \ \ \ }\hlstd{}\hlkwb{resource}\hlstd{=}\hlstr{"ftpSearcher/mybatis/persistence/FeedbackMapper.}\Righttorque\\
\hllin{23\ }\hlstr{}\hlstd{\ \ \ \ }\hlstr{xml"}\hlstd{}\hlkwa{/\symbol{62}}\\
\hllin{24\ }\hlstd{}\hlstd{\ \ \ \ }\hlstd{}\hlkwa{\symbol{60}mapper\ }\Righttorque\\
\hllin{25\ }\hlstd{}\hlstd{\ \ \ \ }\hlstd{}\hlkwb{resource}\hlstd{=}\hlstr{"ftpSearcher/mybatis/persistence/FTPServerMapper}\Righttorque\\
\hllin{26\ }\hlstr{}\hlstd{\ \ \ \ }\hlstr{.xml"}\hlstd{}\hlkwa{/\symbol{62}}\\
\hllin{27\ }\hlstd{}\hlstd{\ \ \ \ }\hlstd{}\hlkwa{\symbol{60}mapper\ }\Righttorque\\
\hllin{28\ }\hlstd{}\hlstd{\ \ \ \ }\hlstd{}\hlkwb{resource}\hlstd{=}\hlstr{"ftpSearcher/mybatis/persistence/UserQueryLogMap}\Righttorque\\
\hllin{29\ }\hlstr{}\hlstd{\ \ \ \ }\hlstr{per.xml"}\hlstd{}\hlkwa{/\symbol{62}}\\
\hllin{30\ }\hlstd{}\hlstd{\ \ \ \ }\hlstd{}\hlkwa{\symbol{60}mapper\ }\Righttorque\\
\hllin{31\ }\hlstd{}\hlstd{\ \ \ \ }\hlstd{}\hlkwb{resource}\hlstd{=}\hlstr{"ftpSearcher/mybatis/persistence/PathPriorityMap}\Righttorque\\
\hllin{32\ }\hlstr{}\hlstd{\ \ \ \ }\hlstr{per.xml"}\hlstd{}\hlkwa{/\symbol{62}}\\
\hllin{33\ }\hlstd{}\hlstd{\ \ \ \ }\hlstd{}\hlkwa{\symbol{60}mapper\ }\Righttorque\\
\hllin{34\ }\hlstd{}\hlstd{\ \ \ \ }\hlstd{}\hlkwb{resource}\hlstd{=}\hlstr{"ftpSearcher/mybatis/persistence/QueryCountMappe}\Righttorque\\
\hllin{35\ }\hlstr{}\hlstd{\ \ \ \ }\hlstr{r.xml"}\hlstd{}\hlkwa{/\symbol{62}}\\
\hllin{36\ }\hlstd{}\hlstd{\ \ \ \ }\hlstd{}\hlkwa{\symbol{60}mapper\ }\Righttorque\\
\hllin{37\ }\hlstd{}\hlstd{\ \ \ \ }\hlstd{}\hlkwb{resource}\hlstd{=}\hlstr{"ftpSearcher/mybatis/persistence/FTPSpiderLogMap}\Righttorque\\
\hllin{38\ }\hlstr{}\hlstd{\ \ \ \ }\hlstr{per.xml"}\hlstd{}\hlkwa{/\symbol{62}}\\
\hllin{39\ }\hlstd{}\hlstd{\ \ }\hlstd{}\hlkwa{\symbol{60}/mappers\symbol{62}}\\
\hllin{40\ }\hlstd{}\hlkwa{\symbol{60}/configuration\symbol{62}}\hlstd{} 
\mbox{}
\normalfont
\normalsize


有了这些信息,MyBatis\citeup{11}便能够和数据库建立连接,并应用给定的连接池信息和事务属性。MyBatis 封装了这些操作,最终暴露一个 SqlSessionFactory 实例供开发者使用,从名字可以看出来,这是一个创建 SqlSession 的工厂类,通过 SqlSession 实例,开发者能够直接进行业务逻辑的操作,而不需要重复编写 JDBC 相关的样板代码。根据全局配置文件生成SqlSession的代码如下:

\noindent
\ttfamily
\hlstd{\hllin{1\ }String\ resource\ }\hlopt{=\ }\hlstd{}\hlstr{"/mybatis{-}config.xml"}\hlstd{}\hlopt{;}\\
\hllin{2\ }\hlstd{InputStream\ inputStream\ }\hlopt{=\ }\hlstd{Resources}\hlopt{.}\hlstd{}\hlkwd{getResourceAsStream}\hlstd{}\hlopt{(}\Righttorque\\
\hllin{3\ }\hlstd{resource}\hlopt{);}\\
\hllin{4\ }\hlstd{SqlSessionFactory\ sqlSessionFactory\ }\hlopt{=\ }\hlstd{}\hlkwa{new\ }\Righttorque\\
\hllin{5\ }\hlstd{}\hlkwd{SqlSessionFactoryBuilder}\hlstd{}\hlopt{().}\hlstd{}\hlkwd{build}\hlstd{}\hlopt{(}\hlstd{inputStream}\hlopt{);}\\
\hllin{6\ }\hlstd{SqlSession\ session\ }\hlopt{=\ }\hlstd{sqlSessionFactory}\hlopt{.}\hlstd{}\hlkwd{openSession}\hlstd{}\hlopt{();}\\
\hllin{7\ }\hlstd{} 
\mbox{}
\normalfont
\normalsize


可以把上面的代码看做是MyBatis创建 SqlSession 的样板代码。其中第二行代码在类路径上加载配置文件,Resources 是 MyBatis 提供的一个工具类,它用于简化资源文件的加载,它可以访问各种路径的文件,不过最常用的还是示例中这种基于类路径的表示方式。

在完成全局配置文件,并通过 MyBatis 获得 SqlSession 对象之后,便可以执行数据访问操作了。对于 MyBatis 而言,要执行的操作其实就是在映射文件中配置的 SQL 语句。配置代码如下:

\noindent
\ttfamily
\hlstd{}\hllin{1\ }\hlopt{\symbol{60}}\hlstd{?xml\ version}\hlopt{=}\hlstd{}\hlstr{"1.0"}\hlstd{\ encoding}\hlopt{=}\hlstd{}\hlstr{"UTF{-}8"}\hlstd{?}\hlopt{\symbol{62}}\\
\hllin{2\ }\hlstd{}\hlopt{\symbol{60}!}\hlstd{DOCTYPE\ mapper\ PUBLIC\ }\hlstr{"{-}//mybatis.org//DTD\ Mapper\ 3.0//EN"}\hlstd{\ }\\
\hllin{3\ }\hlstr{"http://mybatis.org/dtd/mybatis{-}3{-}mapper.dtd"}\hlstd{}\hlopt{\symbol{62}}\\
\hllin{4\ }\hlstd{}\\
\hllin{5\ }\hlopt{\symbol{60}}\hlstd{mapper\ namespace}\hlopt{=}\hlstd{}\hlstr{"ftpSearcher.mybatis.mapper.}\Righttorque\\
\hllin{6\ }\hlstr{FeedbackMapper"}\hlstd{}\hlopt{\symbol{62}}\\
\hllin{7\ }\hlstd{\\
\hllin{8\ }}\hlstd{\ \ \ \ }\hlstd{}\hlopt{\symbol{60}}\hlstd{insert\ id}\hlopt{=}\hlstd{}\hlstr{"insertFeedback"}\hlstd{\ parameterType}\hlopt{=}\hlstd{}\hlstr{"ftpSearcher.}\Righttorque\\
\hllin{9\ }\hlstr{}\hlstd{\ \ \ \ }\hlstr{model.Feedback"}\hlstd{}\hlopt{\symbol{62}}\\
\hllin{10\ }\hlstd{}\hlstd{\ \ \ \ \ \ \ \ }\hlstd{INSERT\ INTO\ feedback\\
\hllin{11\ }}\hlstd{\ \ \ \ \ \ \ \ }\hlstd{}\hlopt{(}\hlstd{username}\hlopt{,\ }\hlstd{comment}\hlopt{,}\hlstd{commentdate}\hlopt{)\ }\hlstd{VALUES\\
\hllin{12\ }}\hlstd{\ \ \ \ \ \ \ \ }\hlstd{}\hlopt{(}\hlstd{\#}\hlopt{\symbol{123}}\hlstd{username}\hlopt{\symbol{125},}\hlstd{\#}\hlopt{\symbol{123}}\hlstd{comment}\hlopt{\symbol{125},}\hlstd{\#}\hlopt{\symbol{123}}\hlstd{commentdate}\hlopt{\symbol{125})}\\
\hllin{13\ }\hlstd{}\hlstd{\ \ \ \ }\hlstd{}\hlopt{\symbol{60}/}\hlstd{insert}\hlopt{\symbol{62}}\\
\hllin{14\ }\hlstd{}\hlstd{\ \ \ \ }\hlstd{}\hlopt{\symbol{60}}\hlstd{delete\ id}\hlopt{=}\hlstd{}\hlstr{"deleteByID"}\hlstd{\ parameterType}\hlopt{=}\hlstd{}\hlstr{"int"}\hlstd{}\hlopt{\symbol{62}}\\
\hllin{15\ }\hlstd{}\hlstd{\ \ \ \ \ \ \ \ }\hlstd{delete\ from\ feedback\ where\ id\ }\hlopt{=\ }\hlstd{\#}\hlopt{\symbol{123}}\hlstd{id}\hlopt{\symbol{125}}\\
\hllin{16\ }\hlstd{}\hlstd{\ \ \ \ }\hlstd{}\hlopt{\symbol{60}/}\hlstd{delete}\hlopt{\symbol{62}}\\
\hllin{17\ }\hlstd{}\hlopt{\symbol{60}/}\hlstd{mapper}\hlopt{\symbol{62}}\hlstd{} 
\mbox{}
\normalfont
\normalsize


在 MyBatis 中,namespace使得映射文件与接口绑定变得非常自然。下面将展示,XML映射文件与接口绑定之间的关系,与上述代码对应接口的定义代码如下:

\noindent
\ttfamily
\hlstd{}\hllin{1\ }\hlkwa{package\ }\hlstd{ftpSearcher}\hlopt{.}\hlstd{mybatis}\hlopt{.}\hlstd{mapper}\hlopt{;}\\
\hllin{2\ }\hlstd{}\hlkwa{import\ }\hlstd{java}\hlopt{.}\hlstd{util}\hlopt{.}\hlstd{List}\hlopt{;}\\
\hllin{3\ }\hlstd{}\hlkwa{import\ }\hlstd{ftpSearcher}\hlopt{.}\hlstd{model}\hlopt{.}\hlstd{Feedback}\hlopt{;}\\
\hllin{4\ }\hlstd{}\\
\hllin{5\ }\hlkwa{public\ interface\ }\hlstd{FeedbackMapper\ }\hlopt{\symbol{123}}\\
\hllin{6\ }\hlstd{}\hlstd{\ \ \ \ }\hlstd{}\hlkwa{public\ }\hlstd{}\hlkwb{void\ }\hlstd{}\hlkwd{insertFeedback}\hlstd{}\hlopt{(}\hlstd{Feedback\ feedback}\hlopt{);}\\
\hllin{7\ }\hlstd{}\hlstd{\ \ \ \ }\hlstd{}\hlkwa{public\ }\hlstd{List}\hlopt{\symbol{60}}\hlstd{Feedback}\hlopt{\symbol{62}\ }\hlstd{}\hlkwd{getAllFeedback}\hlstd{}\hlopt{();}\\
\hllin{8\ }\hlstd{}\hlstd{\ \ \ \ }\hlstd{}\hlkwa{public\ }\hlstd{}\hlkwb{void\ }\hlstd{}\hlkwd{deleteByID}\hlstd{}\hlopt{(}\hlstd{}\hlkwb{int\ }\hlstd{id}\hlopt{);}\\
\hllin{9\ }\hlstd{}\hlopt{\symbol{125}}\\
\hllin{10\ }\hlstd{} 
\mbox{}
\normalfont
\normalsize


表\ref{table8}给出了XML映射文件与接口直接的对应关系,以insertFeedback和getAll\-Feedback属性为例\citeup{10}。

\pictable[htbp]{XML映射与接口对应关系}{width=0.8\textwidth}{table8}

在MyBatis框架中,只需要声明而不需要实现接口。完成了映射文件配置SQL语句及接口定义后,就可以用Mybatis的方式来进行相关数据库操作了。一个删除用户反馈的示例代码如下:

\noindent
\ttfamily
\hlstd{}\hllin{1\ }\hlslc{//假定session对象实例已被构建}\\
\hllin{2\ }\hlstd{FeedbackMapper\ feedbackMapper\ }\hlopt{=\ }\hlstd{session}\hlopt{.}\hlstd{}\hlkwd{getMapper}\hlstd{}\hlopt{(}\Righttorque\\
\hllin{3\ }\hlstd{FeedbackMapper}\hlopt{.}\hlstd{}\hlkwa{class}\hlstd{}\hlopt{);}\\
\hllin{4\ }\hlstd{feedbackMapper}\hlopt{.}\hlstd{}\hlkwd{deleteByID}\hlstd{}\hlopt{(}\hlstd{feedbackID}\hlopt{);}\\
\hllin{5\ }\hlstd{session}\hlopt{.}\hlstd{}\hlkwd{commit}\hlstd{}\hlopt{();}\\
\hllin{6\ }\hlstd{session}\hlopt{.}\hlstd{}\hlkwd{close}\hlstd{}\hlopt{();}\\
\hllin{7\ }\hlstd{} 
\mbox{}
\normalfont
\normalsize


在进行了删除操作之后一定要使用session.commit()将事务提交到数据库,否则删除、插入、修改这类操作是无法生效的。完成操作后,使用session.close()关闭与数据库的连接。
\subsection{数据库操作}
下面列举了一些系统中对数据库的操作。

1. FTP服务器表相关操作:

\noindent
\ttfamily
\hlstd{\hllin{1\ }添加新的FTP服务器信息:INSERT\ INTO\ }\hlkwd{ftpserver\ }\hlstd{}\hlopt{(}\Righttorque\\
\hllin{2\ }\hlstd{domain}\hlopt{,\ }\hlstd{ipv4}\hlopt{,\ }\hlstd{port}\hlopt{,\ }\hlstd{submittime}\hlopt{,\ }\hlstd{description}\hlopt{)\ }\hlstd{}\hlkwd{VALUES\ }\hlstd{}\hlopt{(}\hlstd{\#}\hlopt{\symbol{123}}\Righttorque\\
\hllin{3\ }\hlstd{domain}\hlopt{\symbol{125},}\hlstd{\#}\hlopt{\symbol{123}}\hlstd{ipv4}\hlopt{\symbol{125},}\hlstd{\#}\hlopt{\symbol{123}}\hlstd{port}\hlopt{\symbol{125},}\hlstd{\#}\hlopt{\symbol{123}}\hlstd{submittime}\hlopt{\symbol{125},}\hlstd{\#}\hlopt{\symbol{123}}\hlstd{description}\hlopt{\symbol{125})}\\
\hllin{4\ }\hlstd{获取所有FTP服务器信息:SELECT\ }\hlopt{{*}\ }\hlstd{FROM\ ftpserver\\
\hllin{5\ }获取指定FTP服务器信息:SELECT\ }\hlopt{{*}\ }\hlstd{FROM\ ftpserver\ \Righttorque\\
\hllin{6\ }WHERE\ id\ }\hlopt{=\ }\hlstd{\#}\hlopt{\symbol{123}}\hlstd{id}\hlopt{\symbol{125}}\\
\hllin{7\ }\hlstd{更改指定FTP服务器信息:UPDATE\ ftpserver\ SET\ domain\ }\Righttorque\\
\hllin{8\ }\hlopt{=\ }\hlstd{\#}\hlopt{\symbol{123}}\hlstd{domain}\hlopt{\symbol{125},\ }\hlstd{ipv4\ }\hlopt{=\ }\hlstd{\#}\hlopt{\symbol{123}}\hlstd{ipv4}\hlopt{\symbol{125},}\hlstd{port\ }\hlopt{=\ }\hlstd{\#}\hlopt{\symbol{123}}\hlstd{port}\hlopt{\symbol{125},\ }\hlstd{encoding\ }\hlopt{=\ }\hlstd{\#}\hlopt{\symbol{123}}\Righttorque\\
\hllin{9\ }\hlstd{encoding}\hlopt{\symbol{125},\ }\hlstd{verify}\hlopt{=\ }\hlstd{\#}\hlopt{\symbol{123}}\hlstd{verify}\hlopt{\symbol{125}\ }\hlstd{WHERE\ id\ }\hlopt{=\ }\hlstd{\#}\hlopt{\symbol{123}}\hlstd{id}\hlopt{\symbol{125}}\\
\hllin{10\ }\hlstd{删除指定FTP服务器信息:DELETE\ FROM\ ftpserver\ WHERE\ \Righttorque\\
\hllin{11\ }id\ }\hlopt{=\ }\hlstd{\#}\hlopt{\symbol{123}}\hlstd{serverID}\hlopt{\symbol{125}}\hlstd{} 
\mbox{}
\normalfont
\normalsize


2. 用户反馈表相关操作:

\noindent
\ttfamily
\hlstd{\hllin{1\ }添加用户反馈:INSERT\ INTO\ }\hlkwd{feedback\ }\hlstd{}\hlopt{(}\hlstd{username}\hlopt{,\ }\hlstd{comment}\hlopt{,}\Righttorque\\
\hllin{2\ }\hlstd{commentdate}\hlopt{)\ }\hlstd{}\hlkwd{VALUES\ }\hlstd{}\hlopt{(}\hlstd{\#}\hlopt{\symbol{123}}\hlstd{username}\hlopt{\symbol{125},}\hlstd{\#}\hlopt{\symbol{123}}\hlstd{comment}\hlopt{\symbol{125},}\hlstd{\#}\hlopt{\symbol{123}}\hlstd{commentdate}\hlopt{\symbol{125})}\\
\hllin{3\ }\hlstd{读取所有用户反馈:SELECT\ }\hlopt{{*}\ }\hlstd{FROM\ feedback\ ORDER\ BY\ \Righttorque\\
\hllin{4\ }commentdate\ DESC\\
\hllin{5\ }删除用户反馈:DELETE\ FROM\ feedback\ WHERE\ id\ }\hlopt{=\ }\hlstd{\#}\hlopt{\symbol{123}}\hlstd{id}\hlopt{\symbol{125}}\hlstd{} 
\mbox{}
\normalfont
\normalsize


3. 用户搜索日志统计:SELECT queryword, COUNT(*) AS count
		             FROM userquerylog WHERE querypage = 1 
		             GROUP BY queryword ORDER BY COUNT(*) DESC
\section{基于JDOM的XML文档操作}
\subsection{关键类及关键方法介绍}
在本系统中,使用 SAXBuilder 对ftpSearcher-config.xml 文件以SAX的方式进行语法分析,构建JDOM的Document实例。 Document实例是执行XML文档相关操作的基础。下面是构建JDOM文档的过程:

\noindent
\ttfamily
\hlstd{\hllin{1\ }SAXBuilder\ buider\ }\hlopt{=\ }\hlstd{}\hlkwa{new\ }\hlstd{}\hlkwd{SAXBuilder}\hlstd{}\hlopt{();}\\
\hllin{2\ }\hlstd{}\hlslc{//获得ftpSearcher{-}config.xml的位置}\\
\hllin{3\ }\hlstd{URL\ path\ }\hlopt{=\ }\hlstd{}\hlkwa{this}\hlstd{}\hlopt{.}\hlstd{}\hlkwd{getClass}\hlstd{}\hlopt{().}\hlstd{}\hlkwd{getClassLoader}\hlstd{}\hlopt{()}\\
\hllin{4\ }\hlstd{}\hlstd{\ \ \ \ \ \ \ \ \ \ \ }\hlstd{}\hlopt{.}\hlstd{}\hlkwd{getResource}\hlstd{}\hlopt{(}\hlstd{}\hlstr{"ftpSearcher{-}config.xml"}\hlstd{}\hlopt{);}\\
\hllin{5\ }\hlstd{File\ file\ }\hlopt{=\ }\hlstd{}\hlkwa{new\ }\hlstd{}\hlkwd{File}\hlstd{}\hlopt{(}\hlstd{path}\hlopt{.}\hlstd{}\hlkwd{getPath}\hlstd{}\hlopt{())}\\
\hllin{6\ }\hlstd{Document\ doc\ }\hlopt{=\ }\hlstd{buider}\hlopt{.}\hlstd{}\hlkwd{build}\hlstd{}\hlopt{(}\hlstd{file}\hlopt{);}\\
\hllin{7\ }\hlstd{} 
\mbox{}
\normalfont
\normalsize


在构建好Document实例后,就能通过JDOM提供的方法获取XML里中所需要的内容了。

\subsection{FTP搜索引擎配置文件结构及文档操作}
FTP搜索引擎的配置代码如下:

\noindent
\ttfamily
\hlstd{}\hllin{1\ }\hlopt{\symbol{60}}\hlstd{?xml\ version}\hlopt{=}\hlstd{}\hlstr{"1.0"}\hlstd{\ encoding}\hlopt{=}\hlstd{}\hlstr{"UTF{-}8"}\hlstd{?}\hlopt{\symbol{62}}\\
\hllin{2\ }\hlstd{}\hlopt{\symbol{60}}\hlstd{ftpSearcher}\hlopt{\symbol{62}}\\
\hllin{3\ }\hlstd{}\hlstd{\ \ \ \ }\hlstd{}\hlopt{\symbol{60}}\hlstd{fileInfoDir}\hlopt{\symbol{62}}\hlstd{D}\hlopt{:/}\hlstd{FTPSearcher}\hlopt{/}\hlstd{FTPFileInfo}\hlopt{\symbol{60}/}\hlstd{fileInfoDir}\hlopt{\symbol{62}}\\
\hllin{4\ }\hlstd{}\hlstd{\ \ \ \ }\hlstd{}\hlopt{\symbol{60}}\hlstd{indexDir}\hlopt{\symbol{62}}\hlstd{D}\hlopt{:/}\hlstd{FTPSearcher}\hlopt{/}\hlstd{LuceneIndex}\hlopt{\symbol{60}/}\hlstd{indexDir}\hlopt{\symbol{62}}\\
\hllin{5\ }\hlstd{}\hlstd{\ \ \ \ }\hlstd{}\hlopt{\symbol{60}}\hlstd{logDir}\hlopt{\symbol{62}}\hlstd{D}\hlopt{:/}\hlstd{FTPSearcher}\hlopt{/}\hlstd{FTPSearcherLog}\hlopt{\symbol{60}/}\hlstd{logDir}\hlopt{\symbol{62}}\\
\hllin{6\ }\hlstd{}\hlstd{\ \ \ \ }\hlstd{}\hlopt{\symbol{60}}\hlstd{username}\hlopt{\symbol{62}}\hlstd{}\hlnum{21232}\hlstd{f297a57a5a743894a0e4a801fc3}\hlopt{\symbol{60}/}\hlstd{username}\hlopt{\symbol{62}}\\
\hllin{7\ }\hlstd{}\hlstd{\ \ \ \ }\hlstd{}\hlopt{\symbol{60}}\hlstd{password}\hlopt{\symbol{62}}\hlstd{}\hlnum{21232}\hlstd{f297a57a5a743894a0e4a801fc3}\hlopt{\symbol{60}/}\hlstd{password}\hlopt{\symbol{62}}\\
\hllin{8\ }\hlstd{}\hlopt{\symbol{60}/}\hlstd{ftpSearcher}\hlopt{\symbol{62}}\hlstd{} 
\mbox{}
\normalfont
\normalsize


fileInfoDir:存放FTP服务器搜集模块所搜集的FTP文件信息文本。

indexDir:存放Lucene索引的目录。

logDir:存放对比模块产生的文件对比文本。

username:后台管理系统用户名字符串的MD5值。

password:后台管理密码字符串的MD5值。

存放MD5值相较于原始值来说,更为安全,但是如果密码的原始字符串是常见密码,MD5值也并不安全。可以通过暴力破解来非法登陆系统。

1. 获取内容

以4.6.1中构建好的Document对象实例doc为基础,获取XML文件中的内容需要通过一下的方法:

(1)	使用Document类的getRootElement()方法获取了XML的根元素的Element对象实例。

(2)	Element类的getChildText("子元素名")来获取子元素的内容。

具体的核心代码如下:

\noindent
\ttfamily
\hlstd{\hllin{1\ }Element\ root\ }\hlopt{=\ }\hlstd{doc}\hlopt{.}\hlstd{}\hlkwd{getRootElement}\hlstd{}\hlopt{();}\\
\hllin{2\ }\hlstd{indexDir\ }\hlopt{=\ }\hlstd{root}\hlopt{.}\hlstd{}\hlkwd{getChildText}\hlstd{}\hlopt{(}\hlstd{}\hlstr{"indexDir"}\hlstd{}\hlopt{);}\\
\hllin{3\ }\hlstd{fileInfoDir\ }\hlopt{=\ }\hlstd{root}\hlopt{.}\hlstd{}\hlkwd{getChildText}\hlstd{}\hlopt{(}\hlstd{}\hlstr{"fileInfoDir"}\hlstd{}\hlopt{);}\\
\hllin{4\ }\hlstd{logDir\ }\hlopt{=\ }\hlstd{root}\hlopt{.}\hlstd{}\hlkwd{getChildText}\hlstd{}\hlopt{(}\hlstd{}\hlstr{"logDir"}\hlstd{}\hlopt{);}\\
\hllin{5\ }\hlstd{username\ }\hlopt{=\ }\hlstd{root}\hlopt{.}\hlstd{}\hlkwd{getChildText}\hlstd{}\hlopt{(}\hlstd{}\hlstr{"username"}\hlstd{}\hlopt{);}\\
\hllin{6\ }\hlstd{password\ }\hlopt{=\ }\hlstd{root}\hlopt{.}\hlstd{}\hlkwd{getChildText}\hlstd{}\hlopt{(}\hlstd{}\hlstr{"password"}\hlstd{}\hlopt{);}\\
\hllin{7\ }\hlstd{} 
\mbox{}
\normalfont
\normalsize


2. 修改内容

修改XML文件需要使用以下的类及其方法。

(1)使用root的root.getChild("子元素名")获取子元素的Element实例,如Element eIndexDir = root.getChild("indexDir")。

(2)使用Element类的setText(字符串)方法修改元素内容,如eIndexDir.setText(indexDir)。

(3)使用XMLOutputter().outputString(doc)方法将Document对象输出为一个字符串。

(4)创建FileWriter对象实例。使用FileWriter类的write(字符串)方法把修改结果写入文件。

具体的核心代码如下:

\noindent
\ttfamily
\hlstd{\hllin{1\ }Element\ eIndexDir\ }\hlopt{=\ }\hlstd{root}\hlopt{.}\hlstd{}\hlkwd{getChild}\hlstd{}\hlopt{(}\hlstd{}\hlstr{"indexDir"}\hlstd{}\hlopt{);}\\
\hllin{2\ }\hlstd{eIndexDir}\hlopt{.}\hlstd{}\hlkwd{setText}\hlstd{}\hlopt{(}\hlstd{indexDir}\hlopt{);}\\
\hllin{3\ }\hlstd{\\
\hllin{4\ }Element\ eFileInfoDir\ }\hlopt{=\ }\hlstd{root}\hlopt{.}\hlstd{}\hlkwd{getChild}\hlstd{}\hlopt{(}\hlstd{}\hlstr{"fileInfoDir"}\hlstd{}\hlopt{);}\\
\hllin{5\ }\hlstd{eFileInfoDir}\hlopt{.}\hlstd{}\hlkwd{setText}\hlstd{}\hlopt{(}\hlstd{fileInfoDir}\hlopt{);}\\
\hllin{6\ }\hlstd{\\
\hllin{7\ }Element\ eLogDir\ }\hlopt{=\ }\hlstd{root}\hlopt{.}\hlstd{}\hlkwd{getChild}\hlstd{}\hlopt{(}\hlstd{}\hlstr{"logDir"}\hlstd{}\hlopt{);}\\
\hllin{8\ }\hlstd{eLogDir}\hlopt{.}\hlstd{}\hlkwd{setText}\hlstd{}\hlopt{(}\hlstd{logDir}\hlopt{);}\\
\hllin{9\ }\hlstd{}\\
\hllin{10\ }\hlkwa{if\ }\hlstd{}\hlopt{(}\hlstd{password}\hlopt{.}\hlstd{}\hlkwd{length}\hlstd{}\hlopt{()\ !=\ }\hlstd{}\hlnum{0}\hlstd{}\hlopt{)\ \symbol{123}}\\
\hllin{11\ }\hlstd{}\hlstd{\ \ \ \ }\hlstd{Element\ ePassword\ }\hlopt{=\ }\hlstd{root}\hlopt{.}\hlstd{}\hlkwd{getChild}\hlstd{}\hlopt{(}\hlstd{}\hlstr{"password"}\hlstd{}\hlopt{);}\\
\hllin{12\ }\hlstd{}\hlstd{\ \ \ \ }\hlstd{ePassword}\hlopt{.}\hlstd{}\hlkwd{setText}\hlstd{}\hlopt{(}\hlstd{}\hlkwd{getMD5}\hlstd{}\hlopt{(}\hlstd{password}\hlopt{));}\\
\hllin{13\ }\hlstd{}\hlopt{\symbol{125}}\\
\hllin{14\ }\hlstd{\\
\hllin{15\ }String\ des\ }\hlopt{=\ }\hlstd{}\hlkwa{new\ }\hlstd{}\hlkwd{XMLOutputter}\hlstd{}\hlopt{().}\hlstd{}\hlkwd{outputString}\hlstd{}\hlopt{(}\hlstd{doc}\hlopt{);}\\
\hllin{16\ }\hlstd{FileWriter\ fileWriter\ }\hlopt{=\ }\hlstd{}\hlkwa{new\ }\hlstd{}\hlkwd{FileWriter}\hlstd{}\hlopt{(}\hlstd{file}\hlopt{);}\\
\hllin{17\ }\hlstd{fileWriter}\hlopt{.}\hlstd{}\hlkwd{write}\hlstd{}\hlopt{(}\hlstd{des}\hlopt{);}\\
\hllin{18\ }\hlstd{fileWriter}\hlopt{.}\hlstd{}\hlkwd{close}\hlstd{}\hlopt{();}\\
\hllin{19\ }\hlstd{} 
\mbox{}
\normalfont
\normalsize


\section{用户搜索日志记录}
\subsection{用户搜索日志的意义}
单个来看,每条搜索日志都是平淡无奇的,但是搜集每个用户、每次查询、每次应答的日志,将它们放在一起,就会成为一个内含丰富、深藏玄机的资料集。例如将一个用户在一个时间段内,针对某个搜索目标所进行的一系列查询和点击记录放在一块,就可以大致推测,对这样的用户、这样的需求,哪些结果是他们希望看到的,从而优化结果的排序。此外,综合众多用户的查询日志,可以知道哪些查询词是用户搜得最多的,哪些查询词是某一段时间内搜索量增长最快的,从而分析搜索的趋势。例如百度风云榜(http://top.baidu.com/),某个时期,百度用户最为关心的、搜索频率最高的事件,都在榜单里有所体现。榜单的生成就离不开千千万万平凡的搜索日志。同样,Google也有类似的资料发布在互联网上,在http://www.google.com/intl/en/press/zeitgeist/index.html可以看到相关的内容,顺便领略一下,在Google的用户群中,大家关心的是什么内容。

这些只是日志中所体现的非常基本的内在信息。对搜索引擎日志的分析,还能帮助开发者理解用户的查询意图、理解搜索结果的内容、评判搜索结果质量、改进搜索系统等一系列有用的事情,可以为人们不断改进获取有效信息的便捷性提供有力的帮助。

\subsection{记录用户搜索的方法}
在本系统中,用户搜索日志记录的模块作为一个Struts2架构中的拦截器来完成工作。原理如图\ref{image034}所示:

\pic[htpb]{拦截器原理}{width=0.8\textwidth}{image034}

拦截器将用户请求拦截下来,将与用户查询相关的信息写入数据库,然后再把请求交给Action处理。
代码实现如下:

\noindent
\ttfamily
\hlstd{}\hllin{1\ }\hlkwa{public\ }\hlstd{String\ }\hlkwd{intercept}\hlstd{}\hlopt{(}\hlstd{ActionInvocation\ actionInvocation}\hlopt{)\ }\Righttorque\\
\hllin{2\ }\hlstd{}\hlkwa{throws\ }\hlstd{Exception\ }\hlopt{\symbol{123}}\\
\hllin{3\ }\hlstd{}\hlstd{\ \ \ \ }\hlstd{UserQueryLog\ userQueryLog\ }\hlopt{=\ }\hlstd{}\hlkwa{new\ }\hlstd{}\hlkwd{UserQueryLog}\hlstd{}\hlopt{();}\\
\hllin{4\ }\hlstd{}\hlstd{\ \ \ \ }\hlstd{}\hlkwb{int\ }\hlstd{page\ }\hlopt{=\ }\hlstd{}\hlnum{1}\hlstd{}\hlopt{;}\\
\hllin{5\ }\hlstd{}\hlstd{\ \ \ \ }\hlstd{}\hlkwa{if\ }\hlstd{}\hlopt{(}\hlstd{ServletActionContext}\hlopt{.}\hlstd{}\hlkwd{getRequest}\hlstd{}\hlopt{().}\hlstd{}\hlkwd{getParameter}\hlstd{}\hlopt{(}\Righttorque\\
\hllin{6\ }\hlstd{}\hlstd{\ \ \ \ }\hlstd{}\hlstr{"page"}\hlstd{}\hlopt{)\ !=\ }\hlstd{null}\hlopt{)\ \symbol{123}}\\
\hllin{7\ }\hlstd{}\hlstd{\ \ \ \ \ \ \ \ }\hlstd{page\ }\hlopt{=\ }\hlstd{Integer}\hlopt{.}\hlstd{}\hlkwd{valueOf}\hlstd{}\hlopt{(}\hlstd{ServletActionContext}\hlopt{.}\Righttorque\\
\hllin{8\ }\hlstd{}\hlstd{\ \ \ \ \ \ \ \ }\hlstd{}\hlkwd{getRequest}\hlstd{}\hlopt{()}\\
\hllin{9\ }\hlstd{}\hlstd{\ \ \ \ \ \ \ \ }\hlstd{}\hlopt{.}\hlstd{}\hlkwd{getParameter}\hlstd{}\hlopt{(}\hlstd{}\hlstr{"page"}\hlstd{}\hlopt{));}\\
\hllin{10\ }\hlstd{}\hlstd{\ \ \ \ }\hlstd{}\hlopt{\symbol{125}}\\
\hllin{11\ }\hlstd{}\hlstd{\ \ \ \ }\hlstd{userQueryLog}\hlopt{.}\hlstd{}\hlkwd{setQuerypage}\hlstd{}\hlopt{(}\hlstd{page}\hlopt{);}\\
\hllin{12\ }\hlstd{}\hlstd{\ \ \ \ }\hlstd{userQueryLog}\hlopt{.}\hlstd{}\hlkwd{setQuerytime}\hlstd{}\hlopt{(}\hlstd{}\hlkwa{new\ }\hlstd{}\hlkwd{Date}\hlstd{}\hlopt{(}\hlstd{System}\hlopt{.}\Righttorque\\
\hllin{13\ }\hlstd{}\hlstd{\ \ \ \ }\hlstd{}\hlkwd{currentTimeMillis}\hlstd{}\hlopt{()));}\\
\hllin{14\ }\hlstd{}\hlstd{\ \ \ \ }\hlstd{userQueryLog}\hlopt{.}\hlstd{}\hlkwd{setQueryword}\hlstd{}\hlopt{(}\hlstd{ServletActionContext}\hlopt{.}\Righttorque\\
\hllin{15\ }\hlstd{}\hlstd{\ \ \ \ }\hlstd{}\hlkwd{getRequest}\hlstd{}\hlopt{()}\\
\hllin{16\ }\hlstd{}\hlstd{\ \ \ \ }\hlstd{}\hlopt{.}\hlstd{}\hlkwd{getParameter}\hlstd{}\hlopt{(}\hlstd{}\hlstr{"keywords"}\hlstd{}\hlopt{));}\\
\hllin{17\ }\hlstd{\\
\hllin{18\ }}\hlstd{\ \ \ \ }\hlstd{}\hlkwa{try\ }\hlstd{}\hlopt{\symbol{123}}\\
\hllin{19\ }\hlstd{}\hlstd{\ \ \ \ \ \ \ \ }\hlstd{String\ resource\ }\hlopt{=\ }\hlstd{}\hlstr{"/mybatis{-}config.xml"}\hlstd{}\hlopt{;}\\
\hllin{20\ }\hlstd{}\hlstd{\ \ \ \ \ \ \ \ }\hlstd{InputStream\ inputStream\ }\hlopt{=\ }\hlstd{Resources}\hlopt{.}\Righttorque\\
\hllin{21\ }\hlstd{}\hlstd{\ \ \ \ \ \ \ \ }\hlstd{}\hlkwd{getResourceAsStream}\hlstd{}\hlopt{(}\hlstd{resource}\hlopt{);}\\
\hllin{22\ }\hlstd{}\hlstd{\ \ \ \ \ \ \ \ }\hlstd{SqlSessionFactory\ sqlSessionFactory\ }\hlopt{=\ }\hlstd{}\hlkwa{new\ }\Righttorque\\
\hllin{23\ }\hlstd{}\hlstd{\ \ \ \ \ \ \ \ }\hlstd{}\hlkwd{SqlSessionFactoryBuilder}\hlstd{}\hlopt{()}\\
\hllin{24\ }\hlstd{}\hlstd{\ \ \ \ \ \ \ \ }\hlstd{}\hlopt{.}\hlstd{}\hlkwd{build}\hlstd{}\hlopt{(}\hlstd{inputStream}\hlopt{);}\\
\hllin{25\ }\hlstd{}\hlstd{\ \ \ \ \ \ \ \ }\hlstd{SqlSession\ session\ }\hlopt{=\ }\hlstd{sqlSessionFactory}\hlopt{.}\hlstd{}\hlkwd{openSession}\hlstd{}\hlopt{();}\\
\hllin{26\ }\hlstd{\\
\hllin{27\ }}\hlstd{\ \ \ \ \ \ \ \ }\hlstd{UserQueryLogMapper\ userQueryLogMapper\ }\hlopt{=\ }\hlstd{session\\
\hllin{28\ }}\hlstd{\ \ \ \ \ \ \ \ }\hlstd{}\hlopt{.}\hlstd{}\hlkwd{getMapper}\hlstd{}\hlopt{(}\hlstd{UserQueryLogMapper}\hlopt{.}\hlstd{}\hlkwa{class}\hlstd{}\hlopt{);}\\
\hllin{29\ }\hlstd{}\hlstd{\ \ \ \ \ \ \ \ }\hlstd{userQueryLogMapper}\hlopt{.}\hlstd{}\hlkwd{inserUserQueryLog}\hlstd{}\hlopt{(}\hlstd{userQueryLog}\hlopt{);}\\
\hllin{30\ }\hlstd{\\
\hllin{31\ }}\hlstd{\ \ \ \ \ \ \ \ }\hlstd{session}\hlopt{.}\hlstd{}\hlkwd{commit}\hlstd{}\hlopt{();}\\
\hllin{32\ }\hlstd{}\hlstd{\ \ \ \ \ \ \ \ }\hlstd{session}\hlopt{.}\hlstd{}\hlkwd{close}\hlstd{}\hlopt{();}\\
\hllin{33\ }\hlstd{}\hlstd{\ \ \ \ }\hlstd{}\hlopt{\symbol{125}\ }\hlstd{}\hlkwa{catch\ }\hlstd{}\hlopt{(}\hlstd{Exception\ e}\hlopt{)\ \symbol{123}}\\
\hllin{34\ }\hlstd{}\hlstd{\ \ \ \ \ \ \ \ }\hlstd{e}\hlopt{.}\hlstd{}\hlkwd{printStackTrace}\hlstd{}\hlopt{();}\\
\hllin{35\ }\hlstd{}\hlstd{\ \ \ \ }\hlstd{}\hlopt{\symbol{125}}\\
\hllin{36\ }\hlstd{}\hlstd{\ \ \ \ }\hlstd{}\hlkwa{return\ }\hlstd{actionInvocation}\hlopt{.}\hlstd{}\hlkwd{invoke}\hlstd{}\hlopt{();}\\
\hllin{37\ }\hlstd{}\hlopt{\symbol{125}}\\
\hllin{38\ }\hlstd{}\hlopt{\symbol{125}}\\
\hllin{39\ }\hlstd{} 
\mbox{}
\normalfont
\normalsize


为使得拦截器能够生效,在struts2配置文件struts.xml中编写代码如下:

\noindent
\ttfamily
\hlstd{}\hllin{1\ }\hlkwa{\symbol{60}action\ }\hlstd{}\hlkwb{name}\hlstd{=}\hlstr{"search"}\hlstd{\ }\hlkwb{class}\hlstd{=}\hlstr{"ftpSearcher.action.FTPSearch"}\hlstd{}\hlkwa{\symbol{62}}\\
\hllin{2\ }\hlstd{}\hlstd{\ \ \ \ }\hlstd{}\hlkwa{\symbol{60}interceptor{-}ref\ }\hlstd{}\hlkwb{name}\hlstd{=}\hlstr{"defaultStack"}\hlstd{\ }\hlkwa{/\symbol{62}}\\
\hllin{3\ }\hlstd{}\hlstd{\ \ \ \ }\hlstd{}\hlkwa{\symbol{60}interceptor{-}ref\ }\hlstd{}\hlkwb{name}\hlstd{=}\hlstr{"userQueryLoginterceptor"}\hlstd{\ }\hlkwa{/\symbol{62}}\\
\hllin{4\ }\hlstd{}\hlstd{\ \ \ \ }\hlstd{}\hlkwa{\symbol{60}result\ }\hlstd{}\hlkwb{name}\hlstd{=}\hlstr{"success"}\hlstd{}\hlkwa{\symbol{62}}\hlstd{/results.jsp}\hlkwa{\symbol{60}/result\symbol{62}}\\
\hllin{5\ }\hlstd{}\hlkwa{\symbol{60}/action\symbol{62}}\hlstd{} 
\mbox{}
\normalfont
\normalsize
