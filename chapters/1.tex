% !Mode:: "TeX:UTF-8"

\chapter{引言}
\section{研究背景及意义}
FTP服务器是Internet上所使用的最主要的服务器之一,在FTP服务器上保存有大量的共享软件、技术资料、多媒体数据等各种各样的文件\citeup{1}。每个FTP服务器都有其各自的若干目录,且其目录结构往往都比较的复杂,所以要在FTP服务器找到所需的特定资源时,用户往往需要一个个目录的查看,这并不是一件容易的事情,而要在多个FTP服务器上查找文件则更加困难。FTP搜索引擎可以很好的解决上述的问题。通过FTP搜索引擎将各个服务器的文件信息整合在一起,用户就可以通过FTP搜索引擎快速的找到自己想要的资料。

近几年来,随着Web搜索引擎技术的不断发展成熟,人们对FTP搜索引擎进行了一定的研究和探讨,已经有成果投入使用。目前研究的热点主要集中于FTP搜索引擎体系结构的设计,并借助一些开发环境提出了一些简单的搜索原型\citeup{1}:基于PHP和MySQL数据库的简单FTP搜索引擎,该原型直接采用SQL语言查询数据库,并返回结果。优点是设计成本低;缺点是执行效率不是很高,不支持分词。

基于XML Web Service的FTP搜索技术,该方案大胆提出了一种新的FTP搜索引擎解决方案,通过在FTP服务器和检索服务器分别部署Web服务,FTP服务器向检索服务器提供符合XML标准的实时FTP文件信息,可以保证检索结果的时效性,基于Web服务的系统框架为大范围的分布式部署建立了良好的基础\citeup{2}。但在现行的网络环境下,大范围的在FTP服务器部署Web服务较为困难。

基于IA(Intelligent Agent,集成代理)和Web Service的FTP搜索引擎设计\citeup{3}。该设计方案主要采用在一个自治系统内部(这个系统的划分不一定以单位实体划分,可以使多个互联速度较快的网络一起构成一个自治系统,如中国教育科研网就可以使一个大的自治系统,其内部各成员单位的互联速度是比较快的)部署Search Server,并启动信息搜集Agent,信息处理Agent和信息通信Agent,信息搜集Agent的功能由爬虫(Spider)扮演,各个自治系统Search Server交互各自管辖范围的数据搜集,呈现Peer to Peer的架构。Search Server提供统一的用户接口,用户用客户端软件登陆FTP Search,也可以使用Web浏览器登陆任意的Search Server,由Search Server为用户指派最近的节点。最大的优点是在搜索的数据量非常大时候,这种设计仍有高效的检索效率,维持着较高的性能。缺点是:实现的技术难度较高。

基于Lucene的FTP搜索引擎的设计。该方案针对目前中文FTP搜索引擎不支持中文分词及动态数据更新的缺陷,采用Lucene这个最大的全文搜索引擎工具包构建FTP搜索引擎,该搜索引擎能够生成标准的XML数据文档,并采用基于字典的前向最大匹配中文分词法在Lucene中动态更新全文索引,该引擎还可以对中英文混合分析和检索\citeup{4}。该方案更适合为校园内部提供FTP搜索服务。优点:搜索引擎的系统开发速度快。缺点:在数据量非常大时,性能会受到影响。
\section{本文的主要工作}
本文在研究了搜索引擎相关实现技术的基本原理、处理流程的基础上,结合对FTP搜索引擎的具体需求分析,设计并实现了基于Lucene的FTP搜索引擎。下面是本文的主要工作:

在需求分析阶段,从系统功能需求和非系统功能需求两个方面对FTP搜索引擎的需求进行描述。确定了需要实现如下功能:

1. 搜索功能。在Web前端为用户提供查询功能,能够使用户快捷方便的找到自己指定的资料,支持模糊搜索。

2. 数据采集、更新功能。从FTP服务器采集文件信息,并以一定的策略对数据库进行更新。

3. 数据处理功能。基于Lucene设计索引的数据结构和索引方式,对采集到的文件信息进行索引,使得搜索更为迅速快捷。

4. 后台管理功能。对FTP搜索引擎搜集的FTP服务器信息、文件信息、更新策略等做相关的管理。

在总体设计阶段,根据需求分析,设计并描述了系统的整体功能和总体流程,然后根据系统的特点将其划分为四个主要模块 。这四个主要模块是:数据搜集模块、文件信息对比模块、数据索引模块、搜索模块。并分别对每个模块进行分析和设计。

在详细设计阶段,本文具体描述了各个功能模块的设计。并给出了关键模块的实现代码。
\section{论文结构及安排}
本文内容共分六章,章节结构安排如下:

第一章是绪论。主要阐述了FTP搜索引擎系统的研究背景和意义、技术的发展现状以及本文的主要工作和组织结构等内容。

第二章是相关技术及其核心原理。主要对实现FTP搜索引擎所使用到相关技术进行了介绍,并研究了它们的工作原理。

第三章是系统需求分析及总体设计。根据搜索引擎的工作原理结合功能性和性能需求分析,建立系统的基本功能架构。

第四章是系统详细设计与实现。根据需求分析和总体设计,提出了具体的实现方案,描述了本文具体描述了各个功能模块的设计,并详细描述了程序的实现。

第五章是系统测试与运行。首先介绍了整个系统的实现状况,对系统进行了测试,给出了系统的测试结果。

第六章是结束语。对全文内容进行了总结,指出系统待完善之处,并对需要进一步研究和解决的问题做了展望,阐述了下一步需要进行的工作。
