\chapter{结束语}
\section{总结}
搜索引擎作为Web浏览的主要工具,已经受到业界越来越广泛的关注。整个\\FTP搜索引擎的实现是基于Lucene全文检索工具包与Spider程序来完成数据的搜索工作,使用Struts2框架来实现B/S架构程序开发,使用Mybatis作为持久层,完成与数据库的交互工作。
本文的工作可归纳为4个方面:

1.	研究并阐述了FTP相关技术的核心原理。

2.	基于Lucene完成了对源数据索引的改进、构建与中文分词的设计。

3.  以软件工程的设计为顺序,介绍了整个项目从零到整的开发过程。

4.  重点介绍了核心模块的实现方法。

\section{展望}
为了进一步完善搜索引擎系统,本课题还需要在多个方面进一步进行研究。

1. 关键词自动纠错:有些时候,用户输入的关键词会出现拼写错误,例如“哈利波特”拼写成“哈力波特”,当用户输入“哈力波特”,能够向用户提示“哈利波特”。

2. 多线程的数据搜集模块:对于一个普通的FTP服务器来说,遍历并获取整个服务器的文件信息,耗时会受到数据量及网络状况等因素的影响。当需要遍历的FTP服务器数量增多时,时间会相对较长。通过多线程的方式,同时遍历同一服务器的不同目录或同时遍历不同的FTP服务器可以缩短数据采集的时间,需要好的策略去设计、实现、管理并发的搜集过程。

3. 性能优化:当网站访问量变大后,整个系统的性能可能会遇到瓶颈,构建分布式的搜索引擎可以解决瓶颈问题,提高搜索的速度。

4.界面优化:应该重新设计用户界面,使得其更加美观。